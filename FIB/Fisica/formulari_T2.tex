\documentclass[12pt]{article}
 
\usepackage[margin=0.8in]{geometry}
\usepackage[pdftex]{hyperref}
\usepackage{amsmath,amsthm,amssymb,mathtools,hyperref,enumerate,multicol,mdframed}

\newmdenv[leftline=false,topline=false]{topright}
\let\proof\relax
\usepackage[utf8]{inputenc}
 
\begin{document}
\begin{multicols}{2}
\subsection*{Circuits RC}

\begin{tabular}{|c|c|c|c|}
    \hline
    Càrrega & Descàrrega \\ \hline \hline
    $q(t) = q(0)\left( 1 - e^{-\frac{t}{\tau_C}}\right)$ & $q(t) = q(0)e^{-\frac{t}{\tau_C}}$ \\ \hline
    $I(t) = \frac{\epsilon}{R} e^{-\frac{t}{\tau_C}}$ & $I(t) = -\frac{V}{R}e^{-\frac{t}{\tau_C}}$ \\ \hline
\end{tabular}
\\

$\tau_C = RC$, $q(0) = VC$

\subsubsection*{Solenoides}

\underline{Flux}: $\Phi = NBS = \frac{\mu_0 N^2SI}{l}$ \\
\underline{Coeficient d'autoinducció}: $\frac{\Phi}{I} = \frac{\mu_0N^2S}{l}$ \\
$\epsilon_L= -\frac{\text{d}\Phi}{\text{d}t} = -L \frac{\text{d}I}{\text{d}t}$

\subsection*{Circuits RL}

\begin{tabular}{|c|c|c|c|}
    \hline
    Càrrega & Descàrrega \\ \hline \hline
    $I(t) = \frac{\epsilon}{R} \left(1 - e^{-\frac{t}{\tau_L}}\right)$ & $I(t) = \frac{V}{R}e^{-\frac{t}{\tau_L}}$ \\ \hline
\end{tabular} \\ \\
$\tau_L = \frac{L}{R}$

\subsection*{Corrent alterna}
\underline{f.e.m. alterna}: $V(t) = V_0\text{cos}(\omega t+\varphi)$, $T = \frac{2\pi}{\omega}$, $I(t) = \frac{V(t)}{R} = \frac{V_0}{R}\text{cos}(\omega t + \varphi) = I_0\text{cos}(\omega t + \varphi)$ \\
\underline{Flux}: $\Phi = BSN\text{cos}(\omega t+\theta)$, $B$ camp magnètic \\
\underline{Llei Faraday}: $\epsilon(t) = V_0\text{sin}(\omega t + \theta_0)$ \\
\underline{Voltatge eficaç}: $V_{ef} = \frac{V_0}{\sqrt{2}}$ \\
\underline{Intensitat eficaç}: $I_{ef} = \frac{I_0}{\sqrt{2}}$

\subsubsection*{Circuit amb condensador}
\underline{Voltatge}: $V(t) = V_0\text{cos}(\omega t)$ \\
\underline{Intensitat}: $I(t) = -V_0\omega C \text{sin}(\omega t) = -I_0\text{sin}(\omega t)$ $= I_0\text{cos}(\omega t + \frac{\pi}{2})$ (desfase de $\frac{\pi}{2}$) \\
Sigui $V(t) = V_0 e^{i\omega t}$, llavors, $I(t) = V_0i\omega Ce^{i\omega t}$. Podem reproduir la llei d'Ohm $(V=IR_C)$, $R_C = \frac{1}{i\omega C}$.\\
\underline{Reactancia capacitiva}: $X_C = \vert R_C\vert = \frac{1}{\omega C}$, $R_C = \frac{X_C}{i} = -iX_C$

\subsubsection*{Circuit amb inducció}
\underline{Voltatge}: $V(t) = V_0\text{cos}(\omega t)$ \\
\underline{Autoinducció a la bobina}: $\varepsilon_L = -L \frac{\text{d}I}{\text{d}t}$ \\
\underline{Segona llei Kirchhoff}: $V(t) + \varepsilon_L = 0 \implies I(t) = \frac{V_0}{L\omega}\text{sin}(\omega t) = I_0\text{cos}(\omega t - \frac{\pi}{2})$ (desfase de $\frac{\pi}{2}$) \\
Sigui $V(t) = V_0e^{i\omega t}$, llavors, $I = \frac{V_0}{i\omega L}e^{i\omega t}$. Podem reproduir la llei d'Ohm $V=IR_L$, $R_L=i\omega L$. \\
\underline{Reactancia inductiva}: $X_L = \vert R_L\vert = \omega L$, $R_L = iX_L$

\subsection*{Impedància. Llei d'Ohm}

\underline{Llei d'Ohm}: $V = IZ$ \\
\underline{Impedància}: $\bar{Z} = R + iX \begin{cases} \text{Resistència: } R \\ \text{Condensador: } -iX_C \\ \text{Inducció: } iX_L \end{cases}$

\subsubsection*{Circuit LCR}

\underline{Angle de fase}: tg$(\varphi)=\frac{X_L-X_C}{R}$ \\
\underline{Corrent máxim}: $I_0 = \frac{\varepsilon_0}{Z}$

\subsection*{Potència}

\underline{Potència instantània}: $P(t) = V(t)I(t) = V_0I_0\text{cos}(\omega t)\text{cos}(\omega t - \varphi)$ \\
\underline{Potència mitja}: $\frac{V_0I_0}{2\text{cos}(\varphi)} = V_{ef}I_{ef}\text{cos}(\varphi)$

\subsubsection*{Potència en una resistència}

\underline{Potència instanània}: $P(t) = V_0\text{cos}(\omega t)I_0\text{cos}(\omega t) = \frac{V_0^2}{R}\text{cos}^2(\omega t)$ \\
\underline{Potència mitja}: $P = \frac{V_0^2}{2R}$ \\
\underline{Valors eficaços}: $V_{ef} = \frac{V_0}{\sqrt{2}}$, $I_{ef} = \frac{I_0}{\sqrt{2}}$ \\
\underline{Potència dissipada}: $P = \frac{V_{ef}^2}{R} = RI_{ef}^2$

\section*{Pàgina 59 feta}
\end{multicols}
\end{document}
