\documentclass[12pt]{article}
 
\usepackage[margin=0.8in]{geometry}
\usepackage[pdftex]{hyperref}
\usepackage{amsmath,amsthm,amssymb,graphicx,mathtools,tikz,hyperref,enumerate}
\usepackage{mdframed,cleveref,cancel,stackengine,multicol}

\newmdenv[leftline=false,topline=false]{topright}
\let\proof\relax
\usepackage[utf8]{inputenc}
\usetikzlibrary{positioning}
\newcommand{\n}{\mathbb{N}}
\newcommand{\z}{\mathbb{Z}}
\newcommand{\q}{\mathbb{Q}}
\newcommand{\cx}{\mathbb{C}}
\newcommand{\real}{\mathbb{R}}
\newcommand{\E}{\mathbb{E}}
\newcommand{\F}{\mathbb{F}}
\newcommand{\bb}[1]{\mathbb{#1}}
\let\k\relax
\newcommand{\k}{\mathbf{k}}
\newcommand{\ita}[1]{\textit{#1}}
\newcommand\inv[1]{#1^{-1}}
\newcommand\setb[1]{\left\{#1\right\}}
\newcommand{\vbrack}[1]{\langle #1\rangle}
\newcommand{\determinant}[1]{\begin{vmatrix}#1\end{vmatrix}}
\newcommand{\abs}[1]{\left\vert #1 \right\vert}
\newcommand{\Po}{\mathbb{P}}
\DeclareMathOperator{\Id}{Id}
\DeclareMathOperator{\rg}{rang}
\DeclareMathOperator{\car}{car}
\DeclareMathOperator{\im}{Im}


\hypersetup{
	colorlinks,
	linkcolor=blue
}
 
 \renewcommand*\contentsname{Contenidos}

\newtheoremstyle{break}% name
{}%         Space above, empty = `usual value'
{}%         Space below
{}% Body font
{}%         Indent amount (empty = no indent, \parindent = para indent)
{\bfseries}% Thm head font
{}%        Punctuation after thm head
{\newline}% Space after thm head: \newline = linebreak
{#1 #2 \normalfont #3}%         Thm head spec

\newtheoremstyle{breakthm}% name
{}%         Space above, empty = `usual value'
{}%         Space below
{}% Body font
{}%         Indent amount (empty = no indent, \parindent = para indent)
{\bfseries}% Thm head font
{}%        Punctuation after thm head
{\newline}% Space after thm head: \newline = linebreak
{#1 \normalfont #3 (#2)\addcontentsline{toc}{subsubsection}{#1 #3}}%         Thm head spec
\newtheoremstyle{normal}% name
{}%         Space above, empty = `usual value'
{}%         Space below
{}% Body font
{}%         Indent amount (empty = no indent, \parindent = para indent)
{\bfseries}% Thm head font
{}%        Punctuation after thm head
{5pt plus 1pt minus 1pt}% Space after thm head: \newline = linebreak
{#1 #2 \normalfont #3}%         Thm head spec

\theoremstyle{normal}
\newtheorem{lema}{Lema}[subsection]
\newtheorem{obs}[lema]{Observación}

\theoremstyle{break}
\newtheorem{prop}[lema]{Proposición}
\newtheorem*{proof}{Demostración}
\newtheorem{defi}[lema]{Definición}
\newtheorem{col}[lema]{Corolario}
\newtheorem{ej}[lema]{Ejercicio}
\newtheorem{example}[lema]{Ejemplo}

\theoremstyle{breakthm}
\newtheorem{thm}[lema]{Teorema}

\title{Formulari T2}
\author{Oscar Benedito i Saura}
\date{Octubre 2017}
 
\begin{document}
\begin{multicols}{2}
\subsection*{Circuits RC}

\begin{tabular}{|c|c|c|c|}
    \hline
    Càrrega & Descàrrega \\ \hline \hline
    $q(t) = q(0)\left( 1 - e^{-\frac{t}{\tau_C}}\right)$ & $q(t) = q(0)e^{-\frac{t}{\tau_C}}$ \\ \hline
    $I(t) = \frac{\epsilon}{R} e^{-\frac{t}{\tau_C}}$ & $I(t) = -\frac{V}{R}e^{-\frac{t}{\tau_C}}$ \\ \hline
\end{tabular}
\\

$\tau_C = RC$, $q(0) = VC$

\subsection*{Solenoides}

\underline{Flux}: $\Phi = NBS = \frac{\mu_0 N^2SI}{l}$ \\
\underline{Coeficient d'autoinducció}: $\frac{\Phi}{I} = \frac{\mu_0N^2S}{l}$ \\
$\epsilon_L= -\frac{\text{d}\Phi}{\text{d}t} = -L \frac{\text{d}I}{\text{d}t}$

\subsection*{Circuits RL}

\begin{tabular}{|c|c|c|c|}
    \hline
    Càrrega & Descàrrega \\ \hline \hline
    $I(t) = \frac{\epsilon}{R} \left(1 - e^{-\frac{t}{\tau_L}}\right)$ & $I(t) = \frac{V}{R}e^{-\frac{t}{\tau_L}}$ \\ \hline
\end{tabular} \\ \\
$\tau_L = \frac{L}{R}$

\subsection*{Corrent alterna}
\underline{f.e.m. alterna}: $V(t) = V_0\text{cos}(\omega t+\varphi)$, $T = \frac{2\pi}{\omega}$, $I(t) = \frac{V(t)}{R} = \frac{V_0}{R}\text{cos}(\omega t + \varphi) = I_0\text{cos}(\omega t + \varphi)$ \\
\underline{Flux}: $\Phi = BSN\text{cos}(\omega t+\theta)$, $B$ camp magnètic \\
\underline{Ley Faraday}: $\epsilon(t) = V_0\text{sin}(\omega t + \theta_0$ \\
\underline{Voltatge eficaç}: $V_{ef} = \frac{V_0}{\sqrt{2}}$ \\
\underline{Intensitat eficaç}: $I_{ef} = \frac{I_0}{\sqrt{2}}$

\subsubsection*{Circuit amb condensador}
\underline{Voltatge}: $V(t) = V_0\text{cos}(\omega t)$ \\
\underline{Intensitat}: $I(t) = -V_0\omega C \text{sin}(\omega t) = -I_0\text{sin}(\omega t)$ $= I_0\text{cos}(\omega t + \frac{\pi}{2})$ (desfase de $\frac{\pi}{2}$) \\
Sigui $V(t) = V_0 e^{i\omega t}$, llavors, $I(t) = V_0i\omega Ce^{i\omega t}$. Podem reproduir la llei d'Ohm $(V=IR_C)$, $R_C = \frac{1}{i\omega C}$.\\
\underline{Reactancia capacitiva}: $X_C = \vert R_C\vert = \frac{1}{\omega C}$, $R_C = \frac{X_C}{i} = -iX_C$

\subsubsection*{Circuit amb inducció}
\underline{Voltatge}: $V(t) = V_0\text{cos}(\omega t)$ \\
\underline{Autoinducció a la bobina}: $\varepsilon_L = -L \frac{\text{d}I}{\text{d}t}$ \\
\underline{Segona llei Kirchhoff}: $V(t) + \varepsilon_L = 0 \implies I(t) = \frac{V_0}{L\omega}\text{sin}(\omega t) = I_0\text{cos}(\omega t - \frac{\pi}{2})$ (desfase de $\frac{\pi}{2}$) \\
Sigui $V(t) = V_0e^{i\omega t}$, llavors, $I = \frac{V_0}{i\omega L}e^{i\omega t}$. Podem reproduir la llei d'Ohm $V=IR_L$, $R_L=i\omega L$. \\
\underline{Reactancia inductiva}: $X_L = \vert R_L\vert = \omega L$, $R_L = iX_L$

\subsection*{Impedància. Llei d'Ohm}



\end{multicols}
\end{document}
