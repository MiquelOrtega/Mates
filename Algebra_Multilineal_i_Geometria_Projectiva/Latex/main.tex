\documentclass[12pt]{article}
 
\usepackage[margin=1in]{geometry}
\usepackage[pdftex]{hyperref}
\usepackage{amsmath,amsthm,amssymb,graphicx,mathtools,tikz,hyperref,enumerate}
\usepackage{mdframed,cleveref,cancel,stackengine,pgfplots,pgf,mathrsfs}
\usepackage{xfrac,stmaryrd,commath}
\usepackage[spanish]{babel}

\newmdenv[leftline=false,topline=false]{topright}
\let\proof\relax
\usepackage[utf8]{inputenc}
\usetikzlibrary{positioning,arrows, calc}
\usetikzlibrary{external}
\tikzexternalize[prefix=figures/]
\pgfplotsset{compat=1.11}
\newcommand{\n}{\mathbb{N}}
\newcommand{\z}{\mathbb{Z}}
\newcommand{\q}{\mathbb{Q}}
\newcommand{\cx}{\mathbb{C}}
\newcommand{\real}{\mathbb{R}}
\newcommand{\E}{\mathbb{E}}
\newcommand{\F}{\mathbb{F}}
\newcommand{\R}{\mathcal{R}}
\newcommand{\bb}[1]{\mathbb{#1}}
\let\k\relax
\newcommand{\k}{\mathbf{k}}
\newcommand{\ita}[1]{\textit{#1}}
\newcommand\inv[1]{#1^{-1}}
\newcommand\setb[1]{\left\{#1\right\}}
\newcommand{\vbrack}[1]{\langle #1\rangle}
\newcommand{\determinant}[1]{\begin{vmatrix}#1\end{vmatrix}}
\newcommand{\Po}{\mathbb{P}}
\DeclareMathOperator{\Id}{Id}
\DeclareMathOperator{\rg}{rg}
\DeclareMathOperator{\car}{car}
\DeclareMathOperator{\im}{Im}
\let\emptyset\varnothing


\hypersetup{
	colorlinks,
	linkcolor=blue
}
 
 \renewcommand*\contentsname{Contenidos}

\newtheoremstyle{break}% name
{}%         Space above, empty = `usual value'
{}%         Space below
{}% Body font
{}%         Indent amount (empty = no indent, \parindent = para indent)
{\bfseries}% Thm head font
{}%        Punctuation after thm head
{\newline}% Space after thm head: \newline = linebreak
{#1 #2 \normalfont #3}%         Thm head spec

\newtheoremstyle{breakthm}% name
{}%         Space above, empty = `usual value'
{}%         Space below
{}% Body font
{}%         Indent amount (empty = no indent, \parindent = para indent)
{\bfseries}% Thm head font
{}%        Punctuation after thm head
{\newline}% Space after thm head: \newline = linebreak
{#1 \normalfont #3 (#2)\addcontentsline{toc}{subsubsection}{#1 #3}}%         Thm head spec
\newtheoremstyle{normal}% name
{}%         Space above, empty = `usual value'
{}%         Space below
{}% Body font
{}%         Indent amount (empty = no indent, \parindent = para indent)
{\bfseries}% Thm head font
{}%        Punctuation after thm head
{5pt plus 1pt minus 1pt}% Space after thm head: \newline = linebreak
{#1 #2 \normalfont #3}%         Thm head spec

\theoremstyle{normal}
\newtheorem{lema}{Lema}[subsection]
\newtheorem{obs}[lema]{Observación}
\newtheorem{rec}[lema]{Recordatorio}

\theoremstyle{break}
\newtheorem{prop}[lema]{Proposición}
\newtheorem*{proof}{Demostración}
\newtheorem{defi}[lema]{Definición}
\newtheorem{col}[lema]{Corolario}
\newtheorem{ej}[lema]{Ejercicio}
\newtheorem{example}[lema]{Ejemplo}

\theoremstyle{breakthm}
\newtheorem{thm}[lema]{Teorema}



 
\begin{document}
\date{}

\title{Álgebra Multilineal y Geometría Proyectiva}
 
\maketitle

\tableofcontents

\setcounter{section}{-1}


\section{Formas Cuadráticas}

\subsection{Definición, matriz de una forma cuadrática y bases}

\begin{defi}
	Sea $\E$ un $\k$-e.v. Diremos que una aplicación
	\[
		\begin{aligned}
			\phi \colon \E \times \E &\to \k \\
			(u,v) &\mapsto \phi(u,v)
		\end{aligned}
	\]
	es una forma bilineal simétrica si \begin{itemize}
		\item $\phi(u_1 + u_2, v) = \phi(u_1, v) + \phi(u_2, v)$
		\item $\phi(\lambda u, v) = \lambda\phi(u,v)$
		\item $\phi(u,v) = \phi(v,u)$
	\end{itemize}
	$\forall u,v,u_1,u_2 \in \E$ y $\forall \lambda \in \k$.
\end{defi}
\begin{defi}
	Sea $\phi$ una forma bilineal simétrica sobre un $\k$-e.v. $\E$.
	Diremos que la aplicación
	\[
		\begin{aligned}
			q \colon \E &\to \k \\
			u &\mapsto q(u) = \phi(u,u)
		\end{aligned}
	\]
	es la forma cuadrática asociada a $\phi$.
\end{defi}
\begin{obs}
	Se cumple que $q(\lambda u) = \lambda^2 q(u)$
\end{obs}
\begin{lema}
	Sea $\phi$ una forma bilineal simétrica sobre un $\k$-e.v. $\E$ con $\car \k \neq 2$
	y sea $q$ la forma cuadrática asociada a $\phi$, entonces
	\[
		\phi(u,v) = \frac{1}{2} ( q(u+v) - q(u) - q(v))
	\]
\end{lema}
\begin{proof}
	\[
	\begin{split}
		q(u+v) - q(u) - q(v) = \phi(u+v,u+v) - \phi(u,u) - \phi(v,v) = \\
		=\phi(u,u) + \phi(u,v) + \phi(v,u)  + \phi(v,v) - \phi(u,u) - \phi(v,v) =
		2 \phi(u,v)
	\end{split}
	\]
\end{proof}
\begin{defi}
	Sea $\phi$ una forma bilineal simétrica/cuadrática sobre un $\k$-e.v. $\E$ y sea
	$B = \{u_1, \cdots, u_n \}$ una base. La matriz de $\phi$ en base $B$ es
	\[
		M_B(\phi) = (a_{ij}) = (\phi(u_i,u_j))
	\]
\end{defi}
\begin{obs}
	La matriz $M_B(\phi)$ es simétrica
\end{obs}
\begin{defi}
	Sea $\E$ un $\k$-e.v. y sea $\phi \colon \E \times \E \to \k$ una
	forma bilineal simétrica.
	\begin{itemize}
		\item Diremos que $\phi$ es definida positiva si 
		\[ \phi(x,x) > 0, \quad \forall x \in \E
		\quad x \neq \vec{0} \]
		\item Diremos que $\phi$ es definida negativa si
		\[ \phi(x,x) < 0, \quad \forall x \in \E
		\quad x \neq \vec{0} \]
		\item Diremos que $\phi$ es no definida en cualquier otro caso.
	\end{itemize}
\end{defi}
\begin{obs}
	Si $\phi$ es una forma bilineal simétrica y definida positiva
	entonces define un producto escalar sobre $\E$.
\end{obs}
\begin{defi}
	Dada una matriz cuadrada $A$ (dim $n$) definimos
	\[
		A_k = (a_{ij}), \quad 1 \leq i, j \leq k \quad \text{y}
		\quad \delta_k(A) = \determinant{A_k}
	\]
\end{defi}
\begin{thm}[de Sylvester]
	Sea $\E$ un $\k$-e.v. de dimension $n$ y sea
	$\phi \colon \E\times \E \to \k$ una forma bilineal simétrica,
	entonces
	\[
	\phi \text{ es definida positiva }\iff \delta_k(M_B(\phi)) > 0
	\quad\forall 1 \leq k \leq n \quad\forall B \text{ base de } \E
	\]
\end{thm}
\begin{proof}
	$\implies$
	
	Como $\phi$ es definida positiva, define un producto escalar
	sobre $\E$. Si tomamos una base $B$ cualquiera, mediante
	Gramm-Schmidt podemos construir una base ortogonal
	$B_2 = \{v_1, \cdots, v_n \}$. Por tanto
	\[
		i \neq j \implies \phi(v_i, v_j) = 0, \quad \phi(v_i,v_i) 
		> 0 \quad 1 \leq i, j \leq n
	\]
	Llamamos $\phi(v_i, v_i) = \lambda_i > 0$. Por tanto
	\[
		M_{B_2}(\phi) = \begin{pmatrix}
		\lambda_1 & 0 & \cdots & 0 \\
		0 & \lambda_2 & \ddots & \vdots \\
		\vdots & \ddots & \ddots & 0 \\
		0 & \cdots & 0 & \lambda_n
		\end{pmatrix} \implies \determinant{M_{B_2}(\phi)} =
		\prod_{i=1}^{n} \lambda_i > 0
	\]
	Entonces, como $M_B(\phi) = S_{B,B_2}^TM_{B_2}(\phi)S_{B_2,B}$
	\[
		\determinant{M_B(\phi)} = \determinant{S_{B_2,B}}^2
		\determinant{M_{B_2}(\phi)} > 0
	\]
	Por lo tanto, la matriz de un producto escalar tiene
	determinante positivo independientemente de la base tomada.
	Observamos que $\phi$ también define un producto escalar en
	el subespacio vectorial $<v_1,\cdots,v_k>$ cuando lo
	restringimos a este. Por lo que hemos visto antes se tiene que
	\[
	\determinant{M_B(\phi)_k} = \delta_k(M_B(\phi)) > 0
	\quad \forall 1 \leq k \leq n.
	\]
	
	\noindent $\impliedby$
	
	\indent Tenemos que $\delta_k(M_B(\phi)) > 0 \quad \forall
	1 \leq k \leq n$. Aplicamos la siguiente variación de
	Gramm-Schmidt. Tomamos la base $B = \{ u_1, \cdots, u_n \}$
	Y hacemos la siguiente construcción:
	\[
		\begin{cases}
			v_1 = u_1 \\
			v_2 = \alpha_{2,1}u_1 + u2 \\
			v_3 = \alpha_{3,1}u_1 + \alpha_{3,2}u_2 + u_3 \\
			\vdots \\
			v_n = \alpha_{n,1}u_1 + \cdots + \alpha_{n,n-1}u_{n-1}
			+ u_n
		\end{cases}
		\qquad \alpha_{i,j} \text{ son tales que } \phi(v_k, u_i)
		= 0 \quad \substack{2 \leq k \leq n \\ 1 \leq i \leq k-1}
	\]
	Propiedades de $\{ v_1, \cdots, v_n \}$
	\begin{itemize}
		\item $\forall k$, $\left< v_1, \cdots, v_k \right> =
		\left<u_1,\cdots,u_k\right>$
		En particular, $B_2 = \{v_1,\cdots, v_n\}$ es base de $\E$.
		\item $\phi(v_k, v_i) = 0$ $\forall 1 \leq i \leq k-1$
		porque $v_i \in \left< u_1, \cdots, u_i \right>$ y hemos tomado los 
		$\alpha$ de manera que $\phi(v_k, u_i) = 0 \implies B_2$ 
		es base ortogonal
		\item La matriz $S_{B_2B}$
		\[
			S_{B_2B} = \begin{pmatrix}
				1 & \alpha_{2,1} & \cdots & \alpha_{n,1} \\
				0 & 1 & \ddots & \vdots \\
				\vdots & \ddots & \ddots & \alpha_{n,n-1} \\
				0 & \cdots & 0 & 1
			\end{pmatrix} \implies \determinant{S_{B_2B}} = 1
			\text{ y } \delta_k(S_{B_2B}) = 1
		\]
	\end{itemize}
	Finalmente, tenemos
	\[
		\begin{aligned}
			M_B(\phi) &= S_{B,B_2}^T M_{B_2}(\phi)S_{B,B_2} \\
			\left(\begin{array}{c c|c}
			k & \updownarrow &  \\
			\leftrightarrow &  &  \\
			\hline
			 &  & 
			\end{array}\right) &= \left(\begin{array}{c c|c}
			k & \updownarrow &  \\
			\leftrightarrow &  &  \\
			\hline
			&  & 
			\end{array}\right) \left(\begin{array}{c c c| c c}
			\phi(v_1,v_1) & & & & \\
			& \ddots & & & \\
			& & \phi(v_k,v_k) & & \\
			\hline
			& & & \ddots & \\
			& & & & \phi(v_n,v_n)
			\end{array}\right) \left(\begin{array}{c c|c}
			k & \updownarrow &  \\
			\leftrightarrow &  &  \\
			\hline
			&  & 
			\end{array}\right)
		\end{aligned}
	\]
	\[
		\implies \delta_k(M_B(\phi)) = \delta_k(S_{B,B_2}^t)
		\delta_k(M_{B_2}(\phi)) \delta_k(S_{B,B_2}) =
		\delta_k(M_{B_2}(\phi)) = 
	\]
	\[
		= \prod_{i=1}^{k} \phi(v_i,v_i) > 0\text{ (por hipótesis)}
		\implies
		\frac{\delta_k(M_B(\phi))}{\delta_{k-1}(M_B(\phi))}
		= \phi(v_k,v_k) > 0
	\]
	Finalmente, $\forall x \in \E$ 
	\[
		\phi(x,x) =
		\phi\left(
			\sum_{i=1}^{k} x_iv_i,\sum_{i=1}^{k} x_iv_i
		\right) =
		\sum_{i=1}^{k} x_i^2 \phi(v_i,v_i) > 0 \text{ si } x \neq 
		\vec{0}
	\]
	\endproof{QED}
\end{proof}

\begin{thm}[Método convergencia-pivote]
    Dada una forma bilineal simétrica $\phi$, queremos encontrar una base de
    $\E$, $B_2$, en la cual $M_{B_2}(\phi)$ sea una matriz diagonal. Partimos
    de una base $B$ i de $M_B(\phi)$. El procesos es: operación con filas a
    las dos matrices y luego la misma operacion pero en las columnas de la
    primera matriz únicamente (véase ejemplo).
    
    \begin{gather*}
        \left(M_B(\phi) \vert Id \right) \stackrel{\text{op. filas}}{\sim}
	\left( S_1M_B(\phi) \vert S_1 \right) \substack{\text{misma op.} \\
	\sim \\ \text{en columnas}} \left( S_1 M_B(\phi)S_1^T \vert S_1 \right)
	\sim \dots \sim \\ \sim \left( S_r \dots S_1 M_B(\phi)S_1^T \dots S_r^T
	\vert S_r \dots S_1 \right)
    \end{gather*}
    Donde la matriz de la izquierda es $M_{B_2}$ y es diagonal.
    
    
\end{thm}

\begin{example}
    \[
        q_\phi(x,y,z)=2x^2+2y^2-4xy-2yz; \quad A = M_B(\phi) =
        \begin{pmatrix}
            2 & -2 & 0 \\
            -2 & 2 & -1 \\
            0 & -1 & 0 \\
        \end{pmatrix}
    \]
    \begin{gather*}
        \left(
        \begin{array}{ccc|ccc}
            2 & -2 & 0 & 1 & 0 & 0 \\
            -2 & 2 & -1 & 0 & 1 & 0 \\
            0 & -1 & 0 & 0 & 0 & 1 \\
        \end{array}
        \right)
        \substack{\text{fila} \\ \sim \\ (1) + (2)}
        \left(
        \begin{array}{ccc|ccc}
            2 & -2 & 0 & 1 & 0 & 0 \\
            0 & 0 & -1 & 1 & 1 & 0 \\
            0 & -1 & 0 & 0 & 0 & 1 \\
        \end{array}
        \right)
        \substack{\text{columna} \\ \sim \\ (1) + (2)}
        \\
        \substack{\text{columna} \\ \sim \\ (1) + (2)}
        \left(
        \begin{array}{ccc|ccc}
            2 & 0 & 0 & 1 & 0 & 0 \\
            0 & 0 & -1 & 1 & 1 & 0 \\
            0 & -1 & 0 & 0 & 0 & 1 \\
        \end{array}
        \right)
        \substack{\text{fila} \\ \sim \\ (2) + (3)}
        \left(
        \begin{array}{ccc|ccc}
            2 & 0 & 0 & 1 & 0 & 0 \\
            0 & -1 & -1 & 1 & 1 & 1 \\
            0 & -1 & 0 & 0 & 0 & 1 \\
        \end{array}
        \right)
        \substack{\text{columna} \\ \sim \\ (2) + (3)}
        \\
        \substack{\text{columna} \\ \sim \\ (2) + (3)}
        \left(
        \begin{array}{ccc|ccc}
            2 & 0 & 0 & 1 & 0 & 0 \\
            0 & -2 & -1 & 1 & 1 & 1 \\
            0 & -1 & 0 & 0 & 0 & 1 \\
        \end{array}
        \right)
        \substack{\text{fila} \\ \sim \\ (3) - \frac{1}{2}(2)}
        \left(
        \begin{array}{ccc|ccc}
            2 & 0 & 0 & 1 & 0 & 0 \\
            0 & -2 & -1 & 1 & 1 & 1 \\
            0 & 0 & \frac{1}{2} & \frac{-1}{2} & \frac{-1}{2} & \frac{1}{2} \\
        \end{array}
        \right)
        \substack{\text{columna} \\ \sim \\ (3) - \frac{1}{2}(2)}
        \\
        \substack{\text{columna} \\ \sim \\ (3) - \frac{1}{2}(2)}
        \left(
        \begin{array}{ccc|ccc}
            2 & 0 & 0 & 1 & 0 & 0 \\
            0 & -2 & 0 & 1 & 1 & 1 \\
            0 & 0 & \frac{1}{2} & \frac{-1}{2} & \frac{-1}{2} & \frac{1}{2} \\
        \end{array}
        \right)
    \end{gather*}
    Entonces, en base $B$, los vectores de $B_2$ son:
    \begin{itemize}
        \item $v_1 = (1,0,0); \quad \phi (v_1, v_1) = 2$
        \item $v_2 = (1,1,1); \quad \phi (v_2, v_2) = -2$
        \item $v_3 = (\frac{-1}{2},\frac{-1}{2},\frac{1}{2}); \quad \phi (v_3,v_3) = \frac{1}{2}$
    \end{itemize}
    Y $\phi(v_i, v_j) = 0,$ $i \neq j$.
\end{example}


\begin{prop} 
    Sea $\E$ un $\k$-e.v. de dimension $n$, sea $\phi \colon \E \times \E \to \k$
    una forma bilineal simétrica y sea $q$ su forma cuadrática asociada.
    Consideremos $B=\left\{ u_1, \dots u_n\right\}$ una base \ita{q-ortogonal} de $\E$. 
    Sabemos que
    \[
        D=M_B \left( \phi \right)= \left(
        \begin{array}{ccc}
            \alpha _1 & \cdots & 0 \\
            \vdots & \ddots & \vdots \\
            0 & \cdots & \alpha _n 
        \end{array} \right).
    \]
    Consideremos el subespacio vectorial $\E ^{\perp} \subseteq \E$ definido por
    $\E ^{\perp}= \left\{u \in \E \hspace{0,1cm} | \hspace{0,1cm} \phi \left( u, v \right)
    = 0 \hspace{0,15cm} \forall v \in \E \right\}$. Tenemos que
    \begin{enumerate}[i)]
        \item
            \[
                D=M_B \left( \phi \right)= \left(
                \begin{array}{cccccc}
                    \alpha _1 & 0 &  & \cdots &  & 0 \\
                    0 & \ddots &  &  &  &  \\
                     &  & \alpha _m &  &  & \\
                    \vdots &  &  & 0 &  & \vdots  \\
                     &  & \ &  & \ddots &  \\
                    0 &  &  & \cdots &  & 0 
                \end{array} \right) \implies\ E ^{\perp} =
                \left< u_{m+1} , \dots , u_n \right> .
            \]
        \item
            \[
                \rg q + \dim \E^{\perp} = n \implies i_0 \left( q \right)
                 = \dim \E^{\perp}.
            \]
        \item Sean $\k=\real$,  e $i_+ \left( q \right)$ el número de elementos estrictamente positivos de
        la diagonal de $M_B \left( \phi \right)$. $i_+ \left( q \right)$ no depende de la base
        \emph{q-ortogonal} $B$ elegida.
        \item Sea $\k=\real$ y sean $\delta _0 = 1$, $\delta _1, \dots , \delta _n \neq 0$.
        Entonces, $i_- \left( q \right)$ es igual al número de cambios de signo en la
        secuencia $\delta _0, \dots , \delta _n$.
    \end{enumerate}
\end{prop}
\begin{proof}
    \begin{enumerate}[i)]
        \item[]
        \item \addtocounter{enumi}{1}
            $u_i  \in \left\{u_{m+1}, \dots , u_n \right\}$,
            \[
                \begin{aligned}
                    &\phi \left( u_j, u_i \right) = \left( 0 \ \cdots \ 1 \cdots \ 0 \ \right) \left( 
                    \begin{array}{cccccc}
                        \alpha _1 &  &  &  &  &  \\
                         & \ddots &  &  &  &  \\
                         &  & \alpha _m &  &  & \\
                         &  &  & 0 &  &   \\
                         &  & \ &  & \ddots &  \\
                         &  &  &  &  & 0 
                    \end{array} \right) \left(
                    \begin{array}{c}
                         0  \\
                         \vdots  \\
                         1  \\
                         \vdots \\
                         0 
                    \end{array} \right) = 0 \hspace{0,3cm} \forall \hspace{0,12cm} 1 \leq j \leq n
                    \implies \\
                    &\implies \left< u_{m+1} , \dots , u_n \right> \subseteq \E^{\perp}.
                \end{aligned}
            \]
            Sea $u \in \E$ tal que $u \notin \left< u_{m+1}, \dots u_n \right> $. Se tiene que $\exists \hspace{0,2cm} 1 \leq i \leq m$ t.q. $x_i \neq 0$. Entonces,
            \[
                \begin{aligned}
                    e_i^t M_B\left( \phi \right) u = \left(0 \ \cdots \ 1 _i \ \cdots \ 0 \right) \left( 
                    \begin{array}{cccccc}
                        \alpha _1 &  &  &  &  &  \\
                         & \ddots &  &  &  &  \\
                         &  & \alpha _m &  &  & \\
                         &  &  & 0 &  &   \\
                         &  & \ &  & \ddots &  \\
                         &  &  &  &  & 0 
                    \end{array} \right) \left(
                    \begin{array}{c}
                         \vdots  \\
                         x_i  \\
                         \vdots \\
                    \end{array} \right) = \alpha _i x_i \neq 0 \implies u \notin \E^{\perp}.
                \end{aligned}
            \]
            Así pues, $\E ^{\perp} = \left< u_{m+1} , \dots , u_n \right> .$
        \item
            Sea $B'=\left\{u'_1, \dots , u'_n \right\}$ una base \ita{q-ortogonal} de $\E$.
            \[
                M_{B'}\left( \phi \right) = D' =  \left(
                \begin{array}{ccc}
                    \alpha ' _1 & \cdots & 0 \\
                    \vdots & \ddots & \vdots \\
                    0 & \cdots & \alpha ' _n 
                \end{array} \right).
            \]
            Tenemos que 
            \[
                m=i_+\left( q, B \right); \hspace{0,2cm} \F^+=\left<u_1, \dots u_m \right>;
                \hspace{0,2cm} \E=\F^+ \oplus \left<u_{m+1}, \dots u_n \right>=\F^+ \oplus \F^{ {\leq 0}}.
            \]
            Análogamente,
            \[
                m^\prime=i_+\left( q, B^\prime \right); \hspace{0,2cm} \F^{\prime +}=
                \left<u^\prime_1, \dots u^\prime_m \right>; \hspace{0,2cm} \E=\F^{\prime +}
                \oplus \left<u^{\prime}_{m+1}, \dots u^\prime_n \right>=\F^{\prime+}
                \oplus \F ^{\prime\leq 0}.
            \]
            Consideremos la función que proyecta un vector de $\F^+$ sobre $\F^{\prime+}$.
            \[
                \begin{aligned}
                    f \colon \F^+ &\to \E \to \F^{\prime+} \\
                    v &\longmapsto f(v).
                \end{aligned}
            \]
            Sea $v\in \F^+$ y sean $v_1$ y $v_2$ las proyecciones de $v$
            sobre $\F^{\prime+}$ y $\F^{\prime \leq 0}$ respectivamente.
            Tenemos que $f(v)=v_1$. Entonces, 
            \[
            	f(v)=0 \implies v_1=0 \implies v=v_2.
            \]
            Además, $0\leq \phi \left( v, v \right)$ y $\phi \left( v_2 , v_2 \right)=
             \phi \left( v, v \right) \leq 0$, de modo que $v=0$ y $f$ es inyectiva.
            Considerando la función que proyecta un vector de $\F^{\prime+}$ sobre $\F^{+}$.
            \[
                \begin{aligned}
                    g \colon \F^{\prime +} &\to \E \to \F^{+} \\
                    v &\longmapsto g(v).
                \end{aligned}
            \]
            y siguiendo un razonamiento análogo, obtenemos que $g$ es inyectiva, 
            de modo que $m=m'$.
    \end{enumerate}
        
    
    
\end{proof}
\subsection{Clasificación afín y proyectiva}
\begin{defi}
	Sean $\E$ y $\F$ $\k$-espacios vectoriales. Sean
	\[
		\phi \colon \E \times \E \to \k,
	\]
	\[
		\psi \colon \F \times \F \to \k
	\]
formas bilineales simétricas. Diremos que $\phi$ y
$\psi$ son afínmente equivalentes, y escribiremos 
$\phi \sim \psi$, si existe un isomorfismo $f \colon \E \to \F$ tal que
\[
	\phi \left(u, v \right) = \psi \left( f \left( u \right) , 
	f \left( v \right)  \right) \hspace{0,2cm} \forall u, v \in \E,
\]
\[
	\phi \left( f ^{-1} \left( u ' \right) ,  f ^{-1} \left( v ' \right) 
	\right) = \psi \left( u' , v' \right) \hspace{0,2cm} \forall
	u', v' \in \F.
\]
\end{defi}
\begin{thm}[de Sylvester]
    \begin{enumerate}[i)]
        \item[]
        \item $\k=\real$
        \[
            \phi \sim \psi \iff \rg \phi = \rg \psi \text{ y }
            i_+\left( \phi \right) = i_+\left( \psi \right).
        \]
        \item $\k=\cx$
        \[
            \phi \sim \psi \iff \rg \phi = \rg \psi.
        \]
    \end{enumerate}

\end{thm}
\section{Series Numéricas e Integrales Impropias}

\subsection{Series numéricas}

\begin{defi}
	Una serie de números reales es una pareja de sucesiones de números reales
	$(a_n)_{n \geq 0}$, $(s_n)_{n \geq 0}$, relacionadas por
	\[
		s_n = \sum_{i=0}^{n} a_n
	\]
	Denominaremmos término $n$-ésimo de la serie al elemento $a_n$ y llamaremos
	suma parcial $n$-ésima de la serie a $s_n$
\end{defi}
\begin{obs*}
	Las sumas parciales definen los términos
	\[
		a_0 = s_0 \qquad a_n = s_n - s_{n-1} \quad (n \geq 1)
	\]
\end{obs*}

\begin{defi}
	Llamaremos suma de una serie a
	\[
		s = \lim s_n = \lim\limits_{n \to \infty} \sum_{k=0}^{n} a_n
	\]
	suponiendo que existe
\end{defi}
\begin{obs*}
	Denotaremos $s = \sum\limits_{n \geq 0} a_n = \sum\limits_{n \geq 0}^{\infty} a_n$.
	Esta misma notación nor servirá para representar la serie.
\end{obs*}

\begin{defi}
	Diremos que una serie $\sum a_n$ es convergente o divergente si lo es la
	sucesión de sumas parciales
	\begin{itemize}
		\item convergente \qquad $\lim s_n \in \real$
		\item divergente  \qquad $\lim s_n = \pm \infty$
		\item oscilante \qquad $\nexists \lim s_n$
	\end{itemize}
\end{defi}

\begin{obs}
	Una serie no tiene por qué comenzar por el índice 0, y por tanto, podemos considerar
	series con términos $a_n$ donde $n \geq n_0$. En tal caso, las sumas parciales son
	$s_n = \sum\limits_{k=n_0}^n a_n$, y la suma (si existe) $\sum\limits_{k=n_0}^{\infty}
	a_n = \lim\limits_{n \to \infty} \sum\limits_{k=n_0}^n a_n$.
\end{obs}


\begin{defi}
	Sea $r \in \real$. Llamaremos serie geométrica de razón $r$ a la serie
	\[
		\sum_{n \geq 0} r^n
	\]
\end{defi}

\begin{prop*}
	La serie geométrica es convergente si y solo si $\abs{r} < 1$, en tal caso
	la suma es
	\[
		\sum_{n \geq 0} r^n = \frac{1}{1-r}
	\]
\end{prop*}

\begin{proof}
	Primero, calculamos el término $n$-ésimo
	\[
		s_n = 1 + r + \cdots + r^n = \begin{cases}
			n+1 \qquad \text{si } r = 1 \\
			\frac{r^{n+1} - 1 }{r-1} \qquad \text{si } r \neq 1
		\end{cases}
	\]
	\begin{itemize}
		\item Si $r = 1$, $\lim s_n = \lim\limits_{n  \to \infty} n+1 = \infty$
		\item Si $\abs{r} > 1$, $\lim s_n = \lim\limits_{n \to \infty}
			\frac{r^{n+1} - 1}{r-1} = \infty$
		\item Si $\abs{r} < 1$, $\lim s_n = \lim\limits_{n \to \infty}
			\frac{r^{n+1} - 1}{r-1} = \frac{-1}{r-1}$
		\item Si $r = -1$, $s_n = 0$ si $n$ par y $s_n = 1$ si $n$ impar. Por
			lo tanto la serie es oscilante
	\end{itemize}
\end{proof}

\begin{prop}
	Si $\sum a_n$ es convergente, entonces $\lim a_n = 0$
\end{prop}

\begin{proof}
	Sabemos que $a_n = s_n - s_{n-1}$, por lo tanto
	$\lim a_n = \lim (s_n - s_{n-1})$, como $\lim s_n$ existe (y por lo tanto
	también $\lim s_{n-1}$)
	\[
		\lim a_n = \lim (s_n - s_{n-1}) = \lim s_n - \lim s_{n-1} = 0
	\]
\end{proof}

\begin{prop}[Criterio de Cauchy para series]
	La serie $\sum a_n$ es convergente si $\forall\varepsilon>0$,
	$\exists n_0 \in \n$
	tal que
	\[
		m,n \geq n_0 \implies \abs{s_m - s_n} = \abs{a_m + a_{m-1} \cdots + a_n} <
		\varepsilon
	\]
\end{prop}

\begin{prop}[linealidad]
	Sean $\alpha, \beta \in \real$ y sean $\sum a_n$ y $\sum b_n$ series convergentes.
	Entonces $\sum (\alpha a_n + \beta b_n)$ también lo es y $\sum (\alpha a_n + \beta b_n)
	= \alpha \sum a_n + \beta \sum b_n$.
\end{prop}

\begin{prop}
	Sean dos sucesiones $(a_n)$y $(b_n)$, son iguales salvo en número finito de términos,
	entonces las series $\sum a_n$ y $\sum b_n$ tienen la misma convergencia.
\end{prop}

\begin{proof}
	Sea $d_n = b_n - a_n$, que vale 0 salvo en número finito de términos
	\begin{itemize}
		\item Si $\sum a_n$ converge \qquad $\sum b_n = \sum a_n +
		\overbrace{\sum d_n}^\text{Suma finita}$ $\implies$ $\sum b_n$ converge
		\item Si $\sum a_n$ diverge \qquad $\sum b_n = \sum a_n + \sum d_n
		\implies \sum b_n$ diverge
		\item Si $\sum a_n$ oscila \qquad $\sum b_n = \sum a_n + \sum d_n
		\implies \sum b_n$ oscila
	\end{itemize}
\end{proof}

\begin{prop}[Asociatividad]
	Sea $\sum a_n$ una serie y $(n_k)_{k \geq 0}$ una sucesión estrictamente creciente de
	números naturales. Definimos
	\[
		b_0 = a_0 + \cdots + a_{n_0} \qquad b_k = a_{(n_{k-1}+1)} + \cdots + a_{n_k}
	\]
	Si existe la suma de $\sum a_n$, entonces también existe la suma de $\sum b_k$
	y son iguales.
\end{prop}

\begin{proof}
	Sea $A_n = \sum\limits_{i=0}^{n} a_i$ y $B_k = \sum\limits_{i=0}^{k} b_i$, por la
	definición anterior se tiene que $B_k = A_{n_k}$ y por lo tanto $(B_k)$ es una sucesión
	parcial de $(A_n)$, lo cual implica que si $(A_n)$ converge, $(B_k)$ también
	y lo hace al mismo número.
\end{proof}

%%%%%%%%%%%%%%%%%%%%%%%%%%%%%%%
% SERIES DE NUMEROS POSITIVOS %
%%%%%%%%%%%%%%%%%%%%%%%%%%%%%%%

\subsection{Series de números positivos}

\begin{prop}
	Si una serie $\sum a_n$ es de \textit{términos positivos} ($a_n \geq 0$) entonces la
	sucesión $(s_n)$ de sumas parciales es \textit{creciente}, y por tanto, siempre tiene
	límite:
	\[
		\sum a_n = \lim s_n = \sup_{n \in \n} s_n
	\]
	Este puede ser finito (si la sucesión de sumas parciales es acotada) o infinito (en caso
	contrario).
\end{prop}

\begin{prop}[Criterio de comparación directa]
	Sean $\sum a_n$ y $\sum b_n$ series de términos positivos. Si $\exists n_0$ tal que
	$a_n \leq b_n$ ($\forall n \geq n_0$). Entonces
	\[
		\sum_{n=n_0}^{\infty} a_n \leq \sum_{n=n_0}^{\infty} b_n
	\]
	Por tanto, la convergencia de $\sum b_n$ implica la de $\sum a_n$ y la divergencia de
	$\sum a_n$ implica la de $\sum b_n$.
\end{prop}

\begin{proof}
	Por el enunciado
	\[
		\sum_{i=n_0}^{n} a_i \leq \sum_{k=n_0}^{n} b_k \implies
		\sum_{i=n_0}^{\infty} a_i \leq \sum_{k=n_0}^{\infty} b_k
	\]
	Los términos $a_0, \cdots, a_{n_0}$ se pueden añadir al sumatorio y no alteran
	la convergencia.
\end{proof}

\begin{defi*}
	Llamamos serie harmónica a la serie
	\[
		\sum_{n \geq 1} \frac{1}{n}
	\]
\end{defi*}

\begin{defi}[Serie de Riemman]
	Sea $p \in \real$. Llamaremos serie harmónica generalizada o serie de Riemman
	de parámetro $p$ a la serie
	\[
		\sum_{n \geq 1} \frac{1}{n^p}
	\]
\end{defi}

\begin{prop*}
	La serie de Riemman es convergente si y solo si $p > 1$.
\end{prop*}

\begin{proof}
	Distinguiremos entre varios casos
	\begin{itemize}
		\item Si $p = 1$. Suponemos que la serie es convergente con suma $s$
		\[
			s = \left( 1 + \frac{1}{2} \right) + \left( \frac{1}{3} + \frac{1}{4}
			\right) + \cdots > \left( \frac{1}{2} + \frac{1}{2} \right) + \left(
			\frac{1}{4} + \frac{1}{4} \right) + \cdots =
			1 + \frac{1}{2} + \frac{1}{3} + \frac{1}{4} + \cdots = s
		\]
		absurdo ya que $s>s$.
		\item Si $p < 1$.
		\[
			n^p \leq n \implies \frac{1}{n^p} \geq \frac{1}{n}
		\]
		y por comparación directa con la serie harmónica, diverge.
		\item Si $p > 1$.
		\[\begin{split}
			\sum_{n \geq 1} \frac{1}{n^p} = 1 + \left(\frac{1}{2^p} + \frac{1}{3^p}
			\right) + \left( \frac{1}{4^p} + \frac{1}{5^p} + \frac{1}{6^p} +
			\frac{1}{7^p} \right) + \cdots \leq \\ \leq 1 + \left(\frac{1}{2^p} +
			\frac{1}{2^p} \right) + \left( \frac{1}{4^p} + \frac{1}{4^p} +
			\frac{1}{4^p} + \frac{1}{4^p} \right) = 1 + \frac{1}{2^{p-1}} +
			\frac{1}{2^{2(p-1)}} + \cdots
		\end{split}\]
		que es una serie geométrica de razón $\frac{1}{2^{p-1}} < 1$ y por lo tanto
		convergente.
	\end{itemize}
\end{proof}

\begin{prop}[Criterio de comparación en el límite]
	Sean $\sum a_n$ y $\sum b_n$ series de términos estrictamente positivos.
	Suponemos que existe el límite
	\[
		\lim \frac{\sum a_n}{\sum b_n} = l \in [0,+\infty]
	\]
	\begin{itemize}
		\item Si $l < + \infty$. $\sum b_n$ converge $\implies$ $\sum a_n$ converge y
			$\sum a_n$ diverge $\implies \sum b_n$ diverge.
		\item Si $l > 0$. $\sum a_n$ converge $\implies$ $\sum b_n$ converge y si $\sum b_n$
			diverge $\implies \sum a_n$ diverge.
		\item Si $0 < l < +\infty$. Entonces las dos series tienen el mismo caracter.
	\end{itemize}
\end{prop}
\begin{proof}
	Provaremos cada caso de manera individual
	\begin{itemize}
		\item Caso $l < +\infty$. Fijado $\varepsilon > 0$, por definición de límite, existe
		$n_0\in\n$ tal que
		\[
			n \geq n_0 \implies \frac{a_n}{b_n} < l + \varepsilon \implies
			a_n < (l+\varepsilon)b_n
		\]
		y el resultado queda provado por comparación directa.
		\item Caso $l > 0$. Se deduce del primer caso, considerando
		\[
			\lim \frac{b_n}{a_n} = \frac{1}{l}
		\]
		\item Caso $0 < l < +\infty$. Se trata de una conjunción de los casos
		anteriores
	\end{itemize}
\end{proof}

\begin{lema}
	Sea $\sum a_n$ una serie de términos positivos.
	\begin{itemize}
		\item Suponemos que hay $n_0 \in \n$ y $r < 1$ tal que
		\[
			n \geq n_0 \implies a_n^{1/n} < r
		\]
		entonces $\sum a_n < +\infty$
		\item Suponemos que existe $n_0 \in \n$ tal que
		\[
			n \geq n_0 \implies a_n^{1/n} \geq 1
		\]
		entonces $\sum a_n = +\infty$
	\end{itemize}
\end{lema}

\begin{proof}
	Provaremos cada caso por separado
	\begin{itemize}
		\item $a_n^{1/n} < r \implies a_n < r^n$ que es la serie geométrica de
		razón $r < 1$, de modo que por comparación directa el resultado queda
		demostrado.
		\item $a_n^{1/n} \geq 1 \implies a_n \geq 1$ y por lo tanto diverge.
	\end{itemize}
\end{proof}

\begin{prop}[Criterio de la raíz de Cauchy]
	Sea $\sum a_n$ una serie de términos positivos. Suponemos que existe
	$\lim a_n^{1/n} = \alpha$, entonces, si $\alpha > 1$ la serie diverge y si
	$\alpha < 1$ la serie converge.
\end{prop}

\begin{proof}
	Demostraremos cada caso por separado
	\begin{itemize}
		\item Caso $\alpha < 1$. Sea $\alpha < r < 1$. Existe $n_0 \in \n$ tal que
		\[
			n \geq n_0 \implies a_n^{1/n} \leq r
		\]
		Y el resultado queda provado aplicando el lema anterior.
		\item Caso $\alpha > 1$. Existe $n_0 \in \n$ tal que
		\[
			n \geq n_0 \implies a_n^{1/n} \geq 1
		\]
		y aplicamos el lema anterior.
	\end{itemize}
\end{proof}

\begin{lema}
	Sea $\sum a_n$ una serie de términos estrictamente positivos.
	\begin{itemize}
		\item Suponemos que hay $n_0 \in \n$ y $r < 1$ tal que
		\[
			n \geq n_0 \implies \frac{a_{n+1}}{a_n} \leq r
		\]
		entonces $\sum a_n < + \infty$
		\item Suponemos que existe $n_0 \in \n$ tal que
		\[
			n \geq n_0 \implies \frac{a_{n+1}}{a_n} \geq 1
		\]
		entonces $\sum a_n = + \infty$
	\end{itemize}
\end{lema}

\begin{proof}
	Separaremos los casos.
	\begin{itemize}
		\item
		\[
			\frac{a_n+1}{a_n} \leq r \implies a_{n+1} \leq r a_n \implies a_n \leq
			C r^n \quad (n \geq n_0)
		\]
		donde $C = \frac{a_{n_0}}{r^{n_0}}$ y por el criterio de comparación
		directa $\sum a_n$ converge.
		\item $\frac{a_{n+1}}{a_n} \geq 1\implies a_{n+1} \geq a_n \implies a_n$ es creciente
			$\implies \sum a_n$ diverge
	\end{itemize}
\end{proof}

\begin{prop}[Criterio del cociente de Alembert]
	Sea $\sum a_n$ una serie de términos estrictamente positivos. Suponemos que existe
	$\lim \frac{a_{n+1}}{a_n} = \alpha$, entonces
	\begin{itemize}
		\item Si $\alpha > 1$ la serie diverge
		\item Si $\alpha < 1$ la serie converge
	\end{itemize}
\end{prop}

\begin{proof}
	Separamos los dos casos
	\begin{itemize}
		\item Si $\alpha < 1$. Sea $\alpha < r < 1$ entonces $\exists n_0 \in \n$
		tal que
		\[
			n \geq n_0 \implies \frac{a_{n+1}}{a_n} \leq r
		\]
		y aplicamos el lema anterior.
		\item Si $\alpha > 1$. Entonces  $\exists n_0 \in \n$ tal que
		\[
			n \geq n_0 \implies \frac{a_{n+1}}{a_n} \geq 1
		\]
		y aplicamos el lema anterior.
	\end{itemize}
\end{proof}

\begin{example*}
	Estudiar la convergencia de
	\begin{itemize}
		\item $\sum\limits_{n \geq 0} \frac{1}{n!}$. \\
			$n!$ crece más que	$n^2$ ($n! > n^2$) $\implies \frac{1}{n!} < \frac{1}{n^2}$
			que es la serie de Riemman de parámetro 2 (convergente). Por tanto,
			$\sum\limits_{n \geq 0}^\infty \frac{1}{n!}$ es convergente.
		\item $\sum \frac{x^n}{n!}$ para $x > 0$. \\
			\[
				\lim_{n \to \infty} \frac{\frac{x^{n+1}}{(n+1)!}}{\frac{x^n}{n!}} =
				\lim_{n \to \infty} \frac{x^{n+1}n!}{x^n(n+1)n!} = \lim_{n \to \infty}
				\frac{x}{n+1} = 0 < 1
			\]
			Por lo tanto, aplicando el criterio del cociente de Alembert, la serie coverge.
		\item $\sum \alpha^{n+\sqrt{n}}$ \\
			\[
				\lim_{n \to \infty} \alpha^{\frac{n + \sqrt{n}}{n}} =
				\lim_{n \to \infty} \alpha^{1 + \frac{1}{\sqrt{n}}} = \alpha
			\]
			Por lo tanto, por el criterio de la raíz, $\begin{cases} \alpha < 1
			\text{ convergente} \\ \alpha > 1 \text{ divergente}\end{cases}$. Si $\alpha =1$,
			la serie es $\sum 1^{n + \sqrt{n}} = \sum 1$ que es divergente.
	\end{itemize}
\end{example*}

\begin{obs}
	Los criterios anteriores no deciden cuando $\alpha = 1$. Como $\sfrac{a_{n+1}}{a_n} \to
	\alpha$ implica que $a_n^{\sfrac{1}{n}} \to \alpha$, si el criterio del cociente no
	decide, entonces el de la raíz tampoco.
\end{obs}

\begin{prop}[Criterio de Raabe]
	Sea $\sum a_n$ una serie de términos estrictamente positivos tal que existe el límite
	\[
		L = \lim_{n \to \infty} n\left( 1 - \frac{a_{n+1}}{a_n} \right)
	\]
	Si $L > 1$, la serie $\sum a_n$ es convergente. Si $L < 1$, la serie $\sum a_n$ es
	divergente.
\end{prop}

\begin{prop*}[Criterio de condensación]
	Sea $(a_n)$ una sucesión de números positivos decreciente. Entonces
	\[
		\sum_{n=0}^{\infty} a_n \text{ converge} \iff \sum_{n=0}^{\infty} 2^n a_{2^n}
		\text{ converge}
	\]
\end{prop*}

\begin{prop*}[Criterio logarítmico]
	Sea $\sum a_n$ una serie de términos positivos tal que existe el límite
	\[
		L = \lim_{n \to \infty} \frac{- \ln(a_n)}{\ln(n)} = \lim_{n \to \infty}
		\frac{\ln\left( \frac{1}{a_n} \right)}{\ln(n)}
	\]
	Si $L > 1$, la serie $\sum a_n$ es convergente. Si $L < 1$, la serie $\sum a_n$ es
	divergente.
\end{prop*}

\begin{prop}[Criterio de la integral]
	Sea $n_0 \in \n$ y $f \colon [n_0, +\infty) \to \real$ positiva, localmente integrable
	y decreciente. Consideramos $a_n = f(n)$ ($n \geq n_0$) entonces
	\begin{enumerate}[i)]
		\item La serie $\sum a_n$ y la integral impropia
		$\int_{n_0}^{+\infty} f$ tienen el mismo carácter.
		\item Para $N \geq n_0$
		\[
			\sum_{n \geq n_0}^{\infty} a_n = \sum_{n = n_0}^{N-1} + \int_{N}^{+\infty}f +
			\varepsilon_n
		\]
		donde $\varepsilon_n \in [0, a_n]$
	\end{enumerate}
\end{prop}

\begin{example*}
	\begin{itemize}
		\item[]
		\item $\sum \frac{1}{n^\alpha}$ tiene el mismo carácter que
			$\int_{1}^{\infty} \frac{1}{x^\alpha} \dif x$ (convergente $\iff \alpha > 1$)
		\item Calcular $\sum\limits_{n\geq 1} \frac{1}{n^{1.01}}$ con error $< 10^{-3}$. \\
			Necesitamos que
			\[
				\frac{1}{N^{1.01}} < 10^{-3} \implies N > 1000^{\sfrac{1}{1.01}} \implies
				N \geq 934
			\]
			Calculamos ahora
			\[
				\sum_{n=1}^{933} \frac{1}{n^{1.01}} + \int_{934}^{+\infty}
				\frac{\dif x}{x^{1.01}} \simeq 100.577 \simeq\sum_{n\geq 1}\frac{1}{n^{1.01}}
			\]
	\end{itemize}
\end{example*}

\begin{prop}
	Sea $\sum a_n$ una serie de términos positivos. Dada cualquier permutación $\sigma
	\colon \n \to \n$, la serie $\sum a_{\sigma(n)}$ tiene la misma suma que $\sum a_n$.
\end{prop}

\begin{proof}
	Sea $A_n = \sum\limits_{k=0}^n a_k$ y $B_n = \sum\limits_{k=0}^n a_{\sigma(k)}$ y sean
	$A = \lim A_n$ y $B = \lim B_n$. Sea
	$m \in \n$, entonces $\exists n \in \n$ tal que
	\[
		\setb{0,1, \dots, m} \leq \setb{\sigma(0), \sigma(1), \dots, \sigma(n)}
	\]
	ya que $\sigma$ es suprayectiva. Entonces $a_0 + a_1 + \cdots + a_m \leq a_{\sigma(0)} +
	a_{\sigma(1)} + \cdots + a_{\sigma(n)}$ por lo tanto, $A_m \leq B_n \implies A \leq B$.
	Haciendo el mismo razonamiento para $\inv{\sigma}$ (biyectiva), obtenemos que $B \leq A$.
	Y por lo tanto, $A = \sum a_n = \sum a_{\sigma(n)} = B$.
\end{proof}

\subsection{Series absolutamente convergentes y condicionalmente convergentes}

\begin{defi}
  Diremos que una serie $\sum a_n$ es absolutamente convergente, si la serie $\sum \abs{a_n}$ es convergente.
\end{defi}

\begin{prop}
  Toda serie absolutamente convergente es convergente.
\end{prop}

\begin{proof}
  Aplicamos criterio de Cauchy para series a $\sum \abs{a_n}$: $\forall \varepsilon > 0$ $\exists n_0 \in \n$
  tal que
  \[
    m > n \geq n_0 \implies \abs{\abs{a_{n+1}} + \cdots + \abs{a_m}} < \varepsilon \implies
    \abs{a_{n+1}} + \cdots + \abs{a_m} < \varepsilon
  \]
  De donde se deduce que
  \[
    \abs{a_{n+1} + \cdots + a_n} < \abs{a_{n+1}} + \cdots + \abs{a_m} < \varepsilon
  \]
  (por la desigualdad triangular). Y por lo tanto, $\sum a_n$ cumple el criterio de Cauchy.

\end{proof}

\begin{defi}
  Una serie convergente, que no es absolutamente convergente, se dice que es condicionalmente convergente.
\end{defi}

\begin{example*}
  La serie armónica alternada $\sum \frac{(-1)^n}{n}$ es condicionalmente convergente.
\end{example*}

\begin{prop}
	Si $\sum a_n$ y $\sum b_n$ son absolutamente convergentes, y $\lambda \in \real$,
	entonces $\sum (a_n + b_n)$ y $\sum \lambda a_n$ son absolutamente convergentes.
\end{prop}

\begin{defi}
  Sea $a \in \real$. Definimos $a_+$ (la parte positiva de $a$) como $a_+ = \max(a, 0)$, asimismo, definimos
  la parte negativa de $a$ como $a_- = \max(-a,0)$.
\end{defi}

\begin{obs*}
  Dado un $a$, podemos expresar $a = a_+ - a_-$ y $\abs{a} = a_+ + a_-$
\end{obs*}

\begin{obs*}
  Dada $f \colon X \to \real$, podemos hacer exactamente lo mismo, ($f = f_+ - f_-$, $\abs{f} = f_+ + f_-$)
\end{obs*}

\begin{example*} \hspace{0pt}
  \begin{center}
  \begin{tikzpicture}
  \begin{axis}[
    xmin = 0,
    xmax = 2*3.14151625,
    ymin = -1.2,
    ymax = 1.2,
    samples = 100,
    axis lines = center,
    legend style={at={(0.1,0.1)},anchor=west}
  ]
    \addplot[domain=0:7,line width=3pt]{sin(deg(x))};
    \addplot[domain=0:3.14151625,color=red,thick]{sin(deg(x))};
    \addplot[domain=3.14151625:7,color=green,thick]{-sin(deg(x))};
    \addplot[domain=3.14151625:7,color=red,thick]{0};
    \addplot[domain=0:3.14151625,color=green,thick]{0};

    \legend{$sin(x)$,$sin_+(x)$,$sin_-(x)$}
  \end{axis}
  \end{tikzpicture}
  \end{center}

\end{example*}

\begin{lema}
  Sea $(a_n)$ una sucesión de números reales. Sean $(p_n)$ y $(q_n)$ sus partes positiva y negativa
  (respectivamente).
  \begin{enumerate}[i)]
    \item \label{item:abs_conv_conv}
			$\sum a_n$ converge absolutamente $\iff$ $\sum p_n$, $\sum q_n$ son convergentes.
    \item Si $\sum a_n$ es condicionalmente convergente, entonces $\sum p_n$ y $\sum q_n$ son divergentes
  \end{enumerate}
\end{lema}

\begin{proof}
  \begin{enumerate}[i)]
  	\item[]
  	\item Se tiene que
			\[
				\sum\limits_{k=0}^ n \abs{a_k} = \sum\limits_{k=0}^ n p_k + \sum\limits_{k=0}^n q_k
    	\]
    	Si $\sum \abs{a_n}$ converge $\implies \sum p_n$ y $\sum q_n$ tienen el mismo carácter,
			como ambas son series de términos positivos, $\sum a_n$ y $\sum b_n$ convergen.

			El reciproco es directo por linealidad.
		\item Se tiene que
			\[
				\sum_{k=0}^n a_k = \sum_{k=0}^n p_k - \sum_{k=0}^n q_k
			\]
			$\sum p_n$ y $\sum q_n$ no pueden ser las dos convergentes por \ref{item:abs_conv_conv}
			y tampoco puede ser que solo una de las dos sea divergente, porque entonces
			$\sum a_n$ divergería.
  \end{enumerate}
\end{proof}

\begin{prop}
	Si una serie es absolutamente convergente, entonces todas sus series reordenadas
	son convergentes con la misma suma.

	Es decir, $\forall \sigma \colon \n \to \n$ permutación, $\sum a_n = \sum a_{\sigma(n)}$
\end{prop}

\begin{proof}
	Primero, escribimos
	\[
		a_n = p_n - q_n \stackrel{\text{$a_n$ abs. conv.}}{\implies}
		\sum a_n = \sum p_n - \sum q_n
	\]
	Consideramos ahora $\sigma \colon \n \to \n$ permutación, entonces, $\sum a_{\sigma(n)}$
	también es absolutamente convergente ($\sum \abs{a_{\sigma{n}}} = \sum \abs{a_n}$
	reordenando términos positivos). Ahora tenemos que
	\[
		\sum a_{\sigma(n)} = \sum p_{\sigma(n)} - \sum q_{\sigma(n)}
		\substack{\text{términos} \\ \text{positivos} \\ =} \sum p_n - \sum q_n = \sum a_n
	\]
	\qed
\end{proof}

\begin{thm}[de Riemman para series condicionalmente convergentes]
	Sea $\sum a_n$ una serie condicionalmente convergente. $\forall s \in [-\infty,+\infty]$
	existe una reorenación de la serie $\sigma \colon \n \to \n$ (permutación) tal que
	$\sum a_{\sigma(n)} = s$.
\end{thm}

\begin{defi}
	Una serie alternada es una serie donde los términos cambian de signo alternativamente.
	Es decir, una serie de la forma $\sum (-1)^n a_n$ donde $a_n \geq 0$.
\end{defi}

\begin{prop}[Criterio de Leibnitz]
	Si $(a_n)$ es una sucesión descendiente de términos positivos con $\lim a_n = 0$,
	entonces  $\sum (-1)^n a_n$ es convergente.

	Además, si $S_N$ es la $N$-ésima suma parcial de $\sum (-1)^n a_n$ y
	$S$ es su suma, $\abs{S - S_n} \leq a_{n+1}$.
\end{prop}

\begin{proof}
	Consideramos la serie $(S_{2N})$
	\[
		S_{2N+2} = S_{2N} + \overbrace{(a_{2N+1} + a_{2N+2})}^{\leq 0} \leq S_{2N}
	\]
	Por lo tanto, $(S_{2N})$ es descendiente, acotada inferiormente por $a_0 - a_1$.
	Consideramos ahora $(S_{2N+1})$
	\[
		S_{2N+3} = S_{2N+1} + \overbrace{(a_{2N+2} + a_{2N+3})}^{\geq 0} \geq S_{2N+1}
	\]
	Con lo cual $(S_{2N+1})$ es creciente. Además se tiene que
	\begin{equation}\label{eq:leibn_cotas}
		a_0 - a_1 = S_1 \leq S_{2N+1} \leq S_{2N} \leq S_0 = a_0
	\end{equation}
	Con lo cual deducimos que tanto $(S_{2N})$ como $(S_{2N+1})$ son convergentes
	(monótonas y acotadas). Por último, tenemos que
	\[
		\lim (S_{2N} - S_{2N-1}) = \lim a_{2N} = 0 \implies \begin{rcases}
			\lim S_{2N} = S \\ \lim S_{2N+1} = S
		\end{rcases} \implies \lim S_N = S
	\]
	Para acabar, sabemos por (\ref{eq:leibn_cotas}) que S está dentro del intervalo
	de extremos $S_{N}$ y $S_{N+1}$ de longitud $a_{N+1}$, por lo tanto
	$\abs{S - S_N} \leq a_{N+1}$
\end{proof}

\begin{example*}
	La serie harmónica alternada, $\sum \frac{(-1)^n}{n}$ es convergente por el
	criterio de Laibnitz.
\end{example*}

Hay otros criterios de convergencia para series cualesquiera, entre los cuales
destaca el criterio de Dirichlet.

\begin{prop}[Critero de Dirichlet]
	Sean $(a_n)$ y $(b_n)$ dos sucesiones numéricas. Suponemos que
	\begin{enumerate}[i)]
		\item las sumas parciales $s_n$ de la serie $\sum a_n$ están acotadas.
		\item la sucesión $(b_n)$ es positiva y decreciente con límite 0.
	\end{enumerate}
	Entonces la serie $\sum a_nb_n$ es convergente.
\end{prop}

\subsection{Aplicación: Series de potencias}

\begin{defi}
	Una serie de potencias (centrada en 0) es una expresión
	\[
		\sum_{n \geq 0} a_nx^n
	\]
	donde $a_n$ son los coeficientes de la serie.
\end{defi}

\begin{lema}
	Sea $\sum a_nx^n$ una serie de potencias. El conjunto de los $r \geq 0$ tales
	que $\sum \abs{a_n}r^n$ converge es un intervalo que contiene al 0.
\end{lema}

\begin{proof}
	Si $0 \leq s \leq r$ y $\sum \abs{a_n}r^n$ converge, entonces $\sum \abs{a_n}s^n$
	converge también por comparación directa ($\abs{a_n}s^n \leq \abs{a_n}r^n$) y
	$\sum \abs{a_n}0^n$ converge a 0 trivialmente.
\end{proof}

\begin{defi}
	Sea $\sum a_nx^n$ una serie de potencias y sea $I$ el intervalo de los $r \geq$ tales que
	$\sum \abs{a_n}r^n$ converge. Llamamos radio de convergencia de la serie a $R$ el
	extremo superior del intervalo $I$. Denominamos dominio de convergencia de la serie
	al intervalo $(-R,R)$
\end{defi}

\begin{obs*}
	La serie puede converger en los puntos frontera del dominio de convergencia.
\end{obs*}

\begin{obs*}
	Los casos extremos corresponden a $R = 0$ (la serie solo converge para $x = 0$)
	y $R = +\infty$ (la serie converge para todo $x$).
\end{obs*}

\begin{thm}[de Cauchy-Hadamard]
	Sea $\sum a_nx^n$ una serie de potencias. Su radio de convergencia $R$ viene dado por
	\[
		\frac{1}{R} = \limsup \abs{a_n}^{\sfrac{1}{n}}
	\]
	La serie de potencias es absolutamente convergente si $\abs{x} < R$ y es divergenete
	si $\abs{x} > R$.
\end{thm}

\begin{obs*}
	A priori no se puede afirmar nada cuando $\abs{x} = R$.
\end{obs*}

\begin{proof}
	Separaremos la demostración en varios casos
	\begin{itemize}
		\item Caso $0 < R < + \infty$. Sea $0 < \abs{x} < R$. Existe $C < 1$ tal que
		 	\[
				\abs{x} < CR \implies \frac{1}{R} < \frac{C}{\abs{x}}
			\]
			Por lo tanto, si $n$ es suficientemente grande
			\[
				\abs{a_n}^{\sfrac{1}{n}} \leq \frac{C}{\abs{x}} \implies
				\abs{a_nx^n} \leq C^n
			\]
			Como $C^n$ es la serie geométrica de razón $C < 1$, la serie converge.

			Sea ahora $\abs{x} > R$, tenemos que $\frac{1}{R} > \frac{1}{\abs{x}}$.
			Hay infinitos $n$ tal que
			\[
				\abs{a_n}^{\sfrac{1}{n}} > \frac{1}{\abs{x}} \implies \abs{a_nx^n} > 1
			\]
			Por lo tanto $a_nx^n$ no tiende a 0 y por lo tanto la serie no converge.
		\item Caso $R = +\infty$. Entonces $\limsup \abs{a_n}^{\sfrac{1}{n}} = 0$.
			Por lo tanto, para $n$ suficientemente grande $\exists C < 1$ tal que
			$\abs{a_n}^{\sfrac{1}{n}} < \frac{C}{\abs{x}} \implies \abs{a_nx^n} < C^n$ y por lo tanto
			la serie converge.
		\item Caso $R = 0$. Entonces $\forall x$ hay infinitos $n$ tales que
		 	$\abs{a_n}^{\sfrac{1}{n}} > \frac{1}{\abs{x}} \implies \abs{a_nx^n} > 1 \neq 0$
			y por lo tanto, la serie diverge.
	\end{itemize}
\end{proof}

\begin{obs}
	El radio de convergencia también se puede calcular con las expresiones
	\[
		\frac{1}{R} = \lim \abs{a_n}^{\sfrac{1}{n}} \qquad
		\frac{1}{R} = \lim \frac{\abs{a_{n+1}}}{\abs{a_N}}
	\]
	Suponiendo que los límites existan.
\end{obs}

\begin{example*}
	\begin{itemize}
		\item[]
		\item $\sum n!x^n$, $\frac{1}{R} = \lim \frac{(n+1)!}{n!} = \lim (n+1)
			= +\infty \implies R = 0$
		\item $\sum x^n$, $\frac{1}{R} = \lim 1^{\sfrac{1}{n}} = 1^0 = 1 \implies
			R = 1$
		\item $\sum \frac{x^n}{n!}$, $\frac{1}{R} = \lim \frac{n!}{(n+1)!} =
			\lim \frac{1}{n+1} = 0 \implies R = +\infty$
		\item Las $\sum\limits_{n \geq 0} x^n$, $\sum\limits_{n \geq 1} \frac{x^n}{n}$
			y $\sum \frac{1}{n^2}$ tienen $R = 1$, pero tienen comportamiento
			distinto en la frontera.
	\end{itemize}
\end{example*}

\begin{defi}
	Si una serie de potencias $\sum a_n x^n$ tiene radio de convergencia $R > 0$,
	define una función
	\[\begin{aligned}
		f \colon (-R,R) &\to \real \\ x &\mapsto f(x) = \sum a_n x^n
	\end{aligned}\]
\end{defi}

\begin{obs*}
	Se puede probar que $f$ es continua, integrable y derivable
	``término a término''
\end{obs*}

\begin{obs*}
	La serie derivada término a término tiene radio de convergencia $R$, y por
	lo tanto, la función es de clase $\C^\infty$
\end{obs*}

\begin{proof}
	Primero, consideramos la función derivada $f^\prime (x) = \sum\limits_{n \geq 0}
	n a_n x^{n-1} = \sum\limits_{k \geq 0} (k+1) a_{k+1} x^k$, calculamos ahora
	el raido de convergencia por la definición.
	\[
		\limsup \abs{n+1}^{\sfrac{1}{n}}\abs{a_{n+1}}^{\sfrac{1}{n}} =
		\limsup \abs{n}^{\sfrac{1}{n-1}}\abs{a_{n}}^{\sfrac{1}{n-1}} =
		\limsup \left( n^{\sfrac{1}{n}} \abs{a_n}^{\sfrac{1}{n}} \right)^
		{\frac{n}{n-1}} = \frac{1}{R}
	\]
	Además
	\[
		f^{(k)}(0) = k! a_k \rightarrow f(x) = \sum_{k \geq 0} \frac{f^{(k)}(0)}
		{k!}x^k
	\]
\end{proof}

\begin{defi*}
	Una función tal que alrededor de cada punto se puede expresar como una serie
	de potencias (convergente) se llama analítica.
\end{defi*}

\begin{defi}
	Sea $D$ un intervalo abierto tal que $0 \in D$ y sea $f \colon D \to \real$
	de clase $\C^\infty$. Entonces, $f$ define una serie de potencias
	\[
		\sum_{n \geq 0} \frac{f^{(n)}(0)}{n!} x^n
	\]
	que es la serie de Taylor de $f$ (centrada en 0).
\end{defi}

\begin{prop*}
	Sea $D$ un intervalo abierto tal que $0 \in D$ y sea $f \colon D \to \real$
	de clase $\C^\infty$. Suponemos que la serie de Taylor de $f$ tiene radio
	de convergencia $R > 0$. Recordando la formula de Taylor $f(x) = P_n(x) + R_n(x)$
	(donde $P_n$ es el polinomio de Taylor de grado $\leq n$ de $f$ en 0, y $R_n$
	el correspondiente residuo de Taylor), por tanto en $D \cap (-R,R)$
	\[
		f(x) = \sum_{n \geq 0} \frac{f^{(n)}(0)}{n!}x^n \iff
		\lim_{n \to \infty} R_n(x) = 0
	\]
\end{prop*}

\begin{obs}
	Hay funciones $f \colon \real \to \real$ de clase $\C^\infty$ tales que su
	serie de Taylor (centrada en 0) converge para todo $x$, pero no coincide con
	$f(x)$ en ningún punto salvo el origen.

	Hay funciones $f \colon \real \to \real$ de clase $\C^\infty$ tales que su
	serie de Taylor (centrada en 0) tiene radio de convergencia 0.
\end{obs}

\begin{example*}
	La función $f(x) = \begin{cases} 0 \quad \text{si } x \leq 0 \\
	e^{-\sfrac{1}{x}} \quad \text{si } x > 0 \end{cases}$
	\begin{center}
    \begin{tikzpicture}
    \begin{axis}[
    	xmin = -1,
		xmax = 2,
		ymin = -1.2,
		ymax = 1.2,
		samples = 100,
		axis lines = center
    ]
		\addplot[domain=-1:0,color=blue,thick]{0};
		\addplot[domain=0:2,color=blue,thick]{e^(-1/x)};
    \end{axis}
    \end{tikzpicture}
    \end{center}
	Su serie de Taylor es nula ($f^{(k)}(0) = 0$ $\forall k \in \n$). Pero $f$ no
	se anula en ningún entorno de 0 $\implies$ $f$ no coincide con la serie de
	Taylor en ningún entorno de 0.
\end{example*}

\begin{prop}[Algunas series de Taylor importantes]
	\begin{itemize}
		\item[]
		\item $\displaystyle e^x = \sum_{n \geq 0} \frac{x^n}{n!}
			\qquad \forall x \in \real$
		\item $\displaystyle \cos(x) = \sum_{n \geq 0} (-1)^{n}\frac{x^{2n}}{(2n)!}
			\qquad \forall x \in \real$
		\item $\displaystyle \sin(x) = \sum_{n \geq 0} (-1)^{n}\frac{x^{2n+1}}{(2n+1)!}
			\qquad \forall x \in \real$
		\item $\displaystyle \log(1+x) = \sum_{n \geq 1} (-1)^{n+1} \frac{x^n}{n}
			\qquad \abs{x} < 1$
		\item $\displaystyle (1+x)^p = \sum_{n \geq 0} {p \choose n} x^n \qquad
			\begin{cases} x \in \real \quad \text{si } p \in \n \\ \abs{x} < 1
			\quad \text{si } p \notin \n \end{cases}$
		\item $\displaystyle a = \sum_{n \geq 0} (-1)^n x^n \qquad \abs{x} < 1$
	\end{itemize}
\end{prop}

\begin{example*}
	\begin{itemize}
		\item[]
		\item $\displaystyle \sum_{n \geq 0} \frac{1}{n!} = e^1 = e$
		\item $\displaystyle \frac{1}{(1-x)} = \sum_{n \geq 0} x^n \implies
			\sum_{n \geq 1} n x^{n-1} = \left( \frac{1}{(1-x)} \right)' =
			\frac{1}{(1-x)^2}$.

			Por lo tanto
			\[
				\sum_{n \geq 1} nx^n = \frac{x}{(1-x)^2}
			\]
		\item $f = \arctan(x)$ con $\abs{x} < 1$(a partir de $f'(x) = \frac{1}{1+x^2}$)
			\[
				\frac{1}{1+x^2} = \sum_{n \geq 0} (-1)^n x^{2n} \implies
				\arctan(x) = \int \frac{1}{1+x^2} \dif x = \sum_{n \geq 0} (-1)^n
				\frac{x^{2n+1}}{2n+1}
			\]
	\end{itemize}
\end{example*}

\subsubsection{Series de números complejos}

\begin{defi*}
	La definición de serie, de serie convergente y de serie absolutamente convergente,
	es la misma si, en vez de considerar números reales, consideramos números
	complejos.
\end{defi*}

\begin{prop*}
	Una serie $\sum c_n$ de números complejos es convergente si y solo si lo son
	separadamente sus partes real e imaginaria.
\end{prop*}

\begin{prop*}
	Toda serie $\sum c_n$ de números complejos absolutamente convegente, es
	convergente.
\end{prop*}

\begin{obs*}
	El estudio de las series de potencias es completamente análogo. En el caso
	complejo, si la serie de potencias $\sum c_n z^n$ tiene radio de convergencia
	$R$ ($\frac{1}{R} = \limsup \abs{c_n}^{\sfrac{1}{n}}$). Entonces, el dominio de
	convergencia es un disco abierto $\abs{z} < R$ del plano complejo.
\end{obs*}

\begin{obs*}
	La serie de Taylor de la función exponencial real permite definir la exponencial
	compleja como
	\[
		e^z = \sum_{n \geq 0} \frac{z^n}{n!}
	\]
	para todo $z \in \cx$.
\end{obs*}

\begin{prop*}
	Tomando $z \in \cx$ un imaginario puro, y separando las partes real e imaginaria
	de las potencias, obtenemos la formula de Euler.
	\[
		e^{ix} = \cos(x) + i\sin(x)
	\]
	En particular para $x = \pi$, se tiene que $e^{i\pi} + 1 = 0$.
\end{prop*}



\subsection{Integrales impropias: definición y ejemplos}

\begin{defi}
	Sea $D \subset \real$ un intervalo, y $f \colon D \to \real$ una función.
	Diremos que $f$ es localmente integrable si es integrable para todo intervalo
	compacto $K \subset D$.
\end{defi}
\begin{obs}
	Si consideramos por ejemplo, $f \colon [a,b) \to \real$ con $a < b \leq +\infty$.
	Entonces, $f$ es localmente integrable si es integrable en cualquier intervalo
	$[a,M]$ donde $a < M < b$.

	En tal caso, podemos estudiar la integral impropia
	\[
		\int^b_a f := \lim_{M \to b^-} \int^M_a f
	\]
\end{obs}

\begin{obs*}
	A veces se dice que una integral impropia es
	\begin{itemize}
		\item De primera especie, si el intervalo no es acotado.
		\item De segunda especie, si la función no es acotada.
		\item De tercera especie, sin ni la función ni el intervalo son acotados.
	\end{itemize}
\end{obs*}

\begin{defi}
	Diremos que la integral impropia es convergente, si $\exists\abs{\int^b_a f}
	<+\infty$ y divergente , di $\exists\int^b_a f = \pm\infty$
\end{defi}

\begin{obs}
	De forma totalmente análoga, podemos considerar la integral impropia de una
	función $f \colon (a,b] \to \real$ localmente integrable, con $-\infty \leq a
	< b$.
\end{obs}

\begin{defi}
	Consideramos una función localmente integrable $f \colon (a,b) \to \real$. Si
	tomamos un punto arbitrario $c$ tal que $a < c < b$, podemos descomponer
	\[
		\int^a_b f := \int^c_a f + \int^b_c f = \lim_{M \to a^+} \int^c_M +
		\lim_{N \to b^-} \int^N_c f
	\]
	y estudiar las dos integrales imporopias. Si las dos convergentes, una
	convergente y la otra divergente o las dos son divergentes con el mismo
	signo, entonces se define la integral imporpia del primer miembro como la
	suma de las dos del segundo miembro.
\end{defi}
\begin{obs*}
	Mas generalmente, podemos considerar una función localmente integrable definida
	en un dominio $D$ que sea unión finita y disjunta de intervalos. Entonces,
	definimos $\int_D f$ como la suma de las integrales sobre estos intervalos,
	suponiendo que sean convergentes.
\end{obs*}

\begin{prop}
	Consideramos una función integrable por Riemman, $f \colon [a,b] \to \real$.
	Entonces
	\[
		\int^a_b f = \lim_{M \to b^-} \int^M_a f
	\]
\end{prop}

\begin{obs*}
	Así, la notación introducida para las integrales impropias, no produce ningún
	conflicto con la notación habitual definida para integrales ``propias''
\end{obs*}

\begin{example}
	Algunos ejemplos de integrales inportantes son
	\begin{itemize}
		\item $\displaystyle \int^{+\infty}_1 \frac{\dif x}{x^\alpha}$ Es
			convergente sii $\alpha > 1$ $\left(\text{y vale }
			\frac{1}{\alpha-1}\right)$. Y es divergente si $\alpha \leq 1$.
			\[
				\lim_{M \to +\infty} \int^M_1 \frac{\dif x}{x^\alpha} =
				\begin{cases}
					\alpha = 1 \quad \lim\limits_{M \to +\infty}
					\left[\log x \right]^M_1 = + \infty \\
					\alpha \neq 1 \quad \lim\limits_{M \to +\infty}
					\left[\frac{x^{-\alpha+1}}{-\alpha+1}
					\right]^M_1 = \lim\limits_{M \to +\infty}
					\frac{M^{-\alpha+1}-1}{-\alpha+1}=\begin{cases}
						\frac{1}{\alpha-1} \quad
						\text{si $\alpha > 1$} \\
						+\infty \quad \text{si $\alpha < 1$}
					\end{cases}
				\end{cases}
			\]
		\item $\displaystyle \int^1_0 \frac{\dif x}{x^\alpha}$ Es convergente
			sii $\alpha < 1$ $\left(\text{y vale } \frac{1}{1-\alpha
			}\right)$. Y es dievergente si $\alpha \geq 1$.

			Podemos observar que
			\[
				\int^{+\infty}_0 \frac{\dif x}{x^\alpha} = \int^1_0
				\frac{\dif x}{x^\alpha} + \int^{+\infty}_1
				\frac{\dif x}{x^\alpha} = +\infty
			\]
			independientemente del valor de $\alpha$.
		\item $\displaystyle  \int^{+\infty}_0 e^{-ax} \dif x$ Es convergente
			sii $\alpha > 0$ $\left(\text{y vale }\frac{1}{\alpha}\right)$.
	\end{itemize}
\end{example}

\begin{example*}
	\begin{itemize}
		\item[]
		\item \[
				\int^{+\infty}_0 \frac{\dif x}{1+x^2} =
				\lim_{M \to +\infty} \left[\arctan x\right]^M_0 =
				\frac{\pi}{2}
			\]
			\[
				\int^{+\infty}_{-\infty} \frac{\dif x}{1+x^2} =
				\int^0_{-\infty} \frac{\dif x}{1+x^2} +
				\int^{+\infty}_0 \frac{\dif x}{1+x^2} = \pi
			\]
		\item \[
				\int^1_{-1} \frac{\dif x}{x^2} = \int^0_{-1}
				\frac{\dif x}{x^2} + \int^1_0 \frac{\dif x}{x^2} =
				+\infty
			\]
			Pero $\left[-\frac{1}{x}\right]^1_{-1} = -2$, es decir, que no
			poedemos aplicar la regla de Barrow.
		\item $\displaystyle \int^{+\infty}_{-\infty} x\dif x$ No existe
		\item \[
				\int^1_0 \frac{1}{x^2}\cos x \dif x =
				\left[ \cos \frac{1}{x} \right]^1_0 = \cdots
				\text{ No existe.}
			\]
	\end{itemize}
\end{example*}

\begin{obs}
	Las reglas de cálculo de integrales se aplican a las integrales impropias
	teniendo en cuenta quee hay que aplicar un límite. Explicitamos algunas.
\end{obs}

\begin{prop}[Linealidad]
	Si $\int^b_a f$, $\int^b_a g$ son integrales impropias convergentes, tambi\'en
	lo es $\int^b_a (f+g) = \int^b_a f + \int^b_a g$.
	\\
	Si $\int^b_a f$ es convergente y $\lambda \in \real$ $\int^b_a \lambda f =
	\lambda \int^b_a f$.
\end{prop}

\begin{prop}[Regla de Barrow]
	Si $f$ es continua y $f = F^\prime$ en $[a,b)$, entonces
	\[
		\int^b_a f = \lim_{M \to b} F(M) - F(a)
	\]
	Suponiendo que el límite exista.
\end{prop}

\begin{prop}[Integración por partes]
	Suponemos que $f,g$ son funciones de clase $\C^1$ en $[a,b)$. Entonces
	\[
		\int^b_a f^\prime g = \lim_{t \to b} \left[ f(x)g(x)\right]^{x=t}_{x=a}
		- \int^b_a g^\prime f
	\]
	Suponiendo que los miembros del segundo t\'ermino existan.
\end{prop}

\subsection{Criterios de convergencia de integrales impropias}

\begin{prop}[Criterio de Cauchy par aintegrales impropias]
	Sea $f \colon [a,b) \to \real$ una función localmente integrable. La integral
	impropia $\int^b_a f$ es convergente sii $\forall \varepsilon > 0$ $\exists c_0
	\in [a,b)$ tal que
	\[
		c_0 \leq c_1 < c_2 < b \implies \abs{\int^{c_2}_{c_1} f} < \varepsilon
	\]
\end{prop}
\begin{proof}
	Es consecuencia del criterio de Cauchy aplicado a la funcion
	\[\begin{aligned}
		F \colon [a,b) &\to \real \\
		x &\mapsto F(x) = \int^x_a f
	\end{aligned}\]
	y el límite $\lim\limits_{x \to b^-}F(x)$ existe sii para todo $\varepsilon >0$
	existe $c_0$ tal que
	\[
		a \leq c_0 \leq c_1 < c_2 < b \implies\abs{F(c_2)-F(c_1)} =
		\abs{\int^{c_2}_{c_1} f} < \varepsilon		
	\]
\end{proof}

\begin{defi}
	Diremos que una integral impropia $\int^b_a f$ es absolutamente convergente,
	si la integral impropia $\int^b_a \abs{f}$ es convergente.
\end{defi}

\section{Geometría Proyectiva}
\subsection{Contexto e idea}
Segun el programa de Erlegen, una geometría se basa en el estudio de invariantes
al aplicar unas ciertas transformaciones. Así tenemos que las geometrías estan
"incluidas" en las superiores.

Así, la geometría euclidea, que estudia las transformaciones ortogonales, estaría
"incluída" en la geometría afín, que estudia las tansformaciones lineales. Aquí
se muestra una tabla con las distintas geometrías en la cual cada una esta
"incluída" en la anterior
\begin{center}
	\begin{tabular}{|c|c|}
		\hline Geometría & Transformaciones de estudio \\
		\hline \hline
		G. euclídea  & transformaciones ortogonales \\ \hline
		G. afín & transformaciones lineales \\ \hline
		G. proyectiva & proyectividades \\ \hline
		G. Algebraica & transformaciones por polinomios \\ \hline
		G. Analítica & transformaciones por funciones analíticas \\ \hline
		G. Diferencial & transformaciones por funciones de clase
		$\mathcal{C}^\infty$ \\ \hline
		Topología & transformaciones por funciones de clase $\mathcal{C}^0$\\ \hline
	\end{tabular}
\end{center}

\subsubsection{Problema matemático}
\begin{center}
	\begin{tikzpicture}
	\begin{axis}[%
	hide axis,
	width= 15 cm,
	enlargelimits=0.1
	]
	
	\addplot3[patch, patch type=rectangle,opacity=.75,color=yellow]
	coordinates {(0,3,0) (-3,3,0) (-3,-3,0) (0,-3,0)};
	\addplot3[patch, patch type=rectangle,opacity=.75,color=orange]
	coordinates {(0,3,-3) (0,-3,-3) (0,-3,3) (0,3,3)}
	node[label={[black]right:$\pi_2$}]{};
	\addplot3[patch, patch type=rectangle,opacity=.75,color=yellow]
	coordinates {(3,-3,0) (0,-3,0) (0,3,0) (3,3,0)}
	node[label={[black]$\pi_1$}]{};
	
	\addplot3[color=blue,mark=*] coordinates {(1,0,1)}
	node[label=above:{$O$}]{};
	
	\draw [color=red] (axis cs:1,-3,0) -- (axis cs:1,3,0)
	node[label=$r_1$]{};
	\draw [color=red] (axis cs:0,-3,1) -- (axis cs:0,3,1)
	node[label=50:$r_2$]{};
	
	\draw [color=blue] (axis cs:0.5,1,0) -- (axis cs:1.5,-1,0)
	node[label=below:$s_1$]{};
	\draw [color=blue] (axis cs:0.5,-1,0) -- (axis cs:1.5,1,0)
	node[label=right:$s_2$]{};
	
	\addplot3[color=green,mark=*] coordinates {(2,2,0)}
	node[label=$p$]{};
	\addplot3[color=green,mark=*] coordinates {(0,-2,2)}
	node[label=$\phi(p)$]{};
	\draw [color=green] (axis cs:2,2,0) -- (axis cs:0,-2,2);
	
	\addplot3[color=green,mark=*] coordinates {(2.5,-1,0)}
	node[label=$p'$]{};
	\addplot3[color=green,mark=*] coordinates {(0,0.67,1.67)}
	node[label=$\phi(p')$]{};
	\draw [color=green] (axis cs:2.5,-1,0) -- (axis cs:0,0.67,1.67);
	\end{axis}
	\end{tikzpicture}
\end{center}
La idea básica del problema es enviar los puntos de $\pi_1$ a $\pi_2$ mediante la
siguiente aplicación
\[
\begin{aligned}
\phi \colon \pi_1 &\to \pi_2 \\
p &\mapsto \overline{pO} \cap \pi_2
\end{aligned}
\]
\begin{obs}
	\begin{enumerate}[i)]
		\item[]
		\item $\phi$ manda rectas a rectas
		\item\label{item:prob_r1} $\phi$ no esta definida en $r_1$
		\item\label{item:prob_r2} $r_2$ no esta en la imagen de $\phi$
		\item $\phi(s_1)$ y $\phi(s_2)$ son pararelas, por lo tanto, $\phi$ no
		mantiene el paralelismo
		\item $\phi$ no mantiene el tipo de cónica afín
	\end{enumerate}
\end{obs}
Veremos que la solucion para \ref{item:prob_r1} y \ref{item:prob_r2} consistirá
en añadir puntos en el $\infty$.

\subsection{Definición y caracterizaciones del espacio proyectivo}
\begin{defi}
	Sea $\k$ un cuerpo, $\E$ un $\k$-e.v. de $\dim n+1$. El espacio proyectivo asociado a $\E$ es 
	\[\Po (\E) = \setb{\text{s.e.v. de } \dim 1 \text{ de } \E}\]
	Diremos que $\Po (\E)$ tiene dimensión $n$.
\end{defi}
\begin{obs}
	Se cumple $\Po (\E) = (\E \setminus \setb{0}) / \sim$, donde $v \sim v'  \iff \exists \lambda \neq 0, \ v' = \lambda v$ para
	$v, v' \in \E \setminus \setb{0}$.
\end{obs}
\begin{defi}
	Tenemos la siguiente aplicación $\pi$ dada por el paso al cociente.
	\[
	\begin{aligned}
	\pi \colon \E \setminus \setb{0} &\to \Po (\E)\\
	v &\mapsto \pi(v) = [v]\\
	\setb{\text{s.e.v. de } \dim 1 \text{ de } \E} &\leftrightarrow [v]
	\end{aligned}
	\]
\end{defi}
\begin{defi}
	A los elementos de $\Po (\E)$ los llamaremos puntos de $\Po (\E)$.
	\[p = \pi(v) = [v]\]
\end{defi}
\begin{obs}
	\begin{itemize}
		\item[]
		\item Si $\E$ no es relevante, $\Po^n = \Po (\E)$.
		\item Si queremos remarcar $\k$, $\Po_\k^n = \Po (\E)$.
		\item Normalmente $\Po+\k^n = \Po(\k^{n+1}) = (\k^{n+1} \setminus {0})/\sim$.
	\end{itemize}
\end{obs}
\begin{example}
	\begin{enumerate}
		\item []
		\item\label{item:po1r} $\Po^1_{\real}$.
		\begin{center}
		\begin{tikzpicture}
		\begin{axis}[%
		axis equal,
		enlargelimits=0,
		axis lines=center,
		ticks=none,
		ymin=-4,ymax=4,
		xmin=-4,xmax=4
		]
		
		\addplot[color=orange] {2*x};
		\addplot[mark=*,color=red] coordinates {(1,2)};
		\addplot[color=orange] {0.5*x};
		\addplot[mark=*,color=red] coordinates {(1,0.5)};
		\addplot[color=orange] {4*x};
		\addplot[mark=*,color=red] coordinates {(1,4)};
		\addplot[color=orange] {-4*x};
		\addplot[mark=*,color=red] coordinates {(1,-4)};
		\addplot[color=orange] {-1*x};
		\addplot[mark=*,color=red] coordinates {(1,-1)};
		\addplot[color=orange] {-0.5*x};
		\addplot[mark=*,color=red] coordinates {(1,-0.5)};
		\addplot[color=orange] {12*x};
		
		\addplot[mark=*,color=white] coordinates {(0,0)};
		\addplot[mark=o] coordinates {(0,0)};
		
		\draw [color=red] (axis cs:1,-4) -- (axis cs: 1,1)
		node[label=right:$\bb{A}_\real^1$]{};
		\draw [color=red] (axis cs:1,1) -- (axis cs:1,4);
		\end{axis}
		\end{tikzpicture}
		\end{center}
		Cada objeto de $\Po^1_\real$ (rectas en naranja) tiene su equivalente en
		$\bb{A}^1_\real$ (puntos en rojo), a excepción de la recta $x=0$. Para solucionar
		este problema, podemos usar la siguiente representación
		\begin{center}
		\begin{tikzpicture}
		\begin{axis}[axis equal image,
		axis lines=center,
		ticks=none,
		ymin=-4,ymax=4,
		xmin=-4,xmax=4]
		
			\addplot[color=orange] {2*x};
			\addplot[mark=*,color=red] coordinates {(0.447214,0.894428)};
			\addplot[color=orange] {0.5*x};
			\addplot[mark=*,color=red] coordinates {(0.894427,0.4472135)};
			\addplot[color=orange] {4*x};
			\addplot[mark=*,color=red] coordinates {(0.242536,0.970144)};
			\addplot[color=orange] {-4*x};
			\addplot[mark=*,color=red] coordinates {(0.242536,-0.970144)};
			\addplot[color=orange] {-1*x};
			\addplot[mark=*,color=red] coordinates {(0.707107,-0.707107)};
			\addplot[color=orange] {-0.5*x};
			\addplot[mark=*,color=red] coordinates {(0.894427,-0.4472135)};
			
			\addplot[mark=*,color=blue] coordinates {(0,-1)}
				node[label=left:$v_1$]{};
			\addplot[mark=*,color=blue] coordinates {(0,1)}
				node[label=left:$v_2$]{};
			
			\addplot[mark=*,color=white] coordinates {(0,0)};
			\addplot[mark=o] coordinates {(0,0)};
			
			
			\addplot [domain=-90:90, samples=21, color=red] ({cos(x)},{sin(x)})
				node[label=50:$C$]{};
		\end{axis}
		\end{tikzpicture}
		\end{center}
		Pero el problema ahora viene por que $v_1$ y $v_2$ corresponden al mismo elemento de
		$\Po^1_\real$. Lo podemos solucionar de la siguiente forma:
		\begin{center}
		\begin{tikzpicture}
		\begin{axis}[
			axis lines=center,
			axis equal,
			hide axis,
			enlargelimits=0.1
		]
			\addplot [domain=-180:180, samples=50, color=red] ({cos(x)},{sin(x)});
			
			\addplot [mark=*,color=blue] coordinates {(-1,0)}
				node[label=left:$\infty$]{};
		\end{axis}
		\end{tikzpicture}
		\end{center}
		De tal forma que 
		\[
			\bb{A}^1_\real \cup \setb{\infty} = \Po^1_\real = C/(v_1 \sim v_2)
		\]
		\item $\Po^2_{\real}$.
		\begin{center}
		\begin{tikzpicture}
		\begin{axis}[%
		axis equal,
		ticks=none,
		width= 10 cm,
		enlargelimits=0,
		axis lines=center,
		view={10}{20},
		ymin=-4,ymax=4,
		xmin=-4,xmax=4
		]
		
			\draw [color=orange] (axis cs:-4,-4,-4) -- (axis cs:4,4,4);
			\addplot3 [mark=*,color=red] coordinates {(1,1,1)};
			\draw [color=orange] (axis cs:4,-2,-4) -- (axis cs:-4,2,4);
			\addplot3 [mark=*,color=red] coordinates {(1,-0.5,-1)};
			\draw [color=orange] (axis cs:-4,-2,0) -- (axis cs:4,2,0);
			\addplot3 [mark=*,color=red] coordinates {(1,0.5,0)};
			\draw [color=orange] (axis cs:-4,8,-12) -- (axis cs:4,-8,12);
			\addplot3 [mark=*,color=red] coordinates {(1,-2,3)};
			
			\addplot3[mark=*,color=white] coordinates {(0,0,0)};
			\addplot3[mark=o] coordinates {(0,0,0)};
			
			\addplot3 [patch, patch type=rectangle,opacity=.5,color=red]
			coordinates { (1,4,4) (1,-4,4) (1,-4,-4) (1,4,-4) }
			node[pin={[black]right:$\bb{A}^2_\real$}]{};
		\end{axis}
		\end{tikzpicture}
		\end{center}
		Cada objeto de $\Po^2_\real$ (rectas en naranja) tiene su equivalente en
		$\bb{A}^2_\real$ (puntos en rojo), a excepción de las rectas del plano $x=0$.
		Para solucionar este problema, procedemos de manera análoga a \ref{item:po1r}
		\begin{center}
		\begin{tikzpicture}
		\begin{axis}[%
		axis equal,
		ticks=none,
		width= 10 cm,
		enlargelimits=0,
		axis lines=center,
		view={20}{20},
		ymin=-2,ymax=2,
		xmin=-2,xmax=2
		]
			\draw [color=orange] (axis cs:-4,-4,-4) -- (axis cs:4,4,4);
			\draw [color=orange] (axis cs:4,-2,-4) -- (axis cs:-4,2,4);
			\draw [color=orange] (axis cs:-4,-2,0) -- (axis cs:4,2,0);
			\draw [color=orange] (axis cs:-4,8,-12) -- (axis cs:4,-8,12);
			
			\addplot3[mark=*,color=white] coordinates {(0,0,0)};
			\addplot3[mark=o] coordinates {(0,0,0)};
			
			\addplot3[
			color=blue,
			samples=21,
			variable = \v,
			domain = 0:360,
			ultra thick
			]
			({0},{cos(v)},{sin(v)});
			
			\addplot3[%
			opacity = 0.5,
			surf,
			color=red,
			faceted color=red,
			z buffer = sort,
			samples = 21,
			variable = \u,
			variable y = \v,
			domain = -90:90,
			y domain = 0:180,
			]
			({cos(u)*sin(v)}, {sin(u)*sin(v)}, {cos(v)});
			
			\addplot3 [mark=none,color=red] coordinates {(1,0,0)}
				node[label=50:$C$]{};
		\end{axis}
		\end{tikzpicture}
		\end{center}
		De nuevo, nos encontramos con el problema de que hay elementos de $C$ (en azul)
		que se corresponden con el mismo elemento de $\Po^2_\real$, No obstante, podemos
		visualizarlo como
		\begin{center}
		\begin{tikzpicture}
		\begin{axis}[axis equal,
		ticks=none,
		width= 10 cm,
		enlargelimits=0,
		axis lines=center,
		hide axis,
		view={10}{20},
		ymin=-1.2,ymax=1.2,
		xmin=-1.2,xmax=1.2
		]
			\addplot3[%
			opacity = 0.5,
			surf,
			color=red,
			faceted color=red,
			z buffer = sort,
			samples = 21,
			variable = \u,
			variable y = \v,
			domain = 0:180,
			y domain = 90:270,
			]
			({cos(u)*sin(v)}, {sin(u)*sin(v)}, {cos(v)});
			
			\addplot3[
			color=blue,
			samples=30,
			variable = \v,
			domain = 0:360,
			ultra thick
			]
			({cos(v)},{sin(v)},{0});
			
			\addplot3 [mark=*,color=blue] coordinates {(0,0,0)};
			\addplot3 [mark=*,color=blue] coordinates {({cos(30)},{sin(30)},0)};
			\addplot3 [mark=*,color=blue] coordinates {({cos(210)},{sin(210)},0)};
			\draw (axis cs: {cos(30)}, {sin(30)}, 0) [color=blue] --
				(axis cs: {cos(210)}, {sin(210)}, 0);
				
			\addplot3 [mark=*,color=blue] coordinates {({cos(120)},{sin(120)},0)};
			\addplot3 [mark=*,color=blue] coordinates {({cos(300)},{sin(300)},0)};
			\draw (axis cs: {cos(120)}, {sin(120)}, 0) [color=blue] --
				(axis cs: {cos(300)}, {sin(300)}, 0);
			
			
		\end{axis}
		\end{tikzpicture}
		\end{center}
		Y unir los puntos azules antipolares. Aunque, si lo tratamos de imaginar, este
		objeto no cabe en $\real^3$.
		\item $\Po^1_{\cx} = (\cx^2 \setminus {0})/\sim \ = \cx \cup \setb{\infty} = S^2$. Las igualdades por el momento son
        por analogía o intuición, más adelante se demostrarán. Aunque nos podemos hacer una idea basándonos en la proyección
        estereográfica
		\begin{center}
		\begin{tikzpicture}
		\begin{axis}[axis equal,
		ticks=none,
		width=13 cm,
		enlargelimits=0,
		axis lines=center,
		view={10}{20},
		ymin=-4,ymax=4,
		xmin=-4,xmax=4]
		
			\addplot3[patch,color=orange, patch type=rectangle, opacity=0.8] coordinates
			{(4,4,0) (4,-4,0) (-4,-4,0) (-4,4,0)};
		
			\addplot3[color=blue,mark=*] coordinates {(0,0,2)} node[pin=150:{$\infty$}]{};
			
			\addplot3[mark=*,color=blue] coordinates {(2,3,0)};
			\addplot3[mark=*,color=blue] coordinates {(0.47,0.71,1.53)};
			\draw (axis cs:0,0,2) [color=blue] -- (axis cs:2,3,0);
			
			\addplot3[mark=*,color=blue] coordinates {(2,-1.5,0)};
			\addplot3[mark=*,color=blue] coordinates {(0.78,-0.59,1.22)};
			\draw (axis cs:0,0,2) [color=blue] -- (axis cs:0.78,-0.59,1.22);
			
			\addplot3[%
			opacity = 0.5,
			surf,
			color=red,
			faceted color=red,
			z buffer = sort,
			samples = 21,
			variable = \u,
			variable y = \v,
			domain = 0:180,
			y domain = 0:360,
			]
			({cos(u)*sin(v)}, {sin(u)*sin(v)}, {cos(v)+1});
	
			\draw (axis cs:0.78,-0.59,1.22)[color=blue] -- (axis cs:2,-1.5,0);
			
		\end{axis}
		\end{tikzpicture}
		\end{center}
		
		\item $\Po^2_{\z / 2}$ contiene $7$ puntos pues las rectas de $\z^3$ solo contienen el $0$ y un punto.
	\end{enumerate}
\end{example}
\begin{obs}
	Hemos enunciado la definición algebraica de $\Po_{\real}^{n}$. Existe una definición axiomática que no 
	es igual en algunos casos patológicos.
\end{obs}

\subsection{Variedades lineales proyectivas}

\begin{defi}
	Sea $\E$ un $\k$-e.v. de dimensión $n+1$, sea $\Po^n = \Po(\E)$.
	Llamaremos variedad lineal (proyectiva) de dimensión $r$ a cualquier conjunto 
	de la forma:
	\[V = \pi(H \setminus \setb{0})\]
	donde $H \subseteq \E$ es un subespacio vectorial de dimensión $r+1$.
	
	Por convención defnimos la siguiente notación:
	\[V = \pi(H \setminus \setb{0}) = \pi(H)\]         
\end{defi}    
\begin{example}
	\begin{itemize}
		\item []
		\item $\Po^n = \pi(\E)$ es una variedad lineal de dimensión $n$.
		\item $p \in \Po^n$, $p = \pi(v) = \pi([v])$ es una varidead lineal de dimensión $0$.
		\item $\emptyset = \pi({\emptyset_{\E}})$ es una variedad lineal de dimensión $-1$.
	\end{itemize}
\end{example}
\begin{defi}
	\begin{itemize}
		\item []
		\item $\dim V = 1 \longrightarrow$ Recta
		\item $\dim V = 2 \longrightarrow$ Plano
		\item $\dim V = n-1 \longrightarrow$ Hiperplano
	\end{itemize}
\end{defi}
\begin{lema}
	$V = \pi(H\setminus\setb{0}) \iff H \setminus \setb{0} = \pi^{-1}(V)$
\end{lema}
\begin{ej}
	Demostrar el lema anterior.
\end{ej}
\begin{obs}
	Hay una biyección
	\[\setb{\text{s.e.v. de } \E}
	\mathrel{\mathop{\rightleftarrows}^{\mathrm{\pi}}_{\mathrm{\pi^{-1}}}}
	\setb{\text{variedades lineales de } \Po(\E)}\]
	\begin{itemize}
		\item con dimension $r+1 \leftrightarrow r$
		\item $H_1 \subseteq H_2 \iff V_1 \subseteq V_2$ con $V_i=\pi(H_i)$
		\item Las operaciones $+$ y $\cap$ definidas para subespacios, se definen
		a traves de esta biyección a variedades.
	\end{itemize}
\end{obs}
\begin{prop}
    Sean $V_1, V_2 \subseteq \Po^n$ variedades lineales $\implies V_1 \cap V_2$ variedad lineal que verifica:
    \begin{itemize}
        \item $V_1 \cap V_2 \subseteq V_1, V_2$
        \item Si $W$ variedad lineal, $W \subseteq V_1, V_2 \implies W \subseteq V_1 \cap V_2$
    \end{itemize}
\end{prop}
\begin{proof}
    Veamos que: $V_1 \cap V_2 = \pi(H_1 \cap H_2)$ si $V_i = \pi(H_i)$ \\
    $\supseteq$
    \begin{gather*} 
    \text{Si } v \in H_1 \cap H_2, p = [v] \in \pi(H_1), \pi(H_2) \implies p \in V_1 \cap V_2
    \end{gather*}
    $\subseteq$
    \begin{gather*}
        \text{Sea } p \in V_1 \cap V_2,
        \begin{rcases} p = [v_1], v_1 \in H_1 \\ p = [v_2], v_2 \in H_2 \end{rcases} \implies [v_1] = [v_2] \implies
        \exists \lambda \neq 0 \text{ t.q. } v_2 = \lambda v_1 \implies \\ \implies v_2 \in H_1 \implies v_2 \in H_1
        \cap H_2 \implies p = [v_2] \in \pi (H_1 \cap H_2)
    \end{gather*}
    El resto sigue de lo que sabemos de s.e.v.s.
\end{proof}
\begin{defi}
    Sean $V_1, V_2 \subseteq \Po^n$ variedades lineales, $V_i = \pi (H_i)$, definimos \textit{join} o variedad lineal generada como:
    \[
    V_1 \vee V_2 = \pi (H_1 + H_2)
    \]
\end{defi}
\begin{prop}
    \begin{itemize}
    \item[]
    \item $V_1, V_2 \subseteq V_1 \vee V_2$
    \item Sea $W$ una variedad lineal, $V_1, V_2 \subseteq W \implies V_1 \vee V_2 \subseteq W$
    \end{itemize}
\end{prop}
\begin{proof}
    Propiedades de la suma de s.e.v.
\end{proof}
\begin{prop}
    Sean $V_1, V_2 \subseteq \Po^n$ variedades lineales:
    \begin{itemize}
        \item $V_1 \subseteq V_2 \implies \dim \left( V_1\right) \leq \dim \left( V_2 \right)$
        \item Si $V_1 \subseteq V_2$ y $\dim \left( V_1 \right) = \dim \left( V_2 \right) \implies V_1 = V_2$
    \end{itemize}
\end{prop}
\begin{prop}[Fórmula de Grassmann]
    \[ \dim \left( V_1 \cap V_2 \right) + \dim \left( V_1 \vee V_2 \right) = \dim \left( V_1 \right) + \dim \left( V_2 \right) \]
\end{prop}
\begin{proof}
    Propiedades s.e.v.
\end{proof}
\begin{example}
    \begin{enumerate}
        \item[]
        \item $\Po^2 \quad V_1, V_2$ rectas, tenemos que $\dim \left( V_1 \cap V_2 \right) + \dim \left( V_1 \vee V_2 \right) = 2$ (Grassmann).
        \begin{center} \begin{tabular}{|c|c|c|}
            \hline $\dim \left( V_1 \cap V_2 \right)$ & $1 \leq \dim \left( V_1 \vee V_2 \right) \leq 2$  & Posición relativa \\
            \hline \hline
            0 & 2 & Se cortan en un punto \\ \hline
            1 & 1 & Son la misma recta\\ \hline
        \end{tabular} \end{center}
        Por tanto, dos rectas en un plano, o se cruzan o son la misma.
        \item $\Po^3 \quad V_1, V_2$ rectas, tenemos que $\dim \left( V_1 \cap V_2 \right) + \dim \left( V_1 \vee V_2 \right) = 2$ (Grassmann).
        \begin{center} \begin{tabular}{|c|c|c|}
            \hline $\dim \left( V_1 \cap V_2 \right)$ & $1 \leq \dim \left( V_1 \vee V_2 \right) \leq 3$  & Posición relativa \\
            \hline \hline
            -1 & 3 & Se cruzan \\ \hline
            0 & 2 & Se cortan en un punto $\left( V_1 \vee V_2 \cong \Po^2 \right)$ \\ \hline
            1 & 1 & Son la misma recta\\ \hline
        \end{tabular} \end{center}
        \item $\Po^3 \quad V_1, V_2$ planos, tenemos que $\dim \left( V_1 \cap V_2 \right) + \dim \left( V_1 \vee V_2 \right) = 4$ (Grassmann).
        \begin{center} \begin{tabular}{|c|c|c|}
            \hline $\dim \left( V_1 \cap V_2 \right)$ & $2 \leq \dim \left( V_1 \vee V_2 \right) \leq 3$  & Posición relativa \\
            \hline \hline
            1 & 3 & Se cortan en una recta \\ \hline
            2 & 2 & Son el mismo plano\\ \hline
        \end{tabular} \end{center}
        \item $\Po^3 \quad V_1$ plano, $V_2$ recta, tenemos que $\dim \left( V_1 \cap V_2 \right) + \dim \left( V_1 \vee V_2 \right) = 3$ (Grassmann).
        \begin{center} \begin{tabular}{|c|c|c|}
            \hline $\dim \left( V_1 \cap V_2 \right)$ & $2 \leq \dim \left( V_1 \vee V_2 \right) \leq 3$  & Posición relativa \\
            \hline \hline
            0 & 3 & Se cortan en un punto \\ \hline
            1 & 2 & $V_2 \subseteq V_1$ \\ \hline
        \end{tabular} \end{center}
    \end{enumerate}
\end{example}
\begin{prop}
    Sean $p \in \Po^n, r, V, H \subseteq \Po^n$ (recta, variedad lineal, hiperplano, respectivamente).
    \begin{enumerate}
        \item $\dim \left( V \vee p \right) = \text{ ó } \begin{cases} \dim V & (\iff p \in V) \\ \dim V + 1 & (\iff p \notin V) \end{cases}$
        \item $\dim \left( r \cap H \right) = \text{ ó } \begin{cases} 1 & (\iff r \subseteq H) \\ 0 & (\iff r \nsubseteq H) \end{cases}$
    \end{enumerate}
\end{prop}
\begin{proof}
    Fórmula de Grassmann.
\end{proof}
\begin{defi}
    Sean $V_1, V_2 \subseteq \Po^n, (V_i = \pi \left( H_i \right)$ variedades lineales, $V_1$ y $V_2$ son suplementarias $\iff H_1$ y $H_2$
    son complementarios.
\end{defi}
\begin{obs}
    \begin{gather*}
        V_1, V_2\text{ suplementarias }\iff H_1 \oplus H_2 = \E \iff \left. \begin{cases} (1) & H_1 + H_2 = \E \\ (2) & H_1 \cap H_2 = \emptyset \end{cases} \right\} \iff \\
        \iff \begin{cases} (1) & V_1 \vee V_2 = \Po^n \\ (2) & V_1 \cap V_2 = \emptyset \end{cases} \\
    \end{gather*}
\end{obs}
\begin{prop}
    Sean $V_1, V_2 \subseteq \Po^n$ variedades lineales:
    \begin{enumerate}
        \item $V_1, V_2$ son suplementarias $\implies \dim V_1 + \dim V_2 = n-1$
        \item Si $\dim V_1 + \dim V_2 = n-1$
        \[
        V_1, V_2 \text{ suplementarias} \iff V_1 \vee V_2 = \Po^n \iff V_1 \cap V_2 = \emptyset
        \]
    \end{enumerate}
\end{prop}
\begin{proof}
    \begin{enumerate}
        \item[]
        \item $V_1, V_2$ suplementarias $\iff H_1, H_2$ complementarios $\implies \overbrace{\dim H_1}^{\dim V_1 + 1} +
        \overbrace{\dim H_2}^{\dim V_2 + 1} = \dim \E = n+1$
        \item Fórmula de Grassmann.
    \end{enumerate}
\end{proof}
\begin{defi}
    Sea $\Po^n = \real(\E )$, y sean $p_0, \dots, p_m \in \Po^n$ tal que $p_i = [v_i], v_i \in \E$, decimos que $p_0, \dots,
    p_m$ son linealmente independientes (l.i.) $\iff v_0, \dots, v_m \in \E$ son linealmente independientes.
\end{defi}
\begin{obs}
    La independencia lineal no depende de los representantes elegidos.
\end{obs}
\begin{example}
    $\Po^2_\real = \Po(\real^3)$. Sean $p_0 = [(1,1,0)], p_1 = [(0,1,1)], p_2 = [(2,3,1)]$. Entonces $p_0$ y $p_1$ son l.i.,
    pero $p_0, p_1, p_2$ son l.d..
\end{example}
\begin{prop}
    Sean $p_0, \dots, p_m \in \Po^n$ puntos y $V \subseteq \Po^n$ una variedad lineal:
    \begin{enumerate}
        \item $\dim\left( p_0 \vee \dots \vee p_m \right) \leq m$
        \item $\dim\left( p_0 \vee \dots \vee p_m \right) = m \iff p_0, \dots, p_m$ son l.i.
        \item $\dim V = m \iff \exists p_0, \dots, p_m \in V$ l.i. tales que $V = p_0 \vee \dots \vee p_m
            \iff \forall p_0, \dots, p_m \in V \text{ l.i.}, V = p_0 \vee \dots \vee p_m$
    \end{enumerate}
\end{prop}
\begin{proof}
    Inmediata a partir de propiedades de s.e.v..
\end{proof}
\begin{defi}
    $p_0, \dots, p_r \in \Po^n (r \geq n)$ estan en posición general $\iff n+1$ puntos cualesquiera de ellos son l.i..
\end{defi}

\subsection{Sistemas de referencia proyectivos. Coordenadas proyectivas}

\begin{obs}
    $\Po^n = \Po(\E)$. Sea $B = \{v_0, \dots, v_n\}$ base de $\E$:
    \begin{center} \begin{tabular}{ccc}
        $p = [v] = [\lambda v] \qquad $ &
        $(v)_B = \begin{pmatrix} x_0 \\ \vdots \\ x_n \end{pmatrix}\qquad $&
        $(\lambda v)_B = \begin{pmatrix} \lambda x_0 \\ \vdots \\ \lambda x_n \end{pmatrix} \qquad$
    \end{tabular} \end{center}
    Vemos que las coordenadas estarían definidas salvor por multiplicar por $\lambda$.
\end{obs}
\begin{defi}
    $\Po^n = \Po(\E)$
    \begin{enumerate}
        \item Un sistema de referencia proyectivo en $\Po^n$ es:
            \[ R= \{ p_0, \dots, p_n; \bar{p} \} \]
            Donde $p_0, \dots, p_n, \bar{p}$ estan en posición general.
            \begin{itemize}
                \item $\{ p_0, \dots, p_n \}$ son los vertices de $R$
                \item $\bar{p}$ es el punto unidad
            \end{itemize}
        \item Fijado un sistema de referencia $R$. Sea $B = \{ u_0, \dots, u_n \}$ base de $\E$. Diremos que $B$ está adaptada a $R$ si y solo si:
            \[ \begin{cases} p_i = [u_i], \forall i = 0, \dots, n \\ \bar{p} = [u_0 + \dots + u_n ] \end{cases} \]
        \item \textit{Asignación de coordenadas proyectivas} \\
            $\Po^n = \Po(\E)$, Sea $R = \{ p_0, \dots, p_n; \bar{p} \}$, y $B = \{u_0, \dots, u_n\}$ una base adaptada. Sea $q \in \Po^n, q = [v]$, entonces 
            $q_R = v_B$. (Hace falta ver que la definición es consistente)
    \end{enumerate}
\end{defi}
\begin{prop}
    Sea $R$ un sistema de referencia en $\Po^n$. Sean $B_1 = \{ u_0, \dots, u_n\}, B_2 = \{ u'_0, \dots u'_n \}$ bases adaptadas a $R$. Sea $q = [v] \in \Po^n$, entonces $\exists \lambda \neq 0$ tal que $v_{B_2} = v_{B_1}$.
\end{prop}
\begin{proof}
    $p_i = [u_i] = [u'_i], \forall i = 0, \dots, n$. Por tanto, $\exists \lambda \neq 0$ tal que $u'_i= \lambda_i u_i, \forall i = 1, \dots, n$. Por otro lado,
    \begin{gather*}
        \bar{p} = [u_0 + \dots + u_n] = [u'_0 + \dots + u'_n] \implies \exists \lambda \neq 0 \text{ tal que } u'_0 + \dots + u'_n = \\ = \lambda_0 u_0 + \dots + \lambda_n u_n = \lambda (u_0 + \dots + u_n)
        \stackrel{B_1 \text{ base}}{\implies} \lambda_i = \lambda \forall i = 1, \dots, n
    \end{gather*}
\end{proof}
\begin{obs}
    $R = \{ p_0, \dots, p_n; \bar{p} \}$ en $\Po^n$
    \begin{enumerate}
        \item $q \in \Po^n, q_R = (a_0, \dots, a_n) = \lambda (a_0, \dots, a_n) \stackrel{\text{notación}}{=} (a_0 : \dots : a_n) \forall \lambda \neq 0$
        \item Ejemplos: \\ $\begin{aligned} (p_0&)_R = (1:0:\dots:0) \\ \vdots& \\ (p_n&)_R = (0:\dots:0:1) \end{aligned}$
        \item $q_R = (0: \dots : 0)$ no existe.
        \item Sea $\Po^2_\real = \Po(\real^3), R = \{ p_0, p_1, p_2; \bar{p} \} = \left\{ [(1,1,0)], [(1,0,1)], [(0,1,1)]; [(3,3,2)]\right\}$, entonces $B = \left\{ [(1,1,0)], [(1,0,1)], [(0,1,1)] \right\}$ no es una base
            adaptada, para que lo sea, debemos multiplicar $v_0, v_1, v_2$ por $\lambda_0, \lambda_1, \lambda_2$ tales que $\bar{p} = \lambda_0 p_0 + \lambda_1 p_1 + \lambda_2 p_2$. En este caso:
            \[
                \begin{cases} \lambda_0 + \lambda_1 = 3 \\ \lambda_0 + \lambda_2 = 3 \\ \lambda_1 + \lambda_2 = 2 \end{cases} \iff \begin{cases} \lambda_0 = 2 \\ \lambda_1 = 1 \\ \lambda_2 = 1 \end{cases}
            \]
            Y el sistema de referencia con la base $B$ adaptada es: \[ R' = \left\{ [(2,2,0)], [(1,0,1)], [(0,1,1)]; [(3,3,2)]\right\} \]
    \end{enumerate}
\end{obs}

%%%%%%%%%%%%%%%%%%%%%%%%%%%%%%%%%%
% YA NO FALTAN LUNES Y MIERCOLES %
%%%%%%%%%%%%%%%%%%%%%%%%%%%%%%%%%%
%%%%%%%%%%%%%%%%%%%%%%%%%%%%%%%%%%
% ¿POR QUE LEES ESTO?            %
%%%%%%%%%%%%%%%%%%%%%%%%%%%%%%%%%%

\begin{example}
    Consideremos el espacio $\Po^2_{\R}$, un sistema de referencia $\R = \{ p_0, p_1, p_2; \overline{p}\}$, con 
    $p_0 = [(1, 1, 0)]$, $p_1 = [(1, 0, 1)]$, $p_2 = [(0, 1, 1)]$ y $\overline{p} = [(3,3,2)]$.
    Queremos encontrar $B = \{u_0, u_1, u_2\}$ una base adaptada, es decir, que cumpla $\overline{p} = [u_0+u_1+u_2]$
    y que $p_i = [u_i]$. Comprobamos que para la base $B = \left \{ \begin{pmatrix} 2 \\ 2 \\ 0 \end{pmatrix},
    \begin{pmatrix} 1 \\ 0 \\ 1 \end{pmatrix}, \begin{pmatrix} 0 \\ 1 \\ 1 \end{pmatrix} \right \}$, se cumple
    \[(p_0)_{\R} = (1:0:0) \quad (p_1)_{\R} = (0:1:0)\]
    \[(p_2)_{\R} = (0:0:1) \quad (\overline{p})_{\R} = (1:1:1)\]
    El punto $q = [(3, 2, 1)]$ tiene las coordenadas $q_{\R} = w_B = (1:1:0)$, que, en un abuso de notación,
    escribiremos como $(1, 1, 0)$.
\end{example}

\begin{prop}
    Sean $\R_1, \R_2$ sistemas de referencia de $\Po^n$. Sea $q \in \Po^n$.
    \begin{enumerate}[i)]
        \item $\exists S \in M_{n+1}(k)$ invertible t.q. $q_{\R_1} = S q_{\R_2}$
        (matriz del cambio de sistema de referencia).
        \item Sean $B_1, B_2$ bases adaptadas a $\R_1, \R_2$ respectivamente. Podemos coger $S=S_{B_2, B_1}$.
        \item Cualquier matriz de cambio de referencia es de la forma $S=\lambda S_{B_2, B_1} \ (\lambda \neq 0)$.
    \end{enumerate}
\end{prop}
\begin{proof}
    Cambio de base en $\E$.
\end{proof}
\begin{obs}
    Sea $\R$ sistema de referencia en $\Po^n$, y $B$ una base adaptada a éste. Las variedades lineales se pueden describir con
    \begin{itemize}
        \item Ecuaciones paramétricas: Sea $V \subseteq \Po^n$ una variedad lineal de $\dim m$. Entonces
        $\exists p_0, \dots, p_m \in V$ l.i. t.q. $V = p_0 \vee \dots \vee p_m$ y para $q \in \Po^n$
        \[
            q \in V \iff w \in F \iff 
            \begin{cases} 
                \exists \alpha_0, \dots, \alpha_m \in \k \\ 
                w_B = \alpha_0(v_0)_B + \dots + \alpha_m(v_m)_B 
            \end{cases} 
            \iff
            \]
        \[
            \iff 
            \begin{cases}
                \exists \alpha_0, \dots, \alpha_m \in \k \\ 
                q_R = \alpha_0 (p_0)_R + \dots + \alpha (p_m)_R 
            \end{cases}
        \]
        Donde $V = \pi(F)$, $p_i = \pi(v_i)$, y $q = \pi(w)$.
        \item Ecuaciones implícitas: Las ecuaciones implícitas de $V$ en $\R$ coinciden con las ecuaciones implícitas de $F$ en
        $B$. Es decir, existe una matriz $A$ tal que los vectores $w_B = \begin{pmatrix} x0 \\ \vdots \\ x_n \end{pmatrix}$ que cumplen $A \begin{pmatrix} x0 \\ \vdots \\ x_n \end{pmatrix} = 
        \begin{pmatrix} 0 \\ \vdots \\ 0 \end{pmatrix}$ son todos los pertenecientes a $F$ y los puntos 
        $q_R = \begin{pmatrix} x0 \\ \vdots \\ x_n \end{pmatrix}$ que cumplen $A \begin{pmatrix} x0 \\ \vdots \\ x_n \end{pmatrix} = 
        \begin{pmatrix} 0 \\ \vdots \\ 0 \end{pmatrix}$ son todos los pertenecientes a $V$.
    \end{itemize}
\end{obs}
\begin{obs}
        Sea $\E$ un $\k$-e.v. de dimensión $n+1$, $F$ un subespacio vectorial de dimensión $m+1$, $\Po^n = \Po(\E)$ y $V = \pi(F)$
        una variedad lineal.
        Entonces
        \[\dim V = \dim F - 1 = \dim E - \rg A - 1 = n+1 - \rg A - 1 \implies m = \dim V = n - \rg A\]        
\end{obs}
\begin{obs}
    \begin{itemize}
        \item[]
        \item Para convertir de ecuaciones implícitas a paramétricas se debe resolver el sistema de ecuaciones homogéneo.
        \item Para convertir de ecuaciones paramétricas a implícitas se debe imponer
        \[
            \rg
            \left(
            \begin{array}{ccc|c}
                (p_0)_R & \dots & (p_m)_R & \begin{array}{c}x_0 \\ \vdots \\ x_n \end{array} \\
            \end{array}
            \right)
            =
            \rg
            \left(
            \begin{array}{ccc}
                (p_0)_R & \dots & (p_m)_R\\
            \end{array}
            \right)
        \]
    \end{itemize}
\end{obs}
\begin{example}
    Consideremos el espacio $\Po^2_{\real}$ con un sistema de referencia $\R = \{p_0, p_1, p_2, \overline{p}\}$, y los puntos
    $(q_0)_{\R} = (1, 1, 1)$ y $(q_1)_{\R} = (1, 2, 0)$. Encontremos las ecuaciones paramétricas e implícitas de la recta
    $V = q_0 \vee q_1$.
    \begin{itemize}
        \item Dado que los puntos son dados directamente y se hace su join, las ecuaciones paramétricas son tan solo
        las combinaciones lineales de $q_0$ y $q_1$. Más formalmente, si $(q)_{\R} = \begin{pmatrix} x_0 \\ x_1 \\ x_2 \end{pmatrix}$, 
        se cumple
        \[
            q \in V \iff \exists \alpha_0, \alpha_1 \text{ t.q. } \begin{pmatrix} x_0 \\ x_1 \\ x_2 \end{pmatrix}
            = \alpha_0 \begin{pmatrix} 1 \\ 1 \\ 1 \end{pmatrix} + \alpha_1 \begin{pmatrix} 1 \\ 2 \\ 0 \end{pmatrix}
        \]
        \item Para encontrar las ecuaciones implícitas de esta variedad, imponemos que $(x_0, x_1, x_2)$ sea combinación
        lineal de $q_0$ y $q_1$, es decir,
        \[
            q \in V \iff 
            \rg
            \left(
            \begin{array}{cc|c}
                1 & 1 & x_0 \\
                1 & 2 & x_1 \\
                1 & 0 & x_2 \\
            \end{array}
            \right)
            =
            \rg
            \left(
            \begin{array}{cc}
                1 & 1 \\
                1 & 2 \\
                1 & 0 \\
            \end{array}
            \right)
            =
            2
            \iff
            \begin{vmatrix}
                1 & 1 & x_0 \\
                1 & 2 & x_1 \\
                1 & 0 & x_2                
            \end{vmatrix}
            =
            0
        \]
    \end{itemize}
    Consideremos ahora la recta $V'$ dada por $x_0+x_1+x_2 = 0$. Para encontrar su expresión paramétrica,
    debemos resolver este sistema de ecuaciones. En este caso, podemos encontrar dos soluciones linealmente
    independientes y expresar todas las soluciones como combinaciones lineales de estas.
    \[\begin{pmatrix} x_0 \\ x_1 \\ x_2 \end{pmatrix} = \beta_0 \begin{pmatrix} 1 \\ -1 \\ 0 \end{pmatrix}
    + \beta_1 \begin{pmatrix} 0 \\ 1 \\ -1 \end{pmatrix}\]
    En este mismo ejemplo, $p_0 \vee p_1 : x_2 = 0$, $p_2 \vee \overline{p} : x_1-x_0 = 0$ y 
    $(p_0 \vee p_1) \cap (p_2 \vee \overline{p}) = (1, 1, 0)$. Análogamente para los demás índices.
\end{example}

\begin{example}
    Consideremos $\Po^2_{\sfrac{\z}{2}}$. Sus puntos se pueden expresar como $(a, b, c) \  a,b,c \in \sfrac{\z}{2}$. 
    Podemos representarlo como
    
    \begin{center}
\begin{tikzpicture}[scale=1]
    \draw (0,0) coordinate (A) --
    (3,0) coordinate (B)--
    ($ (3*0.5,{3*sin(60)}) $) coordinate (C) -- cycle;
    \coordinate (O) at
    (barycentric cs:A=1,B=1,C=1);
    \draw (O) circle [radius=3*1.717/6];
    \draw (C) -- ($ (A)!.5!(B) $) coordinate (LC); 
    \draw (A) -- ($ (B)!.5!(C) $) coordinate (LA); 
    \draw (B) -- ($ (C)!.5!(A) $) coordinate (LB); 
    \node[draw,
    circle,
    fill=cyan,
    inner sep=1.5pt, label={-90:\tiny $(0, 0, 1)$}] at (A) {};
    \node[draw,
    circle,
    fill=cyan,
    inner sep=1.5pt, label={-90:\tiny $(0, 1, 0)$}] at (B) {};
    \node[draw,
    circle,
    fill=cyan,
    inner sep=1.5pt, label={90:\tiny $(1, 0, 0)$}] at (C) {};
    \node[draw,
    circle,
    fill=cyan,
    inner sep=1.5pt, label={0:\tiny $(1, 1, 0)$}] at (LA) {};
    \node[draw,
    circle,
    fill=cyan,
    inner sep=1.5pt, label={180:\tiny $(1, 0, 1)$}] at (LB) {};
    \node[draw,
    circle,
    fill=cyan,
    inner sep=1.5pt, label={-90:\tiny $(0, 1, 1)$}] at (LC) {};
    
    \draw[->] 
    (0, 0.8) node[anchor=east] {\tiny $x_0 + x_1 + x_2 = 0$} --
    (0.69, 0.6);
\end{tikzpicture}
    \end{center}
    Como se puede observar, contiene siete puntos y siete rectas. Se denomina plano de Fano.
\end{example}

\begin{obs}
    %\begin{enumerate}[i)]
        %\item []
        %\item En $\Po ^2 _{\left( \sfrac{\z}{2} \right)}$ solamente hay $7$ puntos: $\left| \left\{ \left(a, b, c \right) | a, b, c \in \left\{0, 1 \right\} \right\} \right|  = 2^3 = 8$, pero debemos excluir $\left( 0, 0, 0 \right)$ porque no es punto.
        %\item 
    En $\Po ^2$, sean $\left( p_0 \right)_R = \begin{pmatrix} a_0 \\ b_0 \\ c_0 \end{pmatrix}$ y $\left( p_1 \right)_R = \begin{pmatrix} a_1 \\ b_1 \\ c_1 \end{pmatrix}$. Entonces, $V=p_0 \vee p_1 \colon \begin{pmatrix} x_0 \\ x_1 \\ x_2 \end{pmatrix} = \alpha \begin{pmatrix} a_0 \\ b_0 \\ c_0 \end{pmatrix} + \beta \begin{pmatrix} a_1 \\ b_1 \\ c_1 \end{pmatrix}$. 
    Es decir, aunque no haya operación suma definida entre puntos, está bien definido el conjunto de combinaciones lineales de ellos.
    %\end{enumerate}
\end{obs}
\begin{obs}
    \begin{enumerate}[i)]
        \item []
        \item 
            \[
                \left. \begin{array}{l} V=p_0 \vee \cdots \vee p_m \\ W= q_0 \vee \cdots \vee q_r \end{array} \right\} V \vee W = p_0 \vee \cdots \vee p_m \vee q_0 \vee \cdots \vee q_r.
            \]
            \begin{proof}
                Directa de espacios vectoriales.
            \end{proof}
        \item 
            \[
                \left. \begin{array}{l} V \colon A \begin{pmatrix} x_0 \\ \vdots \\ x_n \end{pmatrix} = 0 \\ W \colon B \begin{pmatrix} x_0 \\ \vdots \\ x_n \end{pmatrix} = 0 \end{array} \right\} V \cap W  \colon \begin{pmatrix} A \\ \hline B \end{pmatrix} \begin{pmatrix} x_0 \\ \vdots \\ x_n \end{pmatrix} = 0.
            \]
            \begin{proof}
                Directa de espacios vectoriales.
            \end{proof}
    \end{enumerate}
\end{obs}
\subsection{Dualidad}
La \textbf{dualidad} es una diferencia fundamental respecto de la geometría afín.
\begin{rec}[(Problema 3, Lista T1)]
    Consideremos $\E$ un espacio vectorial y su dual $\E^*$. Sea $\F \subseteq \E$ un subespacio vectorial. Entonces, $\F^\perp=\left\{w \in \E ^* | w \left( u \right) = 0 \hspace{0,2cm} \forall u\in\F \right\} $ es un subespacio vectorial de $\E^*$ y se satisface que $\dim \F^\perp = \dim \E - \dim \F$. Además, se tiene que
    \begin{itemize}
        \item $\F_1 \subseteq \F_2 \iff \F_2^\perp \subseteq \F_1^\perp$,
        \item $\left( \F_1 \cap \F_2 \right) ^\perp = \F_1^\perp + \F_2^\perp$,
        \item $\left( \F_1 + \F_2 \right)^\perp=\F_1^\perp \cap \F_2^\perp$,
        \item $\left( \F^\perp\right)^\perp=\F$.
    \end{itemize}
    Por tanto,
    \[
        \begin{array}{ccc}
             \boxed{\E = \E^{**}} & & \boxed{\E^*}\\
             \left\{\text{s.e.v. de }\dim = d \right\} & \substack{\perp \\ \longrightarrow \\  \longleftarrow \\ \perp} & \left\{\text{s.e.v. de }\dim = \dim \E - d \right\}
        \end{array}
    \]
    de manera que la operación $\perp$ define una biyección entre $\E$ y $\E^*$ que invierte el orden de las inclusiones e intercambia $+$ y $\cap$.
\end{rec}
\begin{defi}
    Sea $\Po^n=\Po\left(\E\right)$. Llamaremos espacio proyectivo dual de $\Po\left(\E\right)$ a
    \[
        \left( \Po^n \right)^* \equiv \Po\left(\E^*\right).
    \]
\end{defi}
\begin{obs}
    \begin{enumerate}[i)]
        \item []
        \item $\dim \left( \Po ^n \right) ^* = n$.
        \item $\left( \left( \Po ^n \right) \right) ^* = \Po^n. $
    \end{enumerate}
\end{obs}
\begin{defi}
    Sea $\F\subseteq \E$ y sea $V=\pi\left(\F\right)\subseteq\Po^n$ una variedad lineal de $\dim m$. Entonces, se define la variedad lineal dual de $V$ como
    \[
        V^* \equiv \pi \left( \F^\perp\right) \subseteq \left( \Po^n \right)^*.
    \]
    
\end{defi}

\begin{obs}
    \begin{enumerate}[i)]
        \item []
        \item $\dim V^* = n-m-1$,
        \item $\left(V^*\right)^*=V$,
        \item $V_1 \subseteq V_2 \iff V_2^* \subseteq V_1^*$,
        \item $\left(V_1 \vee V_2 \right)^* = V_1^* \cap V_2^*$,
        \item $\left(V_1 \cap V_2 \right) ^*=V_1^* \vee V_2^*$.
    \end{enumerate}
    \vspace{0,3cm}
    \noindent En conclusión,
    \[
        \begin{array}{ccc}
             \boxed{\Po^n = \left(\Po^n\right)^{**}} & & \boxed{\left( \Po^n \right)^*}\\
             \left\{\text{vars. lins. de }\dim = m \right\} & \substack{* \\ \longrightarrow \\  \longleftarrow \\ *} & \left\{\text{vars. lins. de }\dim = n-m-1 \right\}
        \end{array}
    \]
    de manera que la operación $*$ define una biyección entre $\Po^n$ y $\left( \Po^n \right)^*$ que invierte el orden de las inclusiones e intercambia $\vee$ y $\cap$.
\end{obs}
\begin{obs}[Caso particular de los hiperplanos]
    \[
        \begin{array}{ccc}
             \boxed{\Po^n} & & \boxed{\left( \Po^n \right)^*}\\
             H \text{ hiperplano} & \longmapsto & H^* \text{ punto}\\
             \dim m=n-1 & & \dim n-(n-1)-1=0
        \end{array}
    \]
    Recíprocamente,
    \[
        \begin{array}{ccc}
            H_p \equiv p^* \text{ hiperplano} & \longmapsfrom & p\in \left(\Po^n\right) ^*
        \end{array}
    \]
    En ecuaciones,
    \[
        H_p \colon a_0x_0+\cdots+a_nx_n=0 \longleftrightarrow p=\begin{pmatrix} a_0 \\ \vdots \\ a_n \end{pmatrix}.
    \]
    \underline{Conclusión}: los hiperplanos de $\Po^n$ son (\textit{están en biyección con}) un espacio proyectivo de $\dim n$.
\end{obs}

\begin{defi}
    Sean $H_0, \dots, H_m$ hiperplanos de $\Po^n$.
    \begin{enumerate}[(i)]
        \item[]
        \item $H_0, \dots, H_m$ son l.i. $\iff H_0^*, \dots, H_m^*$ son l.i.
        \item $H_0, \dots, H_m$ están en posición general $\iff H_0^*, \dots, H_m^*$ están en posición general.
    \end{enumerate}
\end{defi}
\begin{prop}
    \[
        H_0, \dots, H_m \subseteq \Po^n \text{ son l.i. } \iff \dim \left( H_0 \cap \cdots \cap H_m \right) = n-m-1.
    \]
\end{prop}
\begin{proof}
    \underline{Ejercicio}
\end{proof}

\begin{prop}
    Sean $H_0,\dots,H_m \subseteq \Po^n$ hiperplanos, entonces
    \[
        H_0,\dots, H_m \text{ l.i. } \iff dim(H_0 \cap \cdots \cap H_m) = n - m - 1
    \]
\end{prop}
\begin{proof}
    $H_0,\dots,H_m$ l.i $\stackrel{def}{\iff} H_0^*,\dots,H_m^* \in \left(\Po^n\right)^* \text{ son l.i. }
    \iff dim(H_o^* \vee \cdots \vee H_m^*) = m \iff dim(H_o^* \vee \cdots \vee H_m^*)^* = n - m-1
    = dim(H_0 \cap \cdots \cap H_m)$
\end{proof}

\begin{defi}
    Una recta (o haz) de hiperplanos en $\Po^n$ es la colección de hiperplanos que contienen a una variedad
    fijada $V$ de dimensión $n-2$.
\end{defi}


\begin{example}[Dualizacion del triángulo]
\[
\begin{aligned}
\text{Triángulo en $\Po^2$} & \quad \text{Triángulo en $\left( \Po^2 \right)^*$} \\
\begin{tikzpicture}
\begin{axis}[axis equal, hide axis,
		ticks=none,
		width=0.5*\linewidth,
		enlargelimits=0,
		axis lines=center,
		view={10}{20},
		ymin=-2,ymax=3,
		xmin=-2,xmax=3]
		
		\addplot [mark=*,color=red] coordinates {(1,0)} node[label=0:$q_1$]{};
		\addplot [mark=*,color=red] coordinates {(1,1)} node[label=50:$q_2$]{};
		\addplot [mark=*,color=red] coordinates {(0,1)} node[label=$q_3$]{};
		
		\draw (axis cs:-2,3) [color=blue] -- (axis cs:3,-2) node[above] {$r_1$};
		\draw (axis cs:-2,1) [color=blue] -- (axis cs:3,1) node[above] {$r_2$};
		\draw (axis cs:1,-2) [color=blue] -- (axis cs:1,3) node[label=310:$r_3$] {};

\end{axis}
\end{tikzpicture}
& \quad
\begin{tikzpicture}
\begin{axis}[axis equal, hide axis,
		ticks=none,
		width=0.5*\linewidth,
		enlargelimits=0,
		axis lines=center,
		view={10}{20},
		ymin=0,ymax=4,
		xmin=0,xmax=5]
		
		\addplot [mark=*,color=blue] coordinates {(1,1)} node[label=$(r_1)^*$]{};
		\addplot [mark=*,color=blue] coordinates {(4,3)} node[label=160:$(r_2)^*$]{};
		\addplot [mark=*,color=blue] coordinates {(3,1)} node[label=190:$(r_3)^*$]{};
		
		\draw (axis cs:0,1) [color=red] -- (axis cs:5,1) node[label=100:$(q_1)^*$] {};
		\draw (axis cs:2.5,0) [color=red] -- (axis cs:4.5,4) node[label=20:$(q_2)^*$] {};
		\draw (axis cs:0,0.333) [color=red] -- (axis cs:5,3.666) node[label=230:$(q_3)^*$] {};

\end{axis}
\end{tikzpicture}
\end{aligned}
\]
\end{example}

\begin{example}
\[
\begin{aligned}
\text{Cuadrilátero en $\Po^2$} & \quad \text{Cuadrilátero en $\left( \Po^2 \right)^*$} \\
\definecolor{qqccqq}{rgb}{0.,0.8,0.}
\definecolor{qqqqcc}{rgb}{0.,0.,0.8}
\definecolor{ffqqtt}{rgb}{1.,0.,0.2}
\begin{tikzpicture}[line cap=round,line join=round,>=triangle 45,x=.8cm,y=.8cm]
\clip(2.,0.5) rectangle (8.5,7.);
\draw [line width=1.pt,color=qqqqcc,domain=2.:8.5] plot(\x,{(--4.-2.*\x)/-1.});
\draw [line width=1.pt,color=qqqqcc,domain=2.:8.5] plot(\x,{(--0.9088--0.7*\x)/1.24});
\draw [line width=1.pt,color=qqqqcc,domain=2.:8.5] plot(\x,{(--5.36-0.58*\x)/0.76});
\draw [line width=1.pt,color=qqqqcc,domain=2.:8.5] plot(\x,{(--15.4-1.88*\x)/1.});
\draw [line width=1.pt,dash pattern=on 2pt off 2pt,color=qqccqq,domain=2.:8.5] plot(\x,{(--11.46-2.58*\x)/-0.24});
\draw [line width=1.pt,dash pattern=on 2pt off 2pt,color=qqccqq,domain=2.:8.5] plot(\x,{(--14.942080928528313-1.2454296879136713*\x)/4.177002403922524});
\draw [line width=1.pt,dash pattern=on 2pt off 2pt,color=qqccqq,domain=2.:8.5] plot(\x,{(--7.52--0.12*\x)/2.});
\begin{scriptsize}
\draw [fill=ffqqtt] (4.,4.) circle (2.5pt);
\draw[color=ffqqtt] (3.8004958677685994,4.598347107438018) node {$q_4$};
\draw [fill=ffqqtt] (5.,6.) circle (2.5pt);
\draw[color=ffqqtt] (5.255041322314053,6.135537190082645) node {$q_1$};
\draw [fill=ffqqtt] (6.,4.12) circle (2.5pt);
\draw[color=ffqqtt] (6.1806611570247965,4.647933884297522) node {$q_2$};
\draw [fill=ffqqtt] (4.76,3.42) circle (2.5pt);
\draw[color=ffqqtt] (5.13933884297521,3.5404958677685974) node {$q_3$};
\draw[color=qqqqcc] (5.957520661157028,8.606611570247933) node {$f$};
\draw[color=qqqqcc] (-5.216033057851232,-1.8396694214875977) node {$g$};
\draw[color=qqqqcc] (-2.1747107438016466,9.0198347107438) node {$h$};
\draw[color=qqqqcc] (3.420330578512401,8.598347107438016) node {$r_1$};
\draw [fill=ffqqtt] (3.2970786516853927,2.594157303370786) circle (2.5pt);
\draw[color=ffqqtt] (3.1062809917355416,3.011570247933887) node {$q_5$};
\draw [fill=ffqqtt] (7.4740810556079165,1.3487276154571146) circle (2.5pt);
\draw[color=ffqqtt] (7.635206611570251,1.6892561983471108) node {$q_6$};
\draw[color=qqccqq] (5.131074380165293,9.730578512396692) node {$j$};
\draw[color=qqccqq] (-6.34,5.350413223140498) node {$k$};
\draw[color=qqccqq] (-6.34,3.25123966942149) node {$l$};
\end{scriptsize}
\end{tikzpicture}
& \quad
\definecolor{qqccqq}{rgb}{0.,0.8,0.}
\definecolor{ffqqtt}{rgb}{1.,0.,0.2}
\definecolor{ududff}{rgb}{0.30196078431372547,0.30196078431372547,1.}
\begin{tikzpicture}[line cap=round,line join=round,>=triangle 45,x=.8cm,y=.8cm]
\clip(5.5,2.5) rectangle (12.,7.5);
\draw [line width=1.pt,color=ffqqtt,domain=5.5:12.] plot(\x,{(--18.876--0.12*\x)/2.92});
\draw [line width=1.pt,color=ffqqtt,domain=5.5:12.] plot(\x,{(--11.1072-1.32*\x)/-0.14});
\draw [line width=1.pt,color=ffqqtt,domain=5.5:12.] plot(\x,{(-0.564--0.86*\x)/1.3});
\draw [line width=1.pt,color=ffqqtt,domain=5.5:12.] plot(\x,{(-22.7588--2.06*\x)/-1.48});
\draw [line width=1.pt,color=ffqqtt,domain=5.5:12.] plot(\x,{(--26.1456-1.2*\x)/2.78});
\draw [line width=1.pt,color=ffqqtt,domain=5.5:12.] plot(\x,{(--10.0756-2.18*\x)/-1.44});
\begin{scriptsize}
\draw [fill=ududff] (6.22,6.72) circle (2.5pt);
\draw[color=ududff] (6.21,7.36) node {$r_1$};
\draw [fill=ududff] (9.14,6.84) circle (2.5pt);
\draw[color=ududff] (9.43,6.82) node {$r_4$};
\draw [fill=ududff] (9.,5.52) circle (2.5pt);
\draw[color=ududff] (9.33,5.36) node {$r_3$};
\draw [fill=ududff] (7.7,4.66) circle (2.5pt);
\draw[color=ududff] (7.55,5.32) node {$r_2$};
\draw[color=ffqqtt] (-2.42,6.21) node {$f$};
\draw[color=ffqqtt] (9.3,10.83) node {$g$};
\draw[color=ffqqtt] (-2.42,-1.67) node {$h$};
\draw[color=ffqqtt] (3.52,10.83) node {$i$};
\draw[color=ffqqtt] (-2.42,10.31) node {$j$};
\draw[color=ffqqtt] (11.48,10.83) node {$k$};
\draw [fill=qqccqq] (8.753295272078507,3.193926851025867) circle (2.0pt);
\draw [fill=qqccqq] (11.11824048913044,6.921297554347826) circle (2.0pt);
\draw [fill=qqccqq] (8.4304647937959,5.765842535052128) circle (2.0pt);
\end{scriptsize}
\end{tikzpicture}
\end{aligned}
\]
\end{example}

\begin{defi}
Sea $\mathcal{R} = \setb{p_0,\dots,p_n;\bar{p}}$ una referencia de $\Po^n$, sea
$B = \setb{u_0,\dots,u_n}$ una base adaptada y sea $B^* = \setb{u_0^*,\dots,u_n^*}$
su base dual. Entonces, llamamos referencia dual de $\mathcal{R}$ a
\[
  \mathcal{R}^* = \setb{{p_0^\prime}=[u_0^*],\dots,{p_n^\prime}=[u_n^*];\bar{p}^\prime = [u_0^*+ \cdots + u_n^*]}
\]
\end{defi}

\begin{obs}
$\mathcal{R}^*$ no depende de la base $B$ escogida.
\end{obs}
\begin{example}
  En coordenadas, sea $p \in \Po^n$ y sea $p_\mathcal{R} = (a_0,\dots,a_n)$, entonces $p^*$
  es un hiperplano de $\left( \Po^n \right)^*$ y, en referencia $\mathcal{R}^*$ se tiene
  \[
    p^* \equiv a_0x_0^\prime + \cdots + a_nx_n^\prime = 0
  \]
  Y si $H \subseteq \Po^n$ en referencia $\mathcal{R}$
  \[
    \Po^n \supseteq H \equiv b_0x_0 + \cdots + b_nx_n = 0
  \]
  entonces $\left(H^*\right)_{\mathcal{R}^*} = (b_0,\dots,b_n)$
\end{example}

\begin{obs}
En general, si tenemos $\mathcal{R}$ una referencia de $\Po^n$ y 
$\mathcal{R}^* = \setb{{p_0}^\prime,\dots,{p_n}^\prime; \bar{p}^\prime}$
una refencia de $\left( \Po^n \right)^*$. Entonces $\left( {p_i}^\prime \right)^*$
es un hiperplano tal que $\pi \notin \left( {p_i}^\prime \right)^*$ y
$p_j \in \left( {p_i}^\prime \right)^*$ $\forall i \neq j$
\end{obs}

%%%%%%%%%%%%%%%%%%%%
%   23 d'octubre   %
%%%%%%%%%%%%%%%%%%%% 

\subsection{Los teoremas de Desargues y Pappus}

\begin{thm}[de Desargues]

Sean $ABC$, $A'B'C'$ dos triángulos en $\Po^2$ (sin vértices ni lados en común). Entonces,
\[
AA' \cap BB' \cap CC' \neq \emptyset \iff
\begin{cases}
AB \cap A'B' = Z \\
AC \cap A'C' = Y \\
BC \cap B'C' = X
\end{cases}
\text{ están alineados.}
\]

\begin{center}
\definecolor{wwqqzz}{rgb}{0.4,0.,0.6}
\definecolor{dbwrru}{rgb}{0.8588235294117647,0.3803921568627451,0.0784313725490196}
\definecolor{wrwrwr}{rgb}{0.3803921568627451,0.3803921568627451,0.3803921568627451}
\definecolor{sexdts}{rgb}{0.1803921568627451,0.49019607843137253,0.19607843137254902}
\definecolor{dtsfsf}{rgb}{0.8274509803921568,0.1843137254901961,0.1843137254901961}
\definecolor{rvwvcq}{rgb}{0.08235294117647059,0.396078431372549,0.7529411764705882}
\begin{tikzpicture}[line cap=round,line join=round,>=triangle 45,x=1.5cm,y=1.5cm]
\clip(-0.5,-0.5) rectangle (9.,3.5);
\fill[line width=1.pt,color=rvwvcq,fill=rvwvcq,fill opacity=0.10000000149011612] (3.3336552362849274,1.992339795349051) -- (1.9793647349081451,0.1168150700088304) -- (-0.19217003454083312,0.6438220176250907) -- cycle;
\fill[line width=1.pt,color=rvwvcq,fill=rvwvcq,fill opacity=0.10000000149011612] (4.403859447556563,2.040558909290155) -- (4.738313888759495,1.0095694436469795) -- (5.374944253774689,1.657128809689663) -- cycle;
\draw [line width=1.pt,color=rvwvcq] (3.3336552362849274,1.992339795349051)-- (1.9793647349081451,0.1168150700088304);
\draw [line width=1.pt,color=rvwvcq] (1.9793647349081451,0.1168150700088304)-- (-0.19217003454083312,0.6438220176250907);
\draw [line width=1.pt,color=rvwvcq] (-0.19217003454083312,0.6438220176250907)-- (3.3336552362849274,1.992339795349051);
\draw [line width=1.pt,color=rvwvcq] (4.403859447556563,2.040558909290155)-- (4.738313888759495,1.0095694436469795);
\draw [line width=1.pt,color=rvwvcq] (4.738313888759495,1.0095694436469795)-- (5.374944253774689,1.657128809689663);
\draw [line width=1.pt,color=rvwvcq] (5.374944253774689,1.657128809689663)-- (4.403859447556563,2.040558909290155);
\draw [line width=1.pt,dotted,color=sexdts,domain=-0.5:9.] plot(\x,{(--3.6701280196078825-0.383430099600492*\x)/0.9710848062181263});
\draw [line width=1.pt,dotted,color=sexdts,domain=-0.5:9.] plot(\x,{(--3.5541459610560944-1.8755247253402205*\x)/-1.3542905013767823});
\draw [line width=1.pt,dotted,color=sexdts,domain=-0.5:9.] plot(\x,{(--5.2228066883522954-1.0309894656431755*\x)/0.334454441202932});
\draw [line width=1.pt,dotted,color=sexdts,domain=-0.5:9.] plot(\x,{(--2.529148647580714--1.34851777772396*\x)/3.5258252708257607});
\draw [line width=1.pt,dotted,color=sexdts,domain=-0.5:9.] plot(\x,{(-1.2968069532830016--0.5270069476162603*\x)/-2.171534769448978});
\draw [line width=1.pt,dotted,color=sexdts,domain=-0.5:9.] plot(\x,{(-2.425616974499177--0.6475593660426835*\x)/0.6366303650151943});
\draw [line width=1.pt,color=wrwrwr,domain=-0.5:9.] plot(\x,{(--9.50601349361014--0.23250306512482055*\x)/5.1603138069701515});
\draw [line width=1.pt,color=wrwrwr,domain=-0.5:9.] plot(\x,{(--1.9667522247370517-1.215273416826892*\x)/-3.7556551544955843});
\draw [line width=1.pt,color=wrwrwr,domain=-0.5:9.] plot(\x,{(--2.1171944617344565--0.5677140507842084*\x)/3.11902478948039});
\draw [line width=1.pt,color=dbwrru,domain=-0.5:9.] plot(\x,{(--2.855613634592153-0.8059659723881762*\x)/-0.14358796715525868});
\begin{scriptsize}
\draw [fill=wwqqzz] (8.49396904325508,2.2248428604738715) circle (2.5pt);
\draw[color=wwqqzz] (8.5756934700623,2.511596640794484) node {$O$};
\draw [fill=rvwvcq] (3.3336552362849274,1.992339795349051) circle (2.5pt);
\draw[color=rvwvcq] (3.1818813007858053,2.3100939843529726) node {$A$};
\draw [fill=rvwvcq] (-0.19217003454083312,0.6438220176250907) circle (2.5pt);
\draw[color=rvwvcq] (-0.1104456077336135,0.9305757979457029) node {$B$};
\draw [fill=rvwvcq] (1.9793647349081451,0.1168150700088304) circle (2.5pt);
\draw[color=rvwvcq] (1.8742904718702913,0.4500694633544068) node {$C$};
\draw [fill=rvwvcq] (4.403859447556563,2.040558909290155) circle (2.5pt);
\draw[color=rvwvcq] (4.52449688404727,2.3255941886946276) node {$A'$};
\draw [fill=rvwvcq] (4.738313888759495,1.0095694436469795) circle (2.5pt);
\draw[color=rvwvcq] (4.512821965931953,1.2250796804371424) node {$C'$};
\draw [fill=rvwvcq] (5.374944253774689,1.657128809689663) circle (2.5pt);
\draw[color=rvwvcq] (5.365090988350101,1.4265823368786539) node {$B'$};
\draw [fill=dtsfsf] (4.082899471593713,3.029950328105273) circle (2.0pt);
\draw[color=dtsfsf] (3.8006519608976106,3.1316048144606725) node {$Y$};
\draw [fill=dtsfsf] (3.939311504438454,2.223984355717097) circle (2.0pt);
\draw[color=dtsfsf] (3.8240017971282447,2.511596640794484) node {$Z$};
\draw [fill=dtsfsf] (3.498235188833963,-0.2517976241139916) circle (2.0pt);
\draw[color=dtsfsf] (3.3803549087461957,0.07806455915469361) node {$X$};
\draw[color=dbwrru] (3.6,1) node {$r^*$};
\end{scriptsize}
\end{tikzpicture}
\end{center}

\end{thm}
\begin{proof}
$\implies$ Analítica \\
$R = \{A,B,C;O\}$, es un sistema de referencia porque todos los puntos son linealmente independientes.
\begin{itemize}
    \item $A,B,C$ linealmente independientes ($ABC$ triangulo)
    \item $A,B,O$ linealmente independientes (si no, $AB = A'B'$)
    \item $B,C,O$ linealmente independientes (si no, $BC = B'C'$)
    \item $C,A,O$ linealmente independientes (si no, $AC = A'C'$)
\end{itemize}
Tenemos que 
\begin{itemize}
    \item $A' \in AO : x_1 - x_2 = 0$ i $A \neq A' \implies A' = (a:1:1)$
    \item $B' \in BO : x_0 - x_2 = 0$ i $B \neq B' \implies B' = (1:b:1)$
    \item $C' \in CO : x_0 - x_1 = 0$ i $C \neq C' \implies C' = (1:1:c)$
\end{itemize}

\begin{gather*}
Z : \left. \begin{cases}
AB : x_2 = 0 \\
A'B' : 0 = \begin{vmatrix} a & 1 & x_0 \\ 1 & b & x_1 \\ 1 & 1 & x_2 \end{vmatrix}
\end{cases} \right\}
x_0(1-b)-x_1(a-1)=x_0(1-b) + x_1(1-a) = 0 \implies \\ \implies Z = (a-1:1-b:0).
\end{gather*}
Análogamente, $X = (0:b-1:1-c), Y = (a-1:0:1-c)$. Vemos que $Z-X = Z \implies X,Y,Z$ no son linealmente independientes $\iff X,Y,Z$
están alineados. \\ \\
$\implies$ Sintética \\
$\overbrace{\Po^2 = \Po(\E)}^{\pi \text{ plano en } \Po^3} \subset \Po(\bar{\E}) = \Po^3, \bar{\E} = \E \times \k$ \\
Sea $r$ una recta en $\Po^3$, $r \nsubseteq \pi$, tal que $r \cap \pi = O$. Sean $p,p' \in r, p\neq p', p, p' \neq O$.
$r$ y $AA'$ son coplanarias $\implies pA \cap p'A' = A_0$ (intersección de rectas en un plano). Análogamente,
$\begin{cases} pB \cap p'B' = B_0 \\ pC \cap p'C' = C_0 \end{cases}$. \\
Por construcción,
\begin{align*}
    pA_0 \cap \pi &= A, & pB_0 \cap \pi &= B, & pC_0 \cap \pi = C \\
    p'A_0 \cap \pi &= A', & p'B_0 \cap \pi &= B', & p'C_0 \cap \pi = C'
\end{align*}
$A_0, B_0, C_0$ son linealmente independientes (forman un triángulo). Si estuvieran alineados, $T = A_0 \vee B_0 \vee C_0 \implies A,B,C \in
\pi \cap (p \vee t)$ estarían alineados.\\
Finalmente,
\begin{gather*}
    Z = AB \cap A'B'= \pi \cap \left( \left( A_0 \vee B_0 \vee p \right) \cap \left( A_0 \vee B_0 \vee p' \right) \right)
    \substack{\text{intersección de} \\ = \\ \text{planos distintos}} \pi \cap \left( A_0 \vee B_0 \right)
\end{gather*}
Análogamente, $Y = \pi \cap \left( A_0 \vee C_0 \right)$ y $X = \pi \cap \left( B_0 \vee C_0 \right)$. Entonces, $X,Y,Z \in \pi\cap\left(
A_0 \vee B_0 \vee C_0 \right)$ (que es una recta porque es la intersección de dos planos distintos).
\\ \\
$\impliedby$ Dualidad \\
Hipotésis: $X, Y, Z$ están alineados.

Consideramos el espacio proyectivo dual y dualizamos las hipótesis:

\begin{center}
\definecolor{wwqqzz}{rgb}{0.4,0.,0.6}
\definecolor{dtsfsf}{rgb}{0.8274509803921568,0.1843137254901961,0.1843137254901961}
\definecolor{rvwvcq}{rgb}{0.08235294117647059,0.396078431372549,0.7529411764705882}
\definecolor{wrwrwr}{rgb}{0.3803921568627451,0.3803921568627451,0.3803921568627451}
\definecolor{sexdts}{rgb}{0.1803921568627451,0.49019607843137253,0.19607843137254902}
\definecolor{dbwrru}{rgb}{0.8588235294117647,0.3803921568627451,0.0784313725490196}
\begin{tikzpicture}[line cap=round,line join=round,>=triangle 45,x=1.4cm,y=1.4cm]
\clip(-0.8,0.) rectangle (10.,4.5);
\fill[line width=1.pt,color=sexdts,fill=sexdts,fill opacity=0.10000000149011612] (3.5905034348218994,2.7363496037484754) -- (2.236212933445119,0.3958187481586135) -- (-0.36,0.52) -- cycle;
\fill[line width=1.pt,color=sexdts,fill=sexdts,fill opacity=0.10000000149011612] (4.564320062604457,2.6690117085413894) -- (4.909911174012507,1.1462661459359629) -- (5.7636405841417275,1.6921417570208015) -- cycle;
\draw [line width=1.pt,color=sexdts] (3.5905034348218994,2.7363496037484754)-- (2.236212933445119,0.3958187481586135);
\draw [line width=1.pt,color=sexdts] (2.236212933445119,0.3958187481586135)-- (-0.36,0.52);
\draw [line width=1.pt,color=sexdts] (-0.36,0.52)-- (3.5905034348218994,2.7363496037484754);
\draw [line width=1.pt,color=sexdts] (4.564320062604457,2.6690117085413894)-- (4.909911174012507,1.1462661459359629);
\draw [line width=1.pt,color=sexdts] (4.909911174012507,1.1462661459359629)-- (5.7636405841417275,1.6921417570208015);
\draw [line width=1.pt,color=sexdts] (5.7636405841417275,1.6921417570208015)-- (4.564320062604457,2.6690117085413894);
\draw [line width=1.pt,dotted,color=rvwvcq,domain=-0.8:10.] plot(\x,{(--7.659747632557802-0.9768699515205879*\x)/1.1993205215372704});
\draw [line width=1.pt,dotted,color=rvwvcq,domain=-0.8:10.] plot(\x,{(--4.697871799499361-2.340530855589862*\x)/-1.3542905013767803});
\draw [line width=1.pt,dotted,color=rvwvcq,domain=-0.8:10.] plot(\x,{(--7.872684844357776-1.5227455626054265*\x)/0.34559111140804966});
\draw [line width=1.pt,dotted,color=rvwvcq,domain=-0.8:10.] plot(\x,{(--2.852147643456839--2.2163496037484753*\x)/3.9505034348218993});
\draw [line width=1.pt,dotted,color=rvwvcq,domain=-0.8:10.] plot(\x,{(-1.305325474728563--0.12418125184138651*\x)/-2.596212933445119});
\draw [line width=1.pt,dotted,color=rvwvcq,domain=-0.8:10.] plot(\x,{(-1.70159964186535--0.5458756110848386*\x)/0.8537294101292208});
\draw [line width=1.pt,color=dtsfsf,domain=-0.8:10.] plot(\x,{(--16.72573405137783-0.38750510854135545*\x)/5.603960695352214});
\draw [line width=1.pt,color=dtsfsf,domain=-0.8:10.] plot(\x,{(--0.9933148705940678-1.202578349271157*\x)/-4.2845529561616065});
\draw [line width=1.pt,color=dtsfsf,domain=-0.8:10.] plot(\x,{(--2.0204412296839145--0.6567027381863184*\x)/3.4308235460323857});
\draw [line width=1.pt,color=wwqqzz,domain=-0.8:10.] plot(\x,{(--3.193311356394389-0.8938251320959107*\x)/-0.1608060938655136});
\begin{scriptsize}
\draw [fill=dbwrru] (9.194464130174113,2.34884449520712) circle (2.5pt);
\draw[color=dbwrru] (9.276188556981314,2.63559827552772) node {$r^*$};
\draw [fill=sexdts] (3.5905034348218994,2.7363496037484754) circle (2.5pt);
\draw[color=sexdts] (3.4270545812074604,3.069603997094052) node {$(AB)^*$};
\draw [fill=sexdts] (-0.36,0.52) circle (2.5pt);
\draw[color=sexdts] (-0.23886970700210086,0.8065741632124634) node {$(AC)^*$};
\draw [fill=sexdts] (2.236212933445119,0.3958187481586135) circle (2.5pt);
\draw[color=sexdts] (2.3529621145982897,0.6825725284792257) node {$(BC)^*$};
\draw [fill=sexdts] (4.564320062604457,2.6690117085413894) circle (2.5pt);
\draw[color=sexdts] (4.74632032823829,2.961102566702469) node {$(A'B')^*$};
\draw [fill=sexdts] (4.909911174012507,1.1462661459359629) circle (2.5pt);
\draw[color=sexdts] (4.594546392739168,1.3645815195120332) node {$(B'C')^*$};
\draw [fill=sexdts] (5.7636405841417275,1.6921417570208015) circle (2.5pt);
\draw[color=sexdts] (5.948836894115948,1.9845896931782219) node {$(A'C')^*$};
\draw[color=sexdts] (2.7966090029803388,1.860588058444984) node {$(B)^*$};
\draw[color=sexdts] (0.9753217769908749,0.8375745718957729) node {$(C)^*$};
\draw[color=sexdts] (1.7225165363711676,1.535083767270235) node {$(A)^*$};
\draw[color=sexdts] (4.617896228969802,1.969089488836567) node {$(B')^*$};
\draw[color=sexdts] (5.5051900057339,1.3955819281953425) node {$(C')^*$};
\draw[color=sexdts] (5.330066234004144,2.3410943930362804) node {$(A')^*$};
\draw [fill=wrwrwr] (4.278993131711778,3.926220888888763) circle (2.0pt);
\draw[color=wrwrwr] (4.489472129701314,3.9531156445683706) node {$(BB')^*$};
\draw [fill=wrwrwr] (4.118187037846265,3.032395756792852) circle (2.0pt);
\draw[color=wrwrwr] (4.302673439856241,3.3796080839271463) node {$(AA')^*$};
\draw [fill=wrwrwr] (3.631836574987902,0.32906371097595916) circle (2.0pt);
\draw[color=wrwrwr] (3.415379663092143,0.6360719154542616) node {$(CC')^*$};
\draw[color=dtsfsf] (7.478251167222485,2.759599910260958) node {$Z^*$};
\draw[color=dtsfsf] (7.209728050570193,2.2325929626446976) node {$Y^*$};
\draw[color=dtsfsf] (7.443226412876534,1.597084584636854) node {$X^*$};
\draw[color=wwqqzz] (4.115874750011168,2.186092349619733) node {$O^*$};
\end{scriptsize}
\end{tikzpicture}
\end{center}

Queremos ver que, en efecto, $\left(AA'\right)^*, \left(BB'\right)^*, \left(CC'\right)^*$ están alineados. Por la primera implicación,
$\exists s \subset (\Po^2)^*$ recta tal que:

\begin{align*}
    s &= \left[ \left[ \left(AB\right)^* \vee \left(AC\right)^* \right] \cap \left[ \left(A'B'\right)^* \vee \left(A'C'\right)^* \right] \right]
    \vee [\dots ] \vee [\dots] \\
    &= \left[ (AB \cap AC)^* \cap \left(A'B' \cap A'C' \right)^* \right] \vee [\dots] \vee [\dots] \\
    &= \left(A^* \cap (A')^* \right) \vee \left(B^* \cap (B')^* \right) \vee \left(C^* \cap (C')^* \right) \\
    &= (AA')^* \vee (BB')^* \vee (CC')^*
\end{align*}

$s = (AA' \cap BB' \cap CC')^* \implies AA' \cap BB' \cap CC' = s^*$ (un punto)
\end{proof}
\begin{obs}
El teorema de Desargues es su propio dual: $(\implies) \iff (\impliedby)$.
\end{obs}
\begin{obs}
Definición axiomática de $\Po^2$
\begin{itemize}
    \item $X \rightarrow$ puntos
    \item $Y \rightarrow$ rectas
\end{itemize}
Hay una relación de pertenencia entre puntos y rectas (los puntos pertenecen a rectas). Axiomas:
\begin{itemize}
    \item $\forall p_1, p_2, \exists! r$ tal que $p_1, p_2 \in r$
    \item $\forall r_1, r_2, \exists! p$ tal que $p \in r_1, r_2$
    \item $\exists 4$ puntos no alineados
\end{itemize}
\end{obs}

%%%%%%%%%%%%%%%%%
% 25 de OCTUBRE %
%%%%%%%%%%%%%%%%%

\begin{thm}[ de Pappus]
Sean $r,s \subseteq \Po^2_\k$ rectas diferentes ($r \neq s$), sean
$A,B,C \in r$ tal que $A \neq B$, $B \neq C$ y $C\neq A$ y sean también
$A^\prime,B^\prime,C^\prime \in s$ tal que $A^\prime \neq B^\prime$,
$B^\prime \neq C^\prime$ y $C^\prime \neq A^\prime$. Entonces
$x_0 \equiv AB^\prime \cap A^\prime B$, $x_1 \equiv AC^\prime \cap A^\prime C$
y $x_2 \equiv BC^\prime \cap B^\prime C$ estan alineados.
\end{thm}

\begin{proof}
\begin{center}
\definecolor{qqffqq}{rgb}{0.,1.,0.}
\definecolor{qqqqff}{rgb}{0.,0.,1.}
\definecolor{uuuuuu}{rgb}{0.26666666666666666,0.26666666666666666,0.26666666666666666}
\definecolor{ffqqqq}{rgb}{1.,0.,0.}
\definecolor{xdxdff}{rgb}{0.49019607843137253,0.49019607843137253,1.}
\definecolor{ududff}{rgb}{0.30196078431372547,0.30196078431372547,1.}
\begin{tikzpicture}[line cap=round,line join=round,>=triangle 45,x=1.0cm,y=1.0cm]
\clip(4.,8.) rectangle (19.,17.);
\draw [line width=2.pt,domain=4.:19.] plot(\x,{(-128.24--2.8*\x)/-6.96});
\draw [line width=2.pt,domain=4.:19.] plot(\x,{(-56.4-1.2*\x)/-6.96});
\draw [line width=2.pt,dash pattern=on 3pt off 3pt,color=ffqqqq,domain=4.:19.] plot(\x,{(--7.919145496535805-4.513972286374134*\x)/-2.981039260969977});
\draw [line width=2.pt,dash pattern=on 3pt off 3pt,color=ffqqqq,domain=4.:19.] plot(\x,{(--72.52791676142823-4.795599272231065*\x)/1.977632476688651});
\draw [line width=2.pt,dash pattern=on 3pt off 3pt,color=qqqqff,domain=4.:19.] plot(\x,{(-28.55080831408779-5.033487297921477*\x)/-5.994226327944574});
\draw [line width=2.pt,dash pattern=on 3pt off 3pt,color=qqqqff,domain=4.:19.] plot(\x,{(--108.66232658630884-5.803331817148056*\x)/4.482567659768023});
\draw [line width=2.pt,dash pattern=on 3pt off 3pt,color=qqffqq,domain=4.:19.] plot(\x,{(-6.835751701375273-5.829086570152542*\x)/-4.016593851255923});
\draw [line width=2.pt,dash pattern=on 3pt off 3pt,color=qqffqq,domain=4.:19.] plot(\x,{(--64.90175358645223-6.31730410352219*\x)/1.5015283987980457});
\draw [line width=2.pt,domain=4.:19.] plot(\x,{(--31.871782879699552--0.11126215677628437*\x)/2.646579974541557});
\begin{scriptsize}
\draw [fill=ududff] (17.96,11.2) circle (2.5pt);
\draw[color=ududff] (18.1,11.57) node {$O$};
\draw [fill=ududff] (11.,14.) circle (2.5pt);
\draw[color=ududff] (10.86,14.59) node {$A$};
\draw[color=black] (0.52,18.07) node {$f$};
\draw [fill=ududff] (11.,10.) circle (2.5pt);
\draw[color=ududff] (11.2,10.37) node {$A'$};
\draw[color=black] (0.52,8.53) node {$g$};
\draw [fill=xdxdff] (9.022367523311349,14.795599272231065) circle (2.5pt);
\draw[color=xdxdff] (9.34,15.01) node {$B$};
\draw [fill=xdxdff] (6.517432340231977,15.803331817148056) circle (2.5pt);
\draw[color=xdxdff] (6.66,16.17) node {$C$};
\draw [fill=xdxdff] (8.018960739030023,9.486027713625866) circle (2.5pt);
\draw[color=xdxdff] (8.38,9.25) node {$B'$};
\draw [fill=xdxdff] (5.005773672055426,8.966512702078523) circle (2.5pt);
\draw[color=xdxdff] (5.3,8.77) node {$C'$};
\draw[color=ffqqqq] (13.96,18.95) node {$h$};
\draw[color=ffqqqq] (7.54,18.95) node {$i$};
\draw [fill=uuuuuu] (9.98455175154192,12.462380112935687) circle (2.0pt);
\draw[color=uuuuuu] (10.29,12.42) node {$x_0$};
\draw[color=qqqqff] (0.52,5.59) node {$j$};
\draw[color=qqqqff] (4.34,18.95) node {$k$};
\draw [fill=uuuuuu] (9.125908190188389,12.426282758819337) circle (2.0pt);
\draw[color=uuuuuu] (9.07,13.02) node {$x_1$};
\draw[color=qqffqq] (11.56,18.95) node {$l$};
\draw[color=qqffqq] (5.98,18.95) node {$m$};
\draw [fill=uuuuuu] (7.337971777000363,12.351117956159403) circle (2.0pt);
\draw[color=uuuuuu] (6.97,12.78) node {$x_2$};
\draw[color=black] (0.52,11.91) node {$n$};
\end{scriptsize}
\end{tikzpicture}
\end{center}

Tomamos la referencia $\mathcal{R} = \setb{A,B,A^\prime;B^\prime}$. Entonces
\[
\begin{aligned}
A = (1,0,0) &\qquad A^\prime = (0,0,1) \\
B = (0,1,0) &\qquad B^\prime = (1,1,1) \\
C = (1,a,0) &\qquad C^\prime = (1,1,b)
\end{aligned}
\]
con $a,b \neq 0$. Calculamos ahora $x_0$, $x_1$ y $x_
2$ y vemos que
\[
  \determinant{\line(1,0){25} & x_0 & \line(1,0){25} \\ 
    \line(1,0){25} & x_1 & \line(1,0){25} \\ \line(1,0){25} & x_2 & \line(1,0){25}} = 0
\]
\end{proof}


\subsection{Coordenadas absolutas. Razón doble}

\begin{defi}
  Sea $\R = \setb{p_0,p_1;\bar{p}}$ un sistema de referencia de $\Po^1_\k$ y
  sea $q \in \Po^1_k$ con $q_\R = (x_0,x_1)$ las coordenadas de $q$ en la 
  referencia $\R$. Definimos las coordenas absolutas de $q$ como
  \[
    \alpha = \frac{x_0}{x_1} \in k \cup \setb{\infty}
  \]
\end{defi}

\begin{obs}
  A diferencia de las coordenadas normales, \textbf{NO} está definido salvo constantes.
\end{obs}
\begin{obs}
  Al trabajar con coordenadas absolutas, trabajaremos con aritmética normal de $0/\infty$.
\end{obs}
\begin{obs}
  \[
    \begin{aligned}
      \left( p_0 \right)_\R = (1,0) &\qquad \alpha_0 = \infty \\
      \left( p_1 \right)_\R = (0,1) &\qquad \alpha_1 = 0 \\
      \left( p_2 \right)_\R = (1,1) &\qquad \alpha_2 = 1 \\
    \end{aligned}
  \]
\end{obs}

\begin{defi}
  Sean $q_1, q_2, q_3, q_4 \in \Po^1_\k$ con al menos 3 de ellos diferentes. Sea
  $\R$ un sistema de referencia y sean $\left( q_i \right)_\R = (x_i:y_i)$
  $\forall i=1,2,3,4$. Definimos la razón doble de $q_1,q_2,q_3,q_4$ como
  \[
    \left( q_1,q_2,q_3,q_4 \right) = \frac{\determinant{x_3 & y_3 \\ x_1 & y_1}}
    {\determinant{x_3 & y_3 \\ x_2 & y_2}}:\frac{\determinant{x_4 & y_4 \\ x_1 & y_1}}
    {\determinant{x_4 & y_4 \\ x_2 & y_2}} \in k \cup \setb{\infty}
  \]
\end{defi}
\begin{obs}
  La razón doble no depende del sistema de referencia ni del representante
  de cada punto. De hecho, sean $\R$ y $\bar{\R}$ dos sistemas de referencia
  y $q_i = (x_i, y_i)_\R = (\bar{x_i}, \bar{y_i})_{\bar{\R}}$, por lo tanto
  \[
    \begin{pmatrix} \bar{x_i} \\ \bar{y_i} \end{pmatrix} = S \begin{pmatrix}
      x_i \\ y_i \end{pmatrix}
  \]
  y se tiene que
  \[
    \frac{\determinant{\bar{x_3} & \bar{y_3} \\ \bar{x_1} & \bar{y_1}}}
    {\determinant{\bar{x_3} & \bar{y_3} \\ \bar{x_2} & \bar{y_2}}}:
    \frac{\determinant{\bar{x_4} & \bar{y_4} \\ \bar{x_1} & \bar{y_1}}}
    {\determinant{\bar{x_4} & \bar{y_4} \\ \bar{x_2} & \bar{y_2}}}
    = 
    \frac{\cancel{\det S}\determinant{x_3 & y_3 \\ x_1 & y_1}}
    {\cancel{\det S} \determinant{x_3 & y_3 \\ x_2 & y_2}}:
    \frac{\cancel{\det S} \determinant{x_4 & y_4 \\ x_1 & y_1}}
    {\cancel{\det S} \determinant{x_4 & y_4 \\ x_2 & y_2}}
  \]
\end{obs}
\begin{obs}
  La razón doble es cociente de razones simples
\end{obs}
\begin{obs}
  En coordenadas absolutas, si 
  \[
    \begin{aligned}
      q_1 &\qquad \alpha_1 &\qquad (\alpha_1:1) \\
      q_2 &\qquad \alpha_2 &\qquad (\alpha_2:1) \\
      q_3 &\qquad \alpha_3 &\qquad (\alpha_3:1) \\
      q_4 &\qquad \alpha_4 &\qquad (\alpha_4:1) \\
    \end{aligned}
  \]
  teniendo en cuenta que $(\infty,1) = (1,0)$. emtonces, 
  \[
    \left( q_1,q_2,q_3,q_4 \right) = \frac{\determinant{\alpha_3 & 1 \\ \alpha_1 & 1}}
    {\determinant{\alpha_3 & 1 \\ \alpha_2 & 1}}:\frac{\determinant{\alpha_4 & 1 \\ \alpha_1 & 1}}
    {\determinant{\alpha_4 & 1 \\ \alpha_2 & 1}} = 
    \frac{\alpha_3-\alpha_1}{\alpha_3-\alpha_2}:\frac{\alpha_4-\alpha_1}{\alpha_4-\alpha_2}
  \]
\end{obs}
\begin{obs}\label{obs:rzdo_ref}
  Sean $q_1,q_2,q_3,q_4 \in \Po^1_\k$ tal que $\setb{q_1,q_2,q_3}$ son diferentes dos a dos.
  Tomamos $\R = \setb{q_1,q_2;q_3}$ como nuestro sistema de referncia y sean $(x,y)_\R = q_4$ 
  las coordenadas de $q_4$ en el sistema de referencia $\R$, entonces
  \[
    \left( q_1,q_2,q_3,q_4 \right) = \frac{\determinant{1 & 1 \\ 1 & 0}}
    {\determinant{1 & 1 \\ 0 & 1}}:\frac{\determinant{x & y \\ 1 & 0}}
    {\determinant{x & y \\ 0 & 1}} = \frac{-1}{1}:\frac{-y}{x} = \frac{x}{y}
  \]
  es decir, la coordenada obsoluta de $q_4$ en la referencia $\R = \setb{q_1,q_2;q_3}$.
\end{obs}
\begin{obs}
 Sea $V \subseteq \Po^n$ una variedad de dimensión $n-2$, por lo tanto $V^*$ es una recta de
 $\left( \Po^n \right)^*$ y si $V \subseteq H_1,H_2,H_3,H_4 \subseteq \Po^n$, la razón doble
 de $H_1,H_2,H_3,H_4$ queda definida considerando los hiperplanos como puntos del espacio dual,
 es decir,
 \[
  (H_1,H_2,H_3,H_4) = (H_1^*,H_2^*,H_3^*,H_4^*)
 \]
\end{obs}

\begin{example}
 Consideramos $V \subseteq \Po^4$, con
 \[
  V : \begin{cases}
       x_0 = 0 \\ x_1 = 0
      \end{cases}
 \]
 Podemos identificar $V^* = \setb{H^* \in \left( \Po^4 \right)^* \text{ hiperplano} \vert V \subseteq H}$.
 Tomamos ahora
 \[\bar{\rho}w_1(u_2) +
      \rho w_2(u_1)
  \begin{aligned}
   H_1 &\rightarrow x_0 = 0 &\leftrightarrow (1,0,0,0) \\
   H_2 &\rightarrow x_1 = 0 &\leftrightarrow (0,1,0,0) \\
   H_3 &\rightarrow 2x_0+x_1 = 0 &\leftrightarrow (2,1,0,0) \\
   H_4 &\rightarrow 3x_0+2x_1 = 0 &\leftrightarrow (3,2,0,0)
  \end{aligned}
 \]
 Para calcular la razón doble, aplicamos la observación \ref{obs:rzdo_ref}, y tomamos el sistema
 de referencia $\R = \setb{H_1,H_2;H_3}$ y $B = \setb{(2,0,0,0),(0,1,0,0)}$ una base adaptada.
 Entonces se tiene que $\left( H_4 \right)_\R = \left( \frac{3}{2}, 2 \right) = (3,4)$. Por lo
 tanto,
 \[
  (H_1,H_2,H_3,H_4) = \frac{3}{4}
 \]
\end{example}

\begin{prop}
 Sean $q_1,q_2,q_3,q_4 \in \Po^n$ y sea $\rho = (q_1,q_2,q_3,q_4)$. Si permutamos los cuatro
 puntos, (con $4! = 24$ posiblidades), obtendremos uno de los sisguientes valores para la
 razón doble
 \[\bar{\rho}w_1(u_2) +
      \rho w_2(u_1)
   A_p = \setb{\rho, \frac{\rho}{1}, 1-\rho, \frac{1}{1-\rho}, \frac{\rho}{\rho-1}, \frac{\rho-1}{\rho}}
 \]

\end{prop}

%%%%%%%%%%%%%%%%%
% 27 de octubre %
%%%%%%%%%%%%%%%%%

\begin{defi}[Proyección y sección]
 Sean $V,r \subseteq \Po^n$ dos variedades suplementarias con $\dim r = 1$ ($\implies \dim V = n-2$).
 Entonces, definimos la proyección desde $V$ como
 \[
  \begin{aligned}
   \Phi \colon r &\to V^* \\ p &\mapsto V \vee p
  \end{aligned}
 \]
 De nuevo identificando los hiperplanos de $\Po^n$ con puntos de $\left( \Po^n \right)^*$. Asimismo,
 definimos la sección con $r$ como
 \[
  \begin{aligned}
   \Psi \colon V^* &\to r \\ H &\mapsto H \cap r
  \end{aligned}
 \]
\end{defi}

\begin{obs}
 $p \notin V \implies V \vee p$ es un hiperplano que contiene a $V$.
\end{obs}
\begin{obs}
 $\Psi$ está bien definido y $r \cap H$ es un punto. En efecto, si $r \cap H$ no es un punto,
 $r \subseteq H$, pero si nos miramos ahora $H \cong \Po^{n-1}$, entonces $V$ es un hiperplano
 de $H$, $\implies r \cap V \neq \emptyset$, lo cual es una contradicción ya que hemos supuesto que
 $V$ y $r$ son suplementarias.
\end{obs}
\begin{obs}
 $\Phi(\Psi(H)) = H$ y $\Psi(\Phi(p)) = p$, esd decir, $\Phi$ y $\Psi$ son inversas mutuas.
\end{obs}

\begin{thm}[de la invariancia de razón doble]\label{thm:inv_rz_do}
 Sean $V,r \subseteq \Po^n$ dos variades suplementarias con $\dim r = 1$ ($\implies \dim V = n-2$) y
 sean $\Phi$ y $\Psi$ la proyección desde $V$ y la sección con $r$ respectivamente. Entonces,
 \begin{enumerate}[i)]
  \item \label{item:thm_inv_1} $\forall p_1,p_2,p_3,p_4 \in r$ con $p_1,p_2,p_3$ distintos dos a dos
    \[
     (p_1,p_2,p_3,p_4) = \left( \Phi(p_1), \Phi(p_2), \Phi(p_3), \Phi(p_4) \right)
    \]
  \item \label{item:thm_inv_2} $\forall H_1,H_2,H_3,H_4 \in V^*$ (identificando hiperplanos con elementos del dual) con
  $H_1,H_2,H_3$ distintos dos a dos. Entonces
    \[
     (H_1,H_2,H_3,H_4) = \left( \Psi(H_1), \Psi(H_2), \Psi(H_3), \Psi(H_4) \right)
    \]
 \end{enumerate}
\end{thm}

\begin{proof}
 Primero vemos que $\ref{item:thm_inv_1} \iff \ref{item:thm_inv_2}$ porque $\Psi^{-1}=\Phi$. Pasamos ahora
 a demostrar $\ref{item:thm_inv_1}$. Sea $\rho = (p_1,p_2,p_3,p_4)$, entonces podemos elegir $u_1,u_2 \in \E$
 tal que $p_1 = [u_1]$, $p_2=[u_2]$ y $p_3=[u_1+u_2]$. Por lo tanto $p_4 = [\rho u_1+u_2]$, de la misma forma,
 $\bar{\rho} = (H_1,H_2,H_3,H_4)$ y podemos escoger $w_1,w_2 \in \E^*$ tal que $H_1 = [w_1]$, $H_2 = [w_2]$ y
 $H_3=[2_1+w_2]$. Lo que implica que $H_4 = [\bar{\rho} w_1 + w_2]$. Ahora trabajamos con que $p_i \in H_i$.
 Entonces
 \[
  \begin{cases}
    p_1 \in H_1 \implies w_1(u_1) = 0 \\
    p_2 \in H_2 \implies w_2(u_2) = 0 \\
    p_3 \in H_3 \implies (w_1+w_2)(u_1+u_2) = 0 \implies \cancelto{0}{w_1(u_1)} + w_1(u2) + w_2(u_1) + \cancelto{0}{w_2(u_2)} = 0 \\
    p_4 \in H_4 \implies (\bar{\rho}w_1+w_2)(\rho u_1+u_2) = 0 \implies \cancelto{0}{\bar{\rho}w_1(u_1)} + \bar{\rho}w_1(u_2) +
      \rho w_2(u_1) + \cancelto{0}{w_2(u_2)} = 0
  \end{cases}
 \]
 trabajando ahora con la cuarta ecuación
 \[
  \bar{\rho}w_1(u_2) + \rho w_2(u_1) = w_1(u_2)(\bar{\rho} - \rho) = 0 \stackrel{(1)}{\implies} \bar{\rho} = \rho
 \]
 La cancelación en (1) ocurre ya que $w_1(u_2) \neq 0$. Y $w_1(u_2) \neq 0$ ya que, si lo fuera,
 \[
  w_1(u_2) = 0 \implies p_2 \in H_1 \implies \setb{p_1,p_2} \subseteq r \cap H_1 \implies r \subseteq H_1 \implies r \cap V \neq \emptyset
 \]
 pero como sabemos $r \cap V = \emptyset$ ya que son variedades suplementarias.
\end{proof}

\subsection{Cuaternas armónicas}

\begin{obs}
 Sean $p_1,p_2,p_3,p_4 \in \Po^n$ diferentes dos a dos. y sea
 \[
  A_p = \setb{\rho, \frac{1}{\rho}, 1-\rho, \frac{1}{1-\rho}, \frac{\rho}{\rho-1},\frac{\rho-1}{\rho}}
 \]
 Entonces, los valores de $A_p$ son diferentes, salvo el caso en que
 \[
  A_p = \setb{-1,\frac{1}{2},2} \qquad \text{con $\k = \real$}
 \]
\end{obs}

\begin{defi}
 Sean $p_1,p_2,p_3,p_4 \in \Po^n$ diferentes dos a dos. Entonces $\setb{p_1,p_2,p_3,p_4}$ forman
 una cuaterna armónica si y solo si
 \[
  (p_1,p_2,p_3,p_4) = -1
 \]
\end{defi}

\begin{prop}\label{prop:intercambio_raz_do}
 Sean $p_1,p_2,p_3,p_4 \in \Po^n$ diferentes dos a dos. Entonces, son eqivalentes
 \begin{enumerate}[i)]
  \item $\rho = (p_1,p_2,p_3,p_4) = -1$
  \item $(p_1,p_2,p_3,p_4) = (p_2,p_1,p_3,p_4)$
  \item $(p_1,p_2,p_3,p_4) = (p_1,p_2,p_4,p_3)$
  \item $(p_1,p_2,p_3,p_4) = (p_3,p_4,p_1,p_2)$
 \end{enumerate}
\end{prop}

%TODO hacer esta demo
\begin{proof}
 La demostración se hará en clase de problemas 
\end{proof}

\begin{thm}[Cuaterna armónica en el cuadrilátero]
 Sean $A,B,C,D,p_1,p_2 \in \Po^2$ los vértices de un cuadrilátero, sea $p_1p_2 = r \subseteq \Po^2$,
 $AC = s \subseteq \Po^2$, $s \cap BD = O \in \Po^2$, $s \cap r = p_3 \in \Po^2$ y $DB \cap r = p_4 \in \Po^2$.
 Entonces, $\setb{p_1,p_2,p_3,p_4}$ son una cuaterna armónica. Dicho de otra forma, en un cuadrilátero, la intersección
 de una diagonal con las otras dos y los vértices de la diagonal forman una cuaterna armónica.
 \begin{center}
  \begin{tikzpicture}
  \begin{axis}[hide axis,
    ticks=none,
    width=\linewidth,
    enlargelimits=0,
    axis lines=center,
    ymin=0.5,ymax=5,
    xmin=0,xmax=7]

    \addplot [mark=*,color=blue] coordinates {(3.2,4.18)} node[label=right:$A$]{};
    \addplot [mark=*,color=blue] coordinates {(3,2)} node[label=160:$B$]{};
    \addplot [mark=*,color=blue] coordinates {(2.3,1.9)} node[label=190:$C$]{};
    \addplot [mark=*,color=blue] coordinates {(1.54,2.72)} node[label=190:$D$]{};
    \addplot [mark=*,color=blue] coordinates {(0.28,1.61)} node[label=below:$p_1$]{};
    \addplot [mark=*,color=blue] coordinates {(2.93,1.22)} node[label=190:$p_2$]{};
    
    \addplot [color=red] {-1.08*x+4.38};
    \addplot [color=red] {0.88*x+1.37};
    \addplot [color=red] {0.14*x+1.57};
    \addplot [color=red] {10.9*x-30.7};
    
    \addplot [color=orange] {2.53*x-3.93};
    \addplot [color=orange] {-0.49*x+3.48};
    \addplot [color=orange] {-0.15*x+1.65};
    
    \addplot [mark=*,color=green] coordinates {(2.08,1.35)} node[label=190:$p_3$]{};
    \addplot [mark=*,color=green] coordinates {(5.28,0.88)} node[label=190:$p_4$]{};
    \addplot [mark=*,color=green] coordinates {(2.45,2.27)} node[label=190:$O$]{};

  \end{axis}
  \end{tikzpicture}
 \end{center}
\end{thm}

\begin{proof}
 Podemos poner poner coordemadas y operar y nos saldrá el resultado. Aunque aquí detallamos una demostración
 sintética. Para esta demostración nos valdremos de las proyecciones y secciones definidas anteriormente, por
 ello, recordamos que si $h \subseteq \Po^2$ es una recta, una variedad complementaria de $h$ es un punto. Ahora
 definimos $\Phi_B$ la proyección desde $B$, $\Phi_D$ la proyección desde $D$ y $\Psi_s$ y $\Psi_r$ la sección
 con $r$ y con $s$ respectivamente. entonces
 \[
  \begin{array}{ccc}
   r & \stackrel{\Psi_s \circ \Phi_B}{\to} s & \stackrel{\Psi_r \circ \Phi_D}{\to} r \\
   p_1 &\mapsto C &\mapsto p_2 \\
   p_2 &\mapsto A &\mapsto p_1 \\
   p_3 &\mapsto p_3 &\mapsto p_3 \\
   p_4 &\mapsto O &\mapsto p_4
  \end{array}
 \]
 Por el teorema \ref{thm:inv_rz_do}, tenemos que, si $f = \Psi_r \circ \Phi_D \circ \Psi_s \circ \Phi_B$,
 \[
  (p_1,p_2,p_3,p_4) = \left( f(p_1), f(p_2), f(p_3), f(p_4) \right) = (p_2,p_1,p_3,p_4)
 \]
 Y por la proposición \ref{prop:intercambio_raz_do}, tenemos que $\setb{p_1,p_2,p_3,p_4}$ es una cuaterna
 armónica.
\end{proof}

\section{Proyectividades}
\subsection{Definiciones y propiedades básicas}
\begin{defi}
	\begin{enumerate}
		\item[]
		\item Sean $\Po = \Po(\E), \bar{\Po} = \Po(\bar{\E})$ espacios proyectivos de dimensión $n$. Sea $\varphi : \E \rightarrow \bar{\E}$ una aplicación lineal biyectiva. Una proyectividad, definida por $\varphi$, es una aplicación
		\begin{align*}
			f := [\varphi] : \Po &\rightarrow \bar{\Po} \\
			p = [v] &\mapsto f(p) = [\varphi(v)]
		\end{align*}
		Diremos que $\varphi$ es un representante de $f$.
		\item Si $\Po = \bar{\Po}$, llamaremos homografías a las proyectividades.
	\end{enumerate}
\end{defi}
\begin{obs}
	$\varphi$ no inyectiva $\implies [\varphi]$ no es una aplicación de $\Po(\E)$ en $\Po(\bar{\E})$.
\end{obs}
\begin{obs}
	$[\varphi] = [\psi] \iff \exists \lambda \neq 0$ tal que $\psi = \lambda \varphi$.
\end{obs}
\begin{proof}
	$\impliedby$ \\
	$\exists \lambda \neq 0$ tal que $\psi = \lambda\varphi$, sea $p = [v]$, $[\psi](p) = [\psi(v)] = [\lambda\varphi(v)] = [\varphi(v)] = [\varphi](p), \forall p \implies [\psi] = [\varphi].$ \\ \\
	$\implies$ \\
	Sean $u, v \in \E$ linealmente independientes, y sean $p = [u], q = [v]$. Entonces, como $[\varphi(u)] = [\psi(u)]$ y $[\varphi(v)] = [\psi(v)]$,
	\begin{gather*}
		\exists \lambda_1 \neq 0 \text{ tal que } \psi(u) = \lambda_1 \varphi(u) \\
		\exists \lambda_2 \neq 0 \text{ tal que } \psi(v) = \lambda_2 \varphi(v).
	\end{gather*}
	Por otro lado,
	\begin{gather*}
		[\psi(u+v)] = [\varphi(u+v)] \implies \exists \lambda_3 \neq 0 \text{ tal que } \psi(u) + \psi(v) = \psi(u+v) = \lambda_3 \varphi(u+v) = \\ \lambda_3 \varphi(u) + \lambda_3 \varphi(v) \implies (\lambda_1 - \lambda_3) \varphi(u) + (\lambda_2-\lambda_3) \varphi(v) = 0 \implies \lambda_1 = \lambda_2 = \lambda_3 =: \lambda \neq 0. \\
		\psi(u) = \lambda \varphi(u), \forall u \in \E \implies \exists \lambda \neq 0 \text{ tal que } \psi = \lambda \varphi.
	\end{gather*}
\end{proof}
\begin{prop}
	\begin{enumerate}
		\item[]
		\item Operaciones
		\begin{itemize}
			\item $\Id_\Po = [\Id_\E]$ es una proyectividad.
			\item $f, g$ proyectividades $\implies g \circ f$ proyectividad y $[\psi] \circ [\varphi] = [\psi \circ \varphi]$.
			\item $f = [\varphi]$ proyectividad $\implies f^{-1}$ proyectividad y $f^{-1} = [\varphi^{-1}]$.
		\end{itemize}
		\item Relación con variedades lineales \\
		Sea $f = [\varphi] : \Po \rightarrow \bar{\Po}$,
		\begin{itemize}
			\item $V = \pi(F)$ variedad lineal de $\Po$ de dimensión $d \implies f(V) = \pi(\varphi(F))$ es una variedad lineal de $\bar{\Po}$ de dimensión $d$.
			\item $V_1 \subseteq V_2 \iff f(V_1) \subseteq f(V_2).$
			\item $f(V_1 \vee V_2) = f(V_1) \vee f(V_2)$.
			\item $f(V_1 \cap V_2) = f(V_1) \cap f(V_2)$.
			\item En particular, $p_0, \dots, p_d$ linealmente independientes $\iff f(p_0), \dots, f(p_d)$ linealmente independientes.
		\end{itemize}
	\end{enumerate}
\end{prop}
\begin{proof}
	Inmediata a partir de las propiedades de $\varphi$ (biyectiva).
\end{proof}
\begin{col}
	Toda proyectividad es una colineación (mantiene la alineación de puntos).
\end{col}
\begin{col}
	Toda proyectividad mantiene las razones dobles.
\end{col}
\begin{proof}
	$\rho = (p_1, p_2, p_3, p_4) \implies \exists u,v \in \E$ tales que $p_1 = [u], p_2 = [v], p_3 = [u+v], p_4 = [\rho u + v] \implies f(p_1) = [\varphi(u)], f(p_2) = [\varphi(v)], f(p_3) = [\varphi(u) + \varphi(v)], f(p_4) = [\rho \varphi(u) + \varphi(v)] \implies \rho = (f(p_1), f(p_2), f(p_3), f(p_4))$.
\end{proof}
\begin{defi}
	La matriz de una proyectividad $f$ es $M_{\R, \bar{\R}} (f) := M_{B, \bar{B}}(\varphi)$.
\end{defi}


\end{document}
