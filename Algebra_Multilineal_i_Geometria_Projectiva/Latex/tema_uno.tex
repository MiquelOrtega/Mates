\section{Álgebra multilineal}


%%%%%%%%%%%%%%%%%%%%%%%%%%%%%%
% ESPACIO DUAL               %
%%%%%%%%%%%%%%%%%%%%%%%%%%%%%%


\subsection{Espacio dual}

\begin{defi}
	Sea $\E$ un $\k$-ev. Definimos el espacio dual de $\E$ como
	$\E^* = \{ \phi : \E \to \k \text{ lineales} \}$ (también es un
	$\k$-espacio vectorial)
\end{defi}
\begin{obs}
	Para definir $\E^*$ tenemos que usar bases de $\E$.
\end{obs}
\begin{defi}
	Si $B = \{ u_1,\cdots, u_n \}$ es una base de $\E$ ($\k$-ev.)
	definimos
	\[
		\begin{aligned}
			u_i^* \colon \E &\to \k\\
			u_j &\mapsto u_i^*(u_j) = \delta_{ij}
		\end{aligned}
	\]
	Y llamaremos base dual de $B$ a
	$B^* = \{ u_1^*, \cdots, u_n^* \}$ (que efectivamente es una
	base de $\E^*$).
\end{defi}
\begin{obs}
	En particular si $w \in \E^*$ y
	$\displaystyle w = \sum_{i=1}^{n} a_i u_i^*$, se cumple que:
	\[
		w(u_j) = \sum_{i=1}^{n} a_i u_i^*(u_j) = a_j
		\implies
		w = \sum_{i=1}^{n}w(u_i)u_i^*
	\]
\end{obs}

\begin{prop}[(cambios de base)]
	Sean $B_1$ y $B_2$ bases de $\E$ ($\k$-ev. de $\dim  \E = n$) y
	sean $B_1^*$ y $B_2^*$ las bases duales de $B_1$ y $B_2$.
	Si $S_{B_1B_2}$ es la matriz de cambio de base de $B_1$ a
	$B_2$, entonces:
	\[
		S_{B_1^*B_2^*} = (S_{B_1B_2}^{-1})^T = (S_{B_2B_1})^T
	\]
\end{prop}

\begin{prop}[(aplicaciones lineales)]
	Sean $\E$ y $\bb{F}$ $\k$-ev. y sea $\phi \colon \E \to \bb{F}$ una
	aplicación lineal, entonces $\phi$ induce la aplicación
	lineal siguiente:
	\[
		\begin{aligned}
		\phi^* \colon \bb{F}^* &\to \E^*\\
		w &\mapsto \phi^*(w) = w \circ \phi
		\end{aligned}
	\]
\end{prop}
\begin{obs}
	Si $\E$ y $\bb{F}$ son de dimensión finita, $\phi$ admite expresión
	matricial (en coordenadas). En particular:
	\[
    	\displaystyle \begin{rcases} B_1 \text{ base de } \E \\ B_2
    	\text{ base de } \bb{F} \end{rcases} \implies \phi \text{ viene dada por }
    	M_{B_1,B_2}(\phi)
	\]
	\[
	    \displaystyle \begin{rcases} B_1^*
    	\text{ base de } \E^* \\ B_2^* \text{ base de } \bb{F}^*
    	\end{rcases} \implies \phi^* \text{ viene dada por }
	    M_{B_2^*,B_1^*}(\phi^*) = (M_{B_1,B_2}(\phi))^T
	\]
\end{obs}
\begin{prop}[(espacio bidual)]
	Dado $\E$ $\k$-ev. podemos definir $\E^*,\E^{**},\cdots$. En
	particular tenemos que $\E^{**}$ es canónicamente isomorfo a
	$\E$ mediante el isomorfismo
	\[
		\begin{aligned}
			\phi \colon \E &\to \E^{**}\\
			u &\mapsto \phi(u)
		\end{aligned}
	\]
	donde
	\[
		\begin{aligned}
			\phi(u) \colon \E^* &\to \k \\
			w &\mapsto (\phi(u))(w) = w(u)
		\end{aligned}
	\]
\end{prop}
\begin{obs}
	Como este isomorfismo es canónico (no depende de las bases),
	$\E \cong \E^{**}$ y no distinguimos entre $\E$ y $\E^{**}$
\end{obs}


%%%%%%%%%%%%%%%%%%%%%%%
%                     %
%      TENSORES       %
%                     %
%%%%%%%%%%%%%%%%%%%%%%%

\subsection{Tensores}

\begin{defi}
	Sean $\E_1,\cdots , \E_r$ $\k$-ev. Diremos que
	$f \colon \E_1 \times \cdots \times \E_r \to \k$ es un tensor (o una aplicacion
	multilineal) si $\forall i=1,\cdots , r$ y $\forall v_j \in \E_j$ ($i \neq j$)
	se cumple que
	\[
		 \begin{aligned}
			 \phi_i \colon \E_i &\to \k \\
			 v &\mapsto \phi(u)=f(v_1,\cdots , v_{i-1}, v, v_{i+1}, \cdots , v_r)
		 \end{aligned}
	\]
	es una aplicación lineal.
\end{defi}
\begin{defi}
	Sea $\E$ un $\k$-ev. Llamaremos tensor de tipo $(p,q)$ (o tensor p veces
	covariante y q veces contravariante) (o tensor p-covariante y q-contravariante)
	a un tensor
	\[
		\begin{aligned}
			f \colon \overbrace{\E \times \cdots \times \E}^p\times
			\overbrace{\E^* \times \cdots \times \E^*}^q&\to \k \\
			(v_1, \cdots, v_p, w_1, \cdots, w_q) &\mapsto
			f(v_1,\cdots,v_p,w_1,\cdots,w_q)
		\end{aligned}
	\]
\end{defi}
\begin{obs}
	Al conjunto de tensores  de este tipo se le denota como $T_p^q(\E)$.
\end{obs}
\begin{obs}
	Por convenio $T_0(\E) = T^0(\E) = T_0^0(\E) = \k$.
\end{obs}
\begin{example}
	Sea $\E$ un $\k$-ev.
	\begin{itemize}
		\item $T_1(\E) = T_1^0(\E) = \E^*$
		\item $T^1(\E) = T_0^1 = \E^{**}$ ($\cong \E$)
		\item $T_2(\E) = T_2^0(\E) = \{\text{formas bilineales de } \E \text{ en } \k\}$
	\end{itemize}
\end{example}
\begin{prop}
	$T_p^q(\E) = T_q^p(\E^*)$ (cambiando el orden)
\end{prop}
\begin{prop}
	$T_p^q(\E)$ tiene estructura de $\k$-espacio vectorial. Si $f, g \in T_p^q(\E)$ y
	$\alpha , \beta \in \k$
	\[
		\begin{aligned}
			\alpha f + \beta g \colon \overbrace{\E \times \cdots \E}^p \times
			\overbrace{\E^* \times \cdots \times \E^*}^q &\to \k \\
			(v_1, \cdots, v_p, w_1, \cdots, w_q) &\mapsto (\alpha f + \beta g)
			(v_1, \cdots, v_p, w_1, \cdots, w_q)
		\end{aligned}
	\]
	donde 
	\[
	\begin{aligned}
	(\alpha f + \beta g) (v_1, \cdots, v_p, w_1, \cdots, w_q) = \ 
	&\alpha f(v_1, \cdots, v_p, w_1, \cdots, w_q) + \\ 
	&\beta g(v_1, \cdots, v_p, w_1, \cdots, w_q).
	\end{aligned}
	\]
\end{prop}
\begin{defi}[(producto tensorial)]
	Dados $f \in T_p^q(\E)$ y $g \in T_{p'}^{q'}(\E)$, definimos el producto
	tensorial de $f$ y $g$ como
	\[
		\begin{aligned}
			f \otimes g \colon \overbrace{\E \times \cdots \times \E}^{p+p'}
			\times \overbrace{\E^* \times \cdots \times \E^*}^{q+q'} &\to \k \\
			(v_1, \cdots, v_p, \overline{v_1}, \cdots \overline{v_{p'}},
			w_1, \cdots, w_q, \overline{w_1}, \cdots, \overline{w_{q'}})
			&\mapsto f(v_1, \cdots, v_p, w_1, \cdots, w_p) * \\
			&g(\overline{v_1}, \cdots, \overline{v_{p'}}, \overline{w_1}, \cdots
			\overline{w_{q'}})
		\end{aligned}
	\]
\end{defi}
\begin{obs}
	Si $f$ y $g$ son tensores, entonces $f \otimes g$ también lo es. Además
	$f \otimes g \in T_{p+p'}^{q+q'}(\E)$.
\end{obs}
\begin{prop}
	Sean  $f \in T_p^q(\E)$, $g \in T_{p'}^{q'}$ y $h \in T_{p''}^{q''}(\E)$.
	\begin{itemize}
		\item $\otimes$ {\bfseries NO} es abeliano. En general $f \otimes g \neq
		g \otimes f$.
		\item $\otimes$ es asociativo. $(f \otimes g) \otimes h = f \otimes (g
		\otimes h)$. Denotado por $f \otimes g \otimes h$
		\item $\vec{0} \otimes f = f \otimes \vec{0} = \vec{0}$
		\item $f \otimes (g + h) = f \otimes g + f\otimes h$ \quad ($(f+g) \otimes
		h = f \otimes h + g \otimes h$)
		\item $\alpha \in \k$. $(\alpha f) \otimes g = \alpha(f \otimes g) =
		f \otimes (\alpha g)$
	\end{itemize}
\end{prop}
\begin{example}
	Sea $\E = \real^3$, $B = \{e_1, e_2, e_3\}$, $B^* = \{ e_1^*, e_2^*, e_3^*\}$.
	Y consideramos el producto tensorial de los tensores $e_1^*$ y $e_2^*$ sobre
	los vectores $v_1 = (x_1, y_1, z_1)$ y $v_2 = (x_2,y_2,z_2)$.
	\[
		\begin{rcases}
			(e_1^* \otimes e_2^*)(v_1, v_2) = e_1^*(v_1)e_2^*(v_2) = x_1y_2 \\
			(e_2^* \otimes e_1^*)(v_1, v_2) = e_2^*(v_1)e_1^*(v_2) = y_1x_2
		\end{rcases}
		\implies e_1^* \otimes e_2^* \neq e_2^* \otimes e_1^*
	\]
\end{example}
\begin{example}
	Sea $\E = \real^2$, $B = \{e_1, e_2\}$, $B* = \{e_1^*, e_2^*\}$, entonces
	\[
		e_1 \otimes e_2 \in T^2(\E) \qquad \begin{cases}
			(e_1 \otimes e_2) = (e_1^{**} \otimes e_2^{**})(e_1^*,e_1^*) =
			e_1(e_1)e_2(e_1) = 0 \\
			(e_1 \otimes e_2)(e_1^*,e_2^*) = 1 \\
			(e_1 \otimes e_2)(e_2^*,e_1^*) = 0 \\
			(e_1 \otimes e_2)(e_2^*,e_2^*) = 0
		\end{cases}
	\]
\end{example}
\begin{obs} \label{obs:tens_cero}
	Si $\E$ es un $\k$-ev de dimensión $n$ y $B = \{e_1, \cdots, e_n\}$
	\[
		(\underbrace{e_{i_1}^* \otimes \cdots \otimes e_{i_p}^*}_
		{I = \setb{i_1, \cdots, i_p}} \otimes
		\underbrace{e_{j_1} \otimes \cdots \otimes e_{j_2}}_
		{J = \setb{j_1, \cdots j_q}})(\underbrace{e_{l_1}, \cdots, e_{l_p}}_
		{L = \setb{l_1, \cdots, l_p}}, \underbrace{e_{m_1}^*, \cdots, e_{m_q}^*}_
		{M = \setb{m_1, \cdots, m_q}}) =
		\begin{cases}
			1 \quad \text{Si } I = L \text{ y } J = M \\
			0 \quad \text{en otro caso}
		\end{cases}
	\]
\end{obs}
\begin{obs}
	Sean $f,g \in T_p^q(\E)$ entonces
	\[
		f=g \iff \substack{\forall e_{i_1}, \cdots, e_{i_p} \in B \\ \forall e_{j_1}^*, \cdots,
		e_{j_q}^* \in B^*} \quad f(e_{i_1}, \cdots, e_{j_q}^*) =
		g(e_{i_1}, \cdots, e_{j_q}^*)
	\]
\end{obs}

\subsection{Dimensión y bases de $T_p^q(\E)$}

Recordemos que $T_p^q(\E)$ es un $\k$-ev.

\begin{thm}[(base de $T_p^q(\E)$)]
	Sea $\E$ un $\k$-ev. de dimensión $n$ y sea $B = \setb{e_1, \cdots, e_n}$,
	entonces
\begin{enumerate}[i)]
		\item $\dim_\k T_p^q(\E) = n^{p+q}$
		\item \label{base_2} Una base de $T_p^q(\E)$ es
		\[
			B_p^q = \left\{ e_{i_1}^* \otimes \cdots \otimes e_{i_p}^* \otimes
			e_{j_1} \otimes \cdots \otimes e_{j_q} \vert
			\substack{i_1, \cdots, i_p \in \setb{1,\cdots,n} \\
				j_1, \cdots, j_1 \in \setb{1,\cdots,n}} \right\}
		\]
		\item \label{base_3} Si $f \in T_p^q(\E)$, las coordenadas de $f$ en la base $B_p^q$ son
		\[
			f_{B_p^q} = (f(e_{i_1}, \cdots, e_{i_p}, e_{j_1}^*, \cdots, e_{j_q}^*))
		\]
	\end{enumerate}
\end{thm}
\begin{proof}
	\begin{enumerate}[i)]
		\item[]
		\item Es consecuencia directa de \ref{base_2}
		\item Primero veamos que $B_p^q$ es linealmente independiente. Sea
		\[
			w = \sum \alpha_{IJ}(e_{i_1}^* \otimes \cdots \otimes e_{i_p}^*
			\otimes e_{j_1} \otimes \cdots \otimes e_{j_q}) = 0
		\]
		Sean $I_0$, $J_0$ dos conjuntos de índices cualesquiera, entonces
		\[
			0 = w(e_{i_1}, \cdots, e_{i_p}, e_{j_1}^*, \cdots, e_{j_q}^*)
			= \alpha_{I_0J_0}
		\]
		(Por la \ref{obs:tens_cero}). Veamos ahora que $B_p^q$
		es generadora. Sea $f \in T_p^q(\E)$, definimos $g \in T_p^q(\E)$ como
		\[
			g = \sum_{\forall I,J} (f(e_{i_1}, \cdots, e_{i_p}, e_{j_1}^*, \cdots,
			e_{j_q}^*)(e_{i_1}^* \otimes \cdots \otimes e_{i_p}^* \otimes e_{j_1}
			\otimes \cdots \otimes e_{j_q}))
		\]
		Demostrando ahora que $f=g$ quedan provados \ref{base_2} y \ref{base_3}.
		Tenemos ahora que
		\[
			g(e_{i_1^0}, \cdots, e_{i_p^0}, e_{j_1^0}^*, \cdots, e_{j_q^0}^*) =
			f(e_{i_1^0}, \cdots, e_{i_p^0}, e_{j_1^0}^*, \cdots, e_{j_q^0}^*)
		\]
		Por la \ref{obs:tens_cero} y queda demostrado el teorema.
	\end{enumerate}
\end{proof}
\begin{example}
	Sea $\E = \real^n$, $B = \setb{e_1, \cdots, e_n}$ y
	$B^* = \setb{e_1^* \cdots e_n^*}$
	\begin{itemize}
		\item Sea $u \in \real^n$
		\[
			u = u(e_1^*) + \cdots + u(e_n^*) \qquad (B_0^1 = B)
		\]
		\item Sea $w \in T_1^0(\E) (= \E^*)$
		\[
			w = w(e_1)e_1^* + \cdots + w(e_n)e_n^* \qquad (B_1^0 = B^*)
		\]
		\item Sea $n = 3$ y sea $f \in T_2(\E)$
		\begin{gather*}
			B_2^0 = \setb{e_1^* \otimes e_1^*, e_1^* \otimes e_2^*,
			\cdots e_3^* \otimes e_3^*} \\ f = f(e_1,e_1)e_1^*\otimes e_1^* +
			f(e_1,e_2)e_1^*\otimes e_2^* +\cdots + f(e_3,e_3)e_3^* \otimes e_3^*
		\end{gather*}
	\end{itemize}
\end{example}
\begin{prop}[ (cambio de base) ]
	Sea $\E$ un $\k$-ev. de dimensión $n$ y sean $B = \setb{e_1, \cdots, e_n}$ y
	$\overline{B} = \setb{u_1, \cdots, u_n}$. Sea $S = (s_j^i)$ la matriz de
	cambio de base de $\overline{B}$ a $B$ y sea $T = (t_ j^i)$ su inversa.
	De manera que tenemos esta relación:
	\[
		\begin{tikzpicture}
			\node (B) {$\overline{B}$};
			\node[left=of B] {$B$}
			edge[<-,bend left=30] node[auto] {$S$} (B)
			edge[->,bend right=30] node[auto] {$T$} (B);
		\end{tikzpicture}
		\qquad
		\begin{tikzpicture}
			\node (B) {$\overline{B}^*$};
			\node[left=of B] {$B^*$} 
			edge[->,bend left=30] node[auto] {$S^t$} (B)
			edge[<-,bend right=30] node[auto] {$T^t$} (B);
		\end{tikzpicture}
	\]
	Sea $f \in T_p^q(\E)$ y sean
	\begin{gather*}
		f_B = \left( \alpha_{IJ} \right)_{I,J} = \left( f(e_{i_1}, \cdots,e_{i_p},
		e_{j_1}^*, \cdots, e_{j_q}^*) \right)_{I,J} \\
		f_{\overline{B}} = \left( \overline{\alpha}_{IJ} \right)_{I,J} =
		\left( f(u_{i_1}, \cdots,u_{i_p}, u_{j_1}^*, \cdots, u_{j_q}^*)
		\right)_{I,J}
	\end{gather*}
	Entonces, $\forall I,J$
	\[
		\overline{\alpha}_{IJ} = f(u_{i_1}, \cdots,u_{i_p}, u_{j_1}^*, \cdots,
		u_{j_q}^*) = \sum_{\forall L, M} s_{i_1}^{l_1} \cdots s_{i_p}^{l_p}
		t_{m_1}^{j_1} \cdots t_{m_q}^{j_q} f(e_{i_1}, \cdots, e_{i_p}, e_{j_1}^*,
		\cdots, e_{j_q}^*)
	\]
\end{prop}
\hfill
\begin{example}
	\begin{itemize}
                \item []
		\item $f \in T^1(\E) = \E^{**} = \E$ por lo tanto $f = u$ y $\substack{
		u_B = (x_1, \cdots, u_n) \\ u_{\overline{B}} = (\overline{x}_1, \cdots,
		\overline{x}_n)}$, entonces
		\[
			\begin{pmatrix}
				\overline{x}_1 \\
				\vdots \\
				\overline{x}_n
			\end{pmatrix}
			= T \begin{pmatrix}
				x_1 \\
				\vdots \\
				x_n
			\end{pmatrix}
		\]
		\item $f \in T_1(\E) = \E^*$ por lo tanto $f = w$ y $\substack{
		w_B = (x_1, \cdots, u_n) \\ w_{\overline{B}} = (\overline{x}_1, \cdots,
		\overline{x}_n)}$, entonces
		\[
			\begin{pmatrix}
				\overline{x}_1 \\
				\vdots \\
				\overline{x}_n
			\end{pmatrix}
				= S^t \begin{pmatrix}
				x_1 \\
				\vdots \\
				x_n
			\end{pmatrix}
		\]
		\item $f \in T_2(\E)$ por lo tanto $f$ es una forma bilineal y $\substack{
		f_B = A \in M_{n,n}(k) \\ f_{\overline{B}} = \overline{A} \in M_{n,n}(k)}$,
		entonces
		\[
			\overline{A} = S^tAS
		\]
	\end{itemize}
\end{example}

%%%%%%%%%%%%%%%%%%%%%%%
%                     %
%    PERMUTACIONES    %
%                     %
%%%%%%%%%%%%%%%%%%%%%%%


\subsection{Recordatorio de permutaciones}

\begin{itemize}
	\item Denotaremos como $x_n = \setb{1, \cdots, n}$
	\item Denotaremos como $\mathcal{S}_n =\setb{\sigma\colon x_n \to
		x_n \text{ bilineales}}$
	\item $\#\mathcal{S}_n= n!$
	\item $\mathcal{S}_n$ es un grupo por composición. Además denotaremos
		$s_1 s_2 = s_1 \circ s_2$
	\item\label{obs:perm_biy} Fijada $s_0 \in \mathcal{S}_n$, la aplicación
		\[
			\begin{aligned}
				\phi \colon \mathcal{S}_n &\to \mathcal{S}_n \\
				s &\mapsto s_0s
			\end{aligned}
		\]
		es biyectiva.
	\item Sea $s \in \mathcal{S}_n$, denotaremos $s$  de las siguientes maneras
		\begin{itemize}
			\item $s = \begin{pmatrix}
				1 & 2 & \cdots & n \\
				s(1) & s(2) & \cdots & s(n)
			\end{pmatrix}$
			\item Si $s$ es cíclica la denotaremos como $s = (1,3,7,5)$. En este
			caso, $s(1)=3$, $s(3)=7$, $s(7)=5$ y $s(5)=1$, para el resto de valores
			$s(i)=i$.
		\end{itemize}
	\item Llamaremos trasposición a una permutación del tipo $s=(i,j)$ con
		$i \neq j$
	\item $\forall s \in \mathcal{S}_n$, $s$ se puede expresar como composición
		(o producto) de trasposiciones. Además, la paridad del número de
		trasposiciones se mantiene, es decir
		\[
			s = t_1 \cdots t_p = l_1 \cdots l_q \implies p \equiv q \mod 2
		\]
	\item Sea $s \in \mathcal{S}_n$ y sea $s=t_1 \cdots t_p$ una descomposición de
		$s$ en trasposiciones. Entonces, definimos el signo de $s$ como 
		$\varepsilon(s) = (-1)^p$.
\end{itemize}

\subsection{Tensores simétricos y antisimétricos}

\begin{defi}
	Sea $\E$ un $\k$-ev. de dimensión $n$, sea $f \in T_p(\E)$ y $s \in\mathcal{S}_p$,
	entonces, definimos $(\underline{s}f) \in T_p(\E)$ como
	\[
		(\underline{s}f)(v_1, \cdots, v_p) = f(v_{s(1)}, \cdots, v_{s(p)})
	\]
\end{defi}
\begin{example}
	Sea $\E = \real^4$, $B = \setb{e_1, e_2, e_3, e_4}$, $f = e_1^* \otimes e_2^*
	\otimes e_3^* \in T_3(\E)$ y $s = (1,2,3) \in \mathcal{S}_3$, entonces
	\[
		(\underline{s}f)(v_1,v_2,v_3) = f(v_2,v_3,v_1)
	\]
\end{example}

\begin{prop}
    Sea $\E$ un $\k$-ev. de dimensión $n$, sean $w_1, \dots, w_p \in T_1(\E) = \E^\ast$, 
    y $s \in \mathcal{S}_p \ (t = s^{-1})$. Entonces 
    \[\underline{s}(w_1 \otimes \dots \otimes w_p) = w_{t(1)}
    \otimes \dots \otimes w_{t(p)}\]
\end{prop}
\begin{proof}
    Sean $u_1, \dots, u_n \in \E$ (obsérvese que $w_1 \otimes \dots \otimes w_p \in T_p(\E)$)
    \[\underline{s}(w_1 \otimes \dots \otimes w_p)(u_1,\dots, u_p) = (w_1 \otimes 
    \dots \otimes w_p)(u_{s(1)}, \dots, u_{s(p)}) = w_1(u_{s(1)}) \cdot w_2(u_{s(2)}) 
    \cdots w_p(u_{s(p)})\]
    Dado que $s(i) = j \iff i = t(j)$, $w_i(u_{s(i)}) = w_i(u_j) = w_{t(j)}(u_j)$. Con lo
    que podemos reordenar el último producto como
    \[w_{t(1)}(u_1) \cdot w_{t(2)}(u_2) \cdots w_{t(p)}(u_p) =w_{t(1)}
    \otimes \dots \otimes w_{t(p)}(u_1,\dots, u_p)\]
\end{proof}
\begin{example}
    \label{exemple_1.6.2}
    Sea $\E=\real^3$, $B = \setb{e_1,e_2,e_3}$, $B^\ast = 
    \setb{e_1^\ast, e_2^\ast, e_3^\ast}$ para los dos siguientes ejemplos.

    En el primero fijamos $(1,2) = s \in \mathcal{S}_2$. Entonces 
    $s = (1,2) = s^{-1}=t$ y para los siguientes elementos de $T_2(\E)$ se cumple:
    \[f_1=e_1^\ast \otimes e_2^\ast \qquad \underline{s} f_1 = e_2^\ast \otimes 
    e_1^\ast\]
    \[f_2=e_1^\ast \otimes e_1^\ast \qquad \underline{s} f_2 = e_1^\ast \otimes 
    e_1^\ast\]
    \[f_3=e_2^\ast \otimes e_3^\ast \qquad \underline{s} f_3 = e_3^\ast \otimes 
    e_2^\ast\]

    En el segundo fijamos $(1,2,3) = s \in \mathcal(S)_3$. Entonces $t = s^{-1} = 
    (1, 3, 2)$, es decir, que $t(1) = 3, t(2) = 1, t(3) = 2$, y para el siguiente 
    elemento de $T_3(\E)$ se cumple:
    \[f = e_1^\ast \otimes e_2^\ast \otimes e_3^\ast \qquad 
    \underline{s} f =  e_3^\ast \otimes e_1^\ast \otimes e_2^\ast\]
\end{example}
\begin{obs}
\label{obs_1.6.1}
    Sea $f_i \in T_p(\E)$. Entonces $\underline{s}(\sum \alpha_i f_i) = \sum \alpha_i 
    (\underline{s} f_i)$.
    Por tanto, la proposición anterior sirve para $\forall f \in T_p(\E)$.
\end{obs}
\begin{example}
    Con las mismas hipótesis que en el primer caso del ejemplo \ref{exemple_1.6.2} 
    se cumple:
    \[f = 3e_1^\ast \otimes e_2^\ast + 5e_1^\ast \otimes e_1^\ast + 5e_2^\ast \otimes
    e_3^\ast \qquad \underline{s}f = 3e_2^\ast \otimes e_1^\ast + 5e_1^\ast \otimes e_1^\ast 
    + 5e_3^\ast \otimes e_2^\ast\]
\end{example}
\begin{defi}
    Sea $\E$ un $\k$-ev. de $\dim n$. Sea $f \in T_p(\E)$.
    \begin{enumerate}
        \item
            $f$ es simétrica $\iff \forall s \in \mathcal{S}_p \quad \underline{s}f = f$
        
        \item
            $f$ es antisimétrica $\iff \forall s \in \mathcal{S}_p \quad \underline{s}f 
            = \varepsilon (s) f$
        
        \item
            $S_p(\E) = \setb{f \in T_p(\E) \mid f \text{ simétrica}} \subseteq T_p(\E)$\\
            $A_p(\E) = \setb{f \in T_p(\E) \mid f \text{ antisimétrica}} \subseteq T_P(\E)$
        
    \end{enumerate}
\end{defi}
\begin{obs}
    $S_p(\E), A_p(\E) \subseteq T_p(\E)$ son s.e-v. (ver observación \ref{obs_1.6.1}).
\end{obs}
\begin{example}
   Para los dos ejemplos, sea $\E = \real^3$, sean $B$ y $B^\ast$ bases de $\E$ 
   y de $\E^\ast$ correspondientemente. 
    \begin{enumerate}
        \item
            Definimos $f = e_1^\ast \otimes e_2^\ast \in T_2(\E)$
            y $s = (1,2) \in \mathcal(S)_2$. Entonces            
            $\mathcal{S}_2 = \setb{\Id, s}$ y $\varepsilon(\Id) = 1$, 
            $\varepsilon(s) = -1$.
            
            \[
            \left.
                \begin{array}{r}
                    \underline{\Id}(f) = f = \varepsilon (\Id) \cdot f\\
                    \underline{s}(f) = e_2^\ast \otimes e_1^\ast \neq f\text{, } 
                    \underline{s}(f)\neq -f
                \end{array} 
            \right \} 
            \implies
            \begin{array}{l}
                f \notin S _2(\E)\\
                f \notin A_2(\E)
            \end{array}
            \]      
        \item
            Como anteriormente, $\mathcal{S}_2 = \setb{\Id, s = (1,2)}$.
            \begin{itemize}
                \item
                    Para $f = e_1^\ast \otimes e_2^\ast + e_2^\ast \otimes e_1^\ast$, 
                    \[
                        \left.
                        \begin{array}{r}
                            \underline{\Id}(f) = f\\
                            \underline{s}(f) = e_2^\ast \otimes e_1^\ast
                            + e_1^\ast \otimes e_2^\ast = f
                        \end{array} 
                        \right \} 
                        \implies
                        f \in S_2(\E)
                    \]
                \item
                    Para $f = e_1^\ast \otimes e_2^\ast - e_2^\ast \otimes e_1^\ast$, 
                    \[
                        \left.
                        \begin{array}{r}
                            \underline{\Id}(f) = f = \varepsilon(\Id)f\\
                            \underline{s}(f) = e_2^\ast \otimes e_1^\ast
                            - e_1^\ast \otimes e_2^\ast = -f = \varepsilon(s)f
                        \end{array} 
                        \right \} 
                        \implies
                        f \in A_2(\E)
                    \]
            \end{itemize}
        
    \end{enumerate}
\end{example}
\begin{obs}~
    \begin{itemize}
        \item $\varepsilon (s) = (-1)^n$ si $\varepsilon (s) = t_1 \cdots t_n$, donde 
        $t_1,\dots,t_n$ son transposiciones.
        \item $\varepsilon (s_1 s_2) = \varepsilon (s_1) \varepsilon(s_2)$.
        \item $s \in \mathcal{S}_p$. Definimos $A \in \mathcal{M}(\real)_{p\times p}$, donde
        $a_{i,j} = 1$ si $i = s(j)$ y $a_{i,j} = 0$ en otro caso. Entonces
        $\det A = \varepsilon(s)$.       
    \end{itemize}
\end{obs}
\begin{prop}\label{prop:sim_comp}
    Sea $\E$ un $\k$-e.v. de dimensión $n$, sea $f \in T_p(\E)$.
    \begin{enumerate}
        \item Podemos caracterizar los tensores simétricos como:
        \[
        \begin{split}\text{f simétrica } \iff &
        \forall u_1,\dots,u_p \in \E, \  \forall i,j \quad 
        f(u_1,\dots,u_i,\dots,u_j,\dots,u_p) = \\ 
        &f(u_1,\dots,u_j,\dots,u_i,\dots,u_p)
        \end{split}
        \]
        \item Podemos caracterizar los tensores antisimétricos como: 
        \[
        \begin{split}\text{f antisimétrica} \iff & 
        \forall u_1,\dots,u_p \in \E, \  \forall i \neq j \quad
        f(u_1,\dots,u_i,\dots,u_j,\dots,u_p) = \\ 
        &-f(u_1,\dots,u_j,\dots,u_i,\dots,u_p)\\
        \iff& \forall u_1,\dots,u_p \in \E, \ \forall i \neq j \text{ si } u_i=u_j 
        \text{ entonces } f(u_1,\dots,u_p) = 0.
        \end{split}\]
    \end{enumerate}
\end{prop}
\begin{proof}~
    \begin{enumerate}
        \item La implicación directa es una consecuencia de la definición de simetría.
        
        En el caso de la implicación conversa se cumple:
        \[
        \begin{split} \forall t \text{ transposición } \underline{t}f = f \implies & \forall 
        t_1,\dots ,t_m \text{ transposiciones } \underline{t_1,\dots,t_m} f =\\
        &\underline{t_1}(\underline{t_2}(\dots (\underline{t_m}f)\dots)) = f
        \end{split}
        \]
        Finalmente, 
        \[\forall s \in \mathcal{S}_p \quad s = t_1 \cdots t_m \implies \forall s \in 
        \mathcal{S}_p \quad \underline{s}f = f \implies \text{ f es simétrica.}\]
        
        \item Veamos primero que la tercera condición implica la segunda.
        \[\begin{split}
        &\forall u_1,\dots,u_p \in E, \forall i \neq j \quad 
        0 = f(u_1,\dots,u_i+u_j,\dots,u_i+u_j, \dots, u_p) =\\
        &f(u_1,\dots,u_i,\dots,u_i,\dots,u_p)+
        f(u_1,\dots,u_i,\dots,u_j,\dots,u_p)+\\
        &f(u_1,\dots,u_j,\dots,u_i,\dots,u_p)+
        f(u_1,\dots,u_j,\dots,u_j,\dots,u_p)=\\
        &f(u_1,\dots,u_i,\dots,u_j,\dots,u_p)+
        f(u_1,\dots,u_j,\dots,u_i,\dots,u_p)\implies\\
        &f(u_1,\dots,u_i,\dots,u_j,\dots,u_p) = 
        -f(u_1,\dots,u_j,\dots,u_i,\dots,u_p)
        \end{split}\]
        Veamos ahora que la segunda condición implica la primera. Suponiendo cierta
        la segunda condición se cumple:
        \[\underline{t}f = -f \implies \underline{t_1\cdots t_m}f = (-1)^mf =
        \varepsilon(t_1\cdots t_m)f\]
        Y entonces:
        \[\forall s \in \mathcal{S}_p \quad s = t_1\cdots t_m \text{ y } \varepsilon(s) = (-1)^m 
        \implies \forall s\in \mathcal{S}_p \quad \underline{s}f = \varepsilon(s)f
        \implies f \in A_p(E)\]
        Y, finalmente, que la primera implica la tercera. Por ser $f$ antisimétrica,
        \[
        \forall u_1,\dots,u_p \in E, \ \forall i \neq j \quad
        f(u_1,\dots, u_i, \dots, u_j, \dots, u_p) = -f(u_1,\dots, u_j, \dots, u_i, \dots, u_p)
        \]
        Si $u_i = u_j$, entonces $f(u_1, \dots, u_p) = 0$.
        
    \end{enumerate}
\end{proof}
\begin{prop}
    Sea $\E$ un $\k$-ev. con $\car \k \neq 2$ y sea $f \in T_p(\E)$, entonces $\forall v_i, v_j \in \E$
    \[
        f(\cdots, v_i, \cdots, v_j, \cdots) = -f(\cdots,v_j,\cdots,v_i,\cdots) \iff
        f(\cdots, w, \cdots, w, \cdots) = 0
    \]
\end{prop}
\begin{proof}
    $\implies$
    \begin{gather*}
        f(\cdots,w,\cdots,w,\cdots) = - f(\cdots,w,\cdots,w,\cdots) \implies \\
        2f(\cdots,w,\cdots,w,\cdots) = 0 \implies f(\cdots,w,\cdots,w,\cdots) = 0
    \end{gather*}
    $\impliedby$
    \begin{gather*}
        f(\cdots, v_i+v_j, \cdots, v_i+v_j, \cdots) = 0 \implies \\
        f(\cdots,v_i,\cdots,v_i,\cdots) + f(\cdots,v_j,\cdots,v_j,\cdots) +
        f(\cdots,v_i,\cdots,v_j,\cdots) + f(\cdots,v_j,\cdots,v_i,\cdots) = 0 \\
        \implies f(\cdots,v_i,\cdots,v_j,\cdots) = -f(\cdots,v_j,\cdots,v_i,\cdots)
    \end{gather*}
\end{proof}
\begin{example}
    Sea $f \in T_2(\E)$ (forma bilineal) y sea $B = \setb{e_1,\dots,e_n}$ una base,
    Llamamos $M_B(f) = A = \left( f(e_i,e_j) \right)$ y sea $\mathcal{S}_2 = \setb{\Id, s=(1,2)}$.
    Entonces
    \[
        \underline{\Id}f = f \qquad M_b(\underline{s}f)= \left( \underline{s}f(e_i,e_j) \right)
        = \left( f(e_j,e_i) \right) = A^t
    \]
    Es decir, $f$ es simétrico si y solo si $A^t = A \iff A$ simétrica y $f$ es
    antisimétrico si y solo si $A = -A^t \iff A$ antisimétrica
\end{example}
\begin{example}
    Sea $\dim \E = n$ y $\setb{e_1,\dots,e_n}$ una base de $\E$, entonces
    \[
        \begin{aligned}
            f \colon \overbrace{E \times \cdots \times E}^n &\to \k \\
            (u_1,\dots,u_n) &\mapsto det_B(u_1,\dots,u_n)
        \end{aligned}
    \]
    Como $f$ es multilineal ($\implies f \in T_n(\E)$), $f$ es antisimétrico
    por la proposición \ref{prop:sim_comp}.
\end{example}
\begin{defi}
    Sea $\E$ un $\k$-ev. con $\car \k = 0$ y $f \in T_p(\E)$. Llamamos simetrizado de $f$ a
    \[
        S(f) = \frac{1}{p!}\sum_{s \in \mathcal{S}_p} \underline{s}f
    \]
    y antisimetrizado de $f$ a
    \[
        A(f) = \frac{1}{p!}\sum_{s \in \mathcal{S}_p} \varepsilon(s)\underline{s}f
    \]
\end{defi}
\begin{example}
    Sea $\E = \real^3$ y $B = \setb{e_1,e_2,e_3}$ base de $\E$
    \begin{itemize}
        \item Sea $f = e_1^* \otimes e_2^* \in T_2(\E)$ y $\mathcal{S}_p=\setb{\Id,s=(1,2)}$,
        entonces
        \begin{gather*}
            S(f) = \frac{1}{2} (e_1^* \otimes e_2^* + e_2^* \otimes e_1^*) \\
            A(f) = \frac{1}{2} (e_1^* \otimes e_2^* - e_2^* \otimes e_1^*)
        \end{gather*}
        \item Sea $g = e_1^* \otimes e_2^* \otimes e_3^*$ y $\mathcal{S}_p =
        \setb{\Id,(1,2),(1,3),(2,3),(1,2,3),(1,3,2)}$, entonces
        \[
            S(g) = \frac{1}{6} (e_1^*\otimes e_2^*\otimes e_3^* + e_2^*\otimes e_1^*\otimes e_3^* +
            e_3^*\otimes e_2^*\otimes e_1^* + e_1^*\otimes e_3^*\otimes e_2^* +
            e_3^*\otimes e_1^*\otimes e_2^* + e_2^*\otimes e_3^*\otimes e_1^*)
        \]
    \end{itemize}
\end{example}
\begin{obs}
    $\inv{s} \in \mathcal{S}_p$, por lo tanto no hace falta calcular $\inv{s}$
\end{obs}
\begin{example}
    Sea $f \in T_2(\E)$ (forma bilineal) y sea $M_B(f)=A= \left( f(e_i,e_j) \right)$, entonces
    \[
        M_B(S(f)) = \frac{1}{2}\left( M_B(f) + M_B(f)^t \right) = \frac{1}{2}(A + A^t)
    \]
    \[
        M_B(A(f)) = \frac{1}{2}\left( M_B(f) - M_B(f)^t \right) = \frac{1}{2}(A - A^t)
    \]
\end{example}
\begin{obs}
    Si $f \in T_2(\E) \implies f = S(f) + A(f)$ 
\end{obs}
\begin{prop}
    Sea $\E$ un $\k$-ev. Consideramos $A,S \colon T_p(\E) \to T_p(\E)$, entonces
    \begin{enumerate}[i)]
        \item $A,S$ son lineales
        \item\label{item:s_sp} $f \in S_p(\E) \implies S(f) = f$ y $f \in A_p(\E) \implies A(f) = f$
        \item $\im(S)=S_p(\E)$ y $\im(A)=A_p(\E)$
    \end{enumerate}
\end{prop}
\begin{proof}
    \begin{enumerate}[i)]
        \item[]
        \item Queda como ejercicio. (pista: Consideramos $S(f+g)$)
        \item $f \in S_p(\E) \implies S(f) = f$ queda como ejercicio.
        Suponemos que $f \in A_p(\E)$, entonces
        \[
            A(f) = \frac{1}{p!} \left(\sum_{s \in \mathcal{S}_p} \varepsilon(s)\underline{s}f \right)
            = \frac{1}{p!}\sum_{s \in \mathcal{S}_p} \varepsilon(s) \left( \varepsilon(s)f \right) =
            \frac{p!f}{p!} = f
        \]
        \item $\im(S)=S_p(\E)$ queda como ejercicio. Demostraremos que $\im(A)=A_p(\E)$.
        Por \ref{item:s_sp} sabemos que $A_p(\E) \subseteq \im(A)$, por lo tanto, resta
        ver que $A(h)\in A_p(\E)$, $\forall h \in T_p(\E)$. Sea $s \in \mathcal{S}_p$
        \[
            \underline{s}A(h) = \underline{s}\left( \frac{1}{p!}\sum_{r \in \mathcal{S}_p} \varepsilon(r)\underline{r}h
            \right) = \frac{1}{p!}\left(\sum_{r \in \mathcal{S}_p} \varepsilon(r)\underline{s}(\underline{r}h)\right) =
        \]
        \[
            = \varepsilon(s)\frac{1}{p!} \left( \sum_{r \in \mathcal{S}_p} \varepsilon(sr) \underline{sr}h \right)
            \stackrel{\ref{obs:perm_biy}}{=} \varepsilon(s)\frac{1}{p!} \left( \sum_{t \in \mathcal{S}_p} \varepsilon(t)
            \underline{t}h \right) = \varepsilon(s)A(h)
        \]
    \end{enumerate}
\end{proof}
\begin{obs}
	Las mismas construcciones funcionan para tensores $(0,q)$, $T^q(\E) = T_q(\E^*)$, pero no funcionan para
	tensores $(p,q)$ donde $p,q \neq 0$ porque las construcciones implican permutaciones.
\end{obs}

%%%%%%%%%%%%%%%%%%%%%%%
%                     %
%  PRODUCTO EXTERIOR  %
%                     %
%%%%%%%%%%%%%%%%%%%%%%%

\subsection{Producto exterior}

\begin{obs}
	$S_p(\E) \subseteq T_p(\E)$, $A_p(\E) \subseteq T_p(\E)$ y $S_p$, $A_p$ s.e.v..
	\begin{itemize}
	    \item $f \in S_p(\E)$, $g \in S_{p'}(\E)$ en general $f \otimes g \notin S_{p+p'}(\E)$
	    \item $f \in A_p(\E)$, $g \in A_{p'}(\E)$ en general $f \otimes g \notin A_{p+p'}(\E)$
	\end{itemize}
\end{obs}
\begin{example}
    Sean $\omega_1$, $\omega_2 \in T_1(\E)$:
	\begin{itemize}
		\item $T_1(\E) = \E^* = S_1(\E) = A_1(\E)$, porque $S_1 = \{\Id\}$.
		\item $\omega_1 \otimes \omega_2 \notin S_2(\E), A_2(\E)$.
	\end{itemize}
\end{example}
\begin{obs}
	El producto exterior (que definiremos) manda tensores antisimétricos a antisimétricos.
\end{obs}
\begin{obs}
	Lo haremos en $T_p(\E)$, análogamente se hará en $T^q(\E)$.
\end{obs}
\begin{defi}[(producto exterior de orden 1)] \label{pr_ext}
	Sea $\E$ un $\k$-e.v.; $\omega_1, \dots, \omega_p \in T_1(\E) = \E$, el producto exterior de orden 1 es:
	\[
		\omega_1 \wedge \dots \wedge \omega_p = p! A(\omega_1 \otimes \dots \otimes \omega_p)
	\]
\end{defi}
\begin{obs} \label{obs_pr_ext}
	\begin{gather*}
		\omega_1 \wedge \dots \wedge \omega_p = \cancel{p!}  \left( \frac{1}{\cancel{p!}}\sum_{s \in \mathcal{S}_p} \varepsilon(s)
		\underline{s}\left(\omega_1 \otimes \dots \otimes \omega_p \right) \right) = \\
		= \sum_{s \in \mathcal{S}_p} \varepsilon(s) \left( \omega_{s^{-1}(1)} \otimes \dots \otimes \omega_{s^{-1}(p)} 
		\right) \stackrel{\varepsilon(s) = \varepsilon(s^{-1})}{=} \sum_{r \in \mathcal{S}_p} 
		\varepsilon(r) \left( \omega_{r(1)} \otimes \dots \otimes \omega_{r(p)} \right)
	\end{gather*}
\end{obs}
\begin{example}
    Sea $\E = \real^3$, $B = \{e_1, e_2, e_3\}$, $B^* = \{ e_1^*, e_2^*, e_3^*\}$.
    \begin{itemize}
        \item $e_1^* \wedge e_2^* = e_1^* \otimes e_2^* - e_2^* \otimes e_1^*$
        \item $e_2^* \wedge e_1^* = \dots = -e_1^* \wedge e_2^*$
        \item $e_1^* \wedge e_1^* = e_1^* \otimes e_1^* - e_1^* \otimes e_1^* = 0$
    \end{itemize}
\end{example}
\begin{prop} \label{prop_pr_ext}
    Sean $\omega_1, \dots \omega_p \in T_1(\E) = \E^*$
    \begin{enumerate}[i)]
        \item $\omega_1 \wedge \dots \wedge \omega_p \in A_p(\E)$
        \item \label{prop_pr_ext_ii} $\omega_1 \wedge \dots \wedge \left(\alpha_i \overline{\omega_i} + \beta_i \overline{\overline{\omega_i}} \right).
        \wedge \dots \wedge \omega_p = \alpha_i (\omega_1 \wedge \dots \wedge \overline{\omega_i} \wedge \dots.
        \wedge \omega_p) + \beta_i (\omega_1 \wedge \dots \wedge \overline{\overline{\omega_i}} \wedge \dots
        \wedge \omega_p)$
        \item \label{prop_pr_ext_iii} Sea $s \in \mathcal{S}_p$, $t = s^{-1}$, $\omega_{s(1)} \wedge \dots \wedge \omega_{s(p)} =
        \underline{t}\left(\omega_1 \wedge \dots \wedge \omega_p \right) = \varepsilon(s)\left( \omega_1
        \wedge \dots \wedge \omega_p \right)$.
        \item \label{prop_pr_ext_iv} Si $u_1, \dots, u_p \in \E$, $\left( \omega_1 \wedge \dots \wedge \omega_p\right) \left(u_1, \dots,
        u_p\right) = \det\left(\omega_j(u_i)\right)$.
        \item \label{prop_pr_ext_v} Si $\omega_i = \omega_j , (i \neq j) \implies \omega_1 \wedge \dots \wedge \omega_p = 0$.
        \item $\omega_1 \wedge \dots \wedge \omega_p \neq 0 \iff \{\omega_1,\dots,\omega_p\}$ son linealmente independientes.
    \end{enumerate}
\end{prop}
\begin{proof}
    \begin{enumerate}[i)]
        \item[]
        \item $\omega_1 \wedge \dots \wedge \omega_p = p! A\left(\omega_1 \otimes \dots \otimes \omega_p .
        \right) \implies \omega_1 \wedge \dots \wedge \omega_p \in \im(A) \stackrel{\text{visto}}{=} A_p(\E)$.
        \item Ejercicio (misma proposición que $\otimes$).
        \item $\omega_{s(1)} \wedge \dots \wedge \omega_{s(p)} = \underline{t}(\omega_1 \wedge \dots \wedge 
        \omega_p)$ (Ejercicio, misma proposición que $\otimes$).
        \begin{gather*}
            \omega_{s(1)} \wedge \dots \wedge \omega_{s(p)} \stackrel{\ref{pr_ext} + \ref{obs_pr_ext}}{=}
            \sum_{r \in \mathcal{S}_p} \varepsilon(r) \left( \omega_{rs(1)} \otimes \dots \otimes 
            \omega_{rs(p)} \right) = \\ 
            \stackrel{\varepsilon^2(s)=1}{=} \varepsilon(s) \sum_{r \in \mathcal{S}_p} \overbrace{\varepsilon(r)
            \varepsilon(s)}^{\varepsilon(rs)} \left( \omega_{rs(1)} \otimes \dots \otimes\omega_{rs(p)}
            \right) \stackrel{rs = m \in \mathcal{S}_p}{=} \varepsilon(s) \sum \varepsilon(m)
            \left( \omega_{m(1)} \otimes \dots \otimes \omega_{m(p)} \right) = \\
            = \varepsilon(s) \left( \omega_1 \wedge \dots \wedge \omega_p \right).
        \end{gather*}
        \item 
        \begin{gather*}
        \overbrace{\left( \omega_1 \wedge \dots \wedge \omega_p \right)}^{T_p(\E)} \left(u_1,\dots,u_p
        \right) = \sum_{r\in\mathcal{S}_p} \varepsilon(r) \left( \omega_{r(1)} \otimes \dots \otimes 
        \omega_{r(p)} \right) \left( u_1,\dots,u_p \right) = \\
        \sum_{r \in \mathcal{S}_p} \varepsilon(r) \left( \omega_{r(1)}(u_1) \dots \omega_{r(p)}(u_p) \right)
        \stackrel{\text{def det}}{=} det(\omega_j(u_i))_{i,j}.
        \end{gather*}
        \item
        \begin{gather*}
        \omega_1 \wedge \dots \wedge \omega_i \wedge \dots \wedge \omega_i \wedge \dots \wedge \omega_p
        \stackrel{\ref{prop_pr_ext_iii}}{=} (-1) \omega_1 \wedge \dots \wedge \omega_i \wedge \dots \wedge
        \omega_i \wedge \dots \wedge \omega_p \implies \\
        \implies 2\left( \omega_1 \wedge \dots \wedge \omega_p \right) = 0
        \stackrel{\car \k \neq 2}{\implies} \omega_1 \wedge \dots \wedge \omega_p = 0.
        \end{gather*}
        \item $\implies$ \\
        Suponemos que son l.d. y que $\omega_p = \sum_{j=1}^{p-1} \alpha_j \omega_j$:
        \[
        \omega_1 \wedge \dots \wedge \omega_p = \omega_1 \wedge \dots \wedge \omega_{p-1} \wedge \left(
        \sum_{j=1}^{p-1} \alpha_j \omega_j \right) \stackrel{\ref{prop_pr_ext_ii}}{=} \sum_{j=1}^{p-1}
        \alpha_j \left( \omega_1 \wedge \dots \wedge \omega_{p-1} \wedge \omega_j \right)
        \stackrel{\ref{prop_pr_ext_v}}{=} 0 \text{ } !!
        \]
        $\impliedby$ \\
        Sea $B=\{u_1,\dots,u_n\}$, $B^*=\{\omega_1, \dots, \omega_p,\overbrace{\omega_{p+1},\dots, \dots,
		    \omega_n}^{\text{Steinitz}} \}$ la base dual de $B$, tenemos que:
        \[
        \left( \omega_1 \wedge \dots \wedge \omega_p \right) \left( u_1,\dots,u_p \right) \stackrel{\ref{prop_pr_ext_iv}}{=}
        \det\left( \omega_j \left(u_i \right) \right) = 
        \begin{vmatrix}
            1 & 0 & \dots & 0 \\
            0 & \ddots & \ddots & \vdots \\
            \vdots & \ddots & \ddots & 0 \\
            0 & \dots & 0 & 1
        \end{vmatrix}
        = 1 \neq 0.
        \]
    \end{enumerate}
\end{proof}


\begin{obs} \label{obs_2_pr_ext}
    En el caso particular de $B=\left\{ e_1, \dots e_n \right\}$
    base de $\E$,
    \[
        \begin{aligned}
            I=\left\{ i_1, \dots i_p \right\}, \hspace{0.15cm}
            1 \leq i_1 < \dots < i_p \leq n ; \\
            J=\left\{ j_1, \dots j_p \right\}, \hspace{0.15cm}
            1 \leq j_1 < \dots < j_p \leq n .
        \end{aligned}
    \]
    \[
        \varepsilon _{IJ} = \left( e_{i_1}^* \wedge \cdots \wedge e_{i_p}^* \right) \left( e_{j_1}, \cdots , e_{j_p} \right)= \left\{ \begin{array}{ll}
             1 & \text{si } I=J \\
             0 & \text{en el caso contrario}
        \end{array}\right.
    \]
    En efecto, si $\hspace{0,1cm}\exists j_k \in J$, $j_k \notin I$, entonces en virtud del cálculo dado por \ref{prop_pr_ext_iv} (\ref{prop_pr_ext}), en la posición $k$ hay una fila de ceros. \\
    Por otro lado, si $I=J$ entonces tenemos el determinante de la matriz identidad. \\
    Observemos también que si $I=J$ pero no están ordenadas crecientemente, $\varepsilon _{IJ}=\pm1$ en función de las permutaciones que ordenan estos conjuntos.
\end{obs}

\begin{thm}
    Sea $\E$ un $\k$-e.v. de $\dim n$, sea $B=\left\{
    e_1, \dots , e_n\right\}$ y sea $p\leq n$.
    \begin{enumerate}[i)]
        \item \label{thm_it1}
            $ \dim A_p\left( \E \right) = \binom{n}{p}. $
        \item \label{thm_it2}
            Una base de $A_p$ es $\tilde{B} = \left\{e_{i_1}^* \wedge \cdots \wedge e_{i_p}^* \right\}$, $i\leq i_1 < \cdots < i_p \leq n$.
        \item 
        	Si $w \in A_p\left( \E \right)$, las coordenadas de $w$ en la base anterior son:
        	\[
        		\left( w \left( e_{i_1}, \dots , e_{i_p} \right) \right)_{1\leq i_1 < \cdots < i_p \leq n}
        	\]
    \end{enumerate}
\end{thm}
\begin{proof}
	\begin{enumerate}[i)]
		\item[]
		\item 
			$\ref{thm_it2} \implies \ref{thm_it1}$ trivialmente.
		\item 
			\underline{L.I.}: Sea
			\[
				w=\sum_{\substack{I=\left\{ i_1, \dots , i_p \right\} \\ 1\leq i_1 < \cdots < i_p \leq n}} \alpha _I e_{i_1}^* \wedge \cdots \wedge e_{i_p}^*=0.
			\]
			Sea $I_0=\left\{ i_1^0, \dots , i_p^0 \right\} $ con $1\leq i_1^0 < \cdots < i_p^0 \leq n$ cualesquiera. Entonces, 
			\[
				0=w \left( e_{i_1^0} , \dots , e_{i_p^0} \right)= \sum_{\substack{I=\left\{ i_1, \dots , i_p \right\} \\ 1\leq i_1 < \cdots < i_p \leq n}} \alpha _I e_{i_1}^* \wedge \cdots \wedge e_{i_p}^* \left( e_{i_1^0} , \dots , e_{i_p^0} \right) \stackrel{\ref{obs_2_pr_ext}} = \alpha _ {I_0}.
			\]
			\underline{Generadores}:
			\[
				A_p \left( \E \right) = A \left( T_p \left( \E \right) \right) = A_p \left( \left[ e_{i_1} \otimes \cdots \otimes e_{i_p} \right]_I \right) = \left[ \left\{ e_{i_1} \wedge \cdots \wedge e_{i_p} \right\}_I \right] \stackrel{\ref{prop_pr_ext_ii}}= \left[ \left\{ e_{i_1} \wedge \cdots \wedge e_{i_p} \right\}_{I\text{ ordenado}} \right].
			\]
		\item
			Sea $w \in A_p\left(\E\right)$. Consideremos
			\[
				\tilde{w}=\sum_{\substack{I=\left\{ i_1, \dots , i_p \right\} \\ 1\leq i_1 < \cdots < i_p \leq n}} \left( w \left( e_{i_1} , \dots , e_{i_p} \right) \right) e_{i_1}^* \wedge \cdots \wedge e_{i_p}^*.
			\]
			Veremos que $\tilde{w}=w$. Como tensores, basta ver que coinciden sobre vectores $\left( e_{j_1}, \dots , e_{j_p} \right)$. Como ambos son alternados, podemos suponer $1 \leq j_1  < \cdots < j_p \leq n$. Entonces,
			\[
				\tilde{w} \left( e_{j_1}, \dots , e_{j_p} \right) = w \left( e_{j_1}, \dots , e_{j_p} \right).
			\]
		
	\end{enumerate}
\end{proof}

\begin{example}
	Sea $\E=\real^3$ y $B = \setb{e_1,e_2,e_3}$
	\begin{center}
		\begin{tabular}{|c|c|c|c|}
			\hline
			Espacio & $\dim$ & base & coordenadas \\
			\hline \hline
			$A_1(\E)= T_1(\E) (=\E^*)$ & 3 & $\setb{e_1^*,e_2^*,e_3^*}$ &
			$w = (a_1,b_1,c_1)$ \\ \hline $A_2(\E) \subseteq T_2(\E)$ & 3 &
			$\setb{e_1^* \wedge e_2^*, e_1^* \wedge e_3^*, e_2^* \wedge e_3^*}$ &
			$\begin{aligned}
				w_1 \wedge w_2 =& (w_1 \wedge w_2)(e_1,e_2) e_1^* \wedge e_2^* + \\
				 & (w_1 \wedge w_2)(e_1,e_3) e_1^* \wedge e_3^* + \\
				 & (w_1 \wedge w_2)(e_2,e_3) e_2^* \wedge e_3^*
			\end{aligned}$ \\
			\hline $A_3(\E) \subseteq T_3(\E)$ & 1 &
			$\setb{e_1^* \wedge e_2^* \wedge e_3^*}$ &
			$t =(w_1 \wedge w_2 \wedge w_3) = (t(e_1,e_2,e_3))$ \\ \hline
		\end{tabular}
	\end{center}
	Además, si $w_1=(a_1,b_1,c_2)_{B^*}$, $w_2=(a_2,b_2,c_2)_{B^*}$ y
	$w_3=(a_3,b_3,c_3)_{B^*}$
	\[
		w_1 \wedge w_2 = \determinant{w_1(e_1) & w_1(e_2) \\ w_2(e_1) & w_2(e_2)}
		(e_1^* \wedge e_2^*) + \determinant{w_1(e_1) & w_1(e_3) \\ w_2(e_1) & w_2(e_3)}
        (e_1^* \wedge e_3^*) + \determinant{w_1(e_2) & w_1(e_3) \\ w_2(e_2) & w_2(e_3)}
        (e_2^* \wedge e_3^*) =
    \]
    \[
		\determinant{a_1 & b_1 \\ a_2 & b_2}(e_1^* \wedge e_2^*) +
		\determinant{a_1 & b_1 \\ a_3 & b_3}(e_1^* \wedge e_3^*)
		+ \determinant{a_2 & b_2 \\ a_3 & b_3}(e_2^* \wedge e_3^*)
	\]
	\[
		w_1 \wedge w_2 \wedge w_3 = (w_1 \wedge w_2 \wedge w_3)	(e_1,e_2,e_3)
		(e_1^* \wedge e_2^* \wedge e_3^*) =
		\determinant{a_1 & b_1 & c_1 \\ a_2 & b_2 & c_2 \\ a_3 & b_3 & c_3}
		(e_1^* \wedge e_2^* \wedge e_3^*)
	\]
\end{example}

\begin{obs}
	De forma análoga, podemos hacer el producto exterior de tensores 1-contravariantes $\left( T^1 \left( \E \right) = \E \right)$ 
	y obtendremos $A^p\left( \E \right)$ con dimensión $\binom{n}{p}$ y base \break $\left\{ e_{i_1} 
	\wedge \cdots \wedge e_{i_p} \right\}_{1 \leq i_1 < \cdots < i_p \leq n}$ si $B=\left\{e_i, \dots , e_n\right\}$ es base de $\E$.
\end{obs}

\begin{defi}
	Sea $\E$ un $\k$-ev. y sean $f \in A_p(\E)$ y $g \in A_q(\E)$. Definimos el
	producto exterior de $f$ y $g$ como
	\[
		f \wedge g = \frac{(p+q)!}{p!q!}A(f \otimes g)
	\]
\end{defi}
\begin{prop}
	Sea $\E$ un $\k$-ev. y sean $f \in A_p(\E)$, $g \in A_q(\E)$ y $h \in A_r(\E)$.
	\begin{enumerate}[i)]
		\item $(f \wedge g) \wedge h = f \wedge (g \wedge h) = f \wedge g \wedge h$
		\item $f \wedge g = (-1)^{pq} g \wedge f$
		\item $\wedge$ es lineal en cada factor.
	\end{enumerate}
\end{prop}
\begin{obs}
	$w_1 \wedge w_2 \wedge \cdots \wedge w_r = w_1 \wedge (w_2 \wedge (
	\cdots w_{r-1} \wedge (w_r)))$
\end{obs}
\begin{example}
	Sea $\E = \real ^4$, $B = \setb{e_1,e_2,e_3,e_4}$ y $f,g \in A_2(\E)$
	\begin{itemize}
		\item $f = e_1^* \wedge e_2^* + e_2^* \wedge e_3^* + e_3^* \wedge e_4^*$
		\item $g = e_1^* \wedge e_2^* + e_1^* \wedge e_3^*$
	\end{itemize}
	\begin{gather*}
		f \wedge g = (e_1^* \wedge e_2^* + e_2^* \wedge e_3^* + e_3^* \wedge e_4^*)
		\wedge (e_1^* \wedge e_2^* + e_1^* \wedge e_3^*) = \\
		 = \cancelto{0}{e_1^* \wedge e_2^* \wedge e_1^* \wedge e_2^*} + 
		 \cancelto{0}{e_1^* \wedge e_2^* \wedge e_1^* \wedge e_3^*} +
		 \cancelto{0}{e_2^* \wedge e_3^* \wedge e_1^* \wedge e_2^*} +
		 \cancelto{0}{e_2^* \wedge e_3^* \wedge e_1^* \wedge e_3^*} + \\
		 + e_3^* \wedge e_4^* \wedge e_1^* \wedge e_2^* +
		 \cancelto{0}{e_3^* \wedge e_4^* \wedge e_1^* \wedge e_3^*} =
		 e_3^* \wedge e_4^* \wedge e_1^* \wedge e_2^*
	\end{gather*}
\end{example}
